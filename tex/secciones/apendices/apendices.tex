
\makeatletter
\def\PY@reset{\let\PY@it=\relax \let\PY@bf=\relax%
    \let\PY@ul=\relax \let\PY@tc=\relax%
    \let\PY@bc=\relax \let\PY@ff=\relax}
\def\PY@tok#1{\csname PY@tok@#1\endcsname}
\def\PY@toks#1+{\ifx\relax#1\empty\else%
    \PY@tok{#1}\expandafter\PY@toks\fi}
\def\PY@do#1{\PY@bc{\PY@tc{\PY@ul{%
    \PY@it{\PY@bf{\PY@ff{#1}}}}}}}
\def\PY#1#2{\PY@reset\PY@toks#1+\relax+\PY@do{#2}}

\expandafter\def\csname PY@tok@gd\endcsname{\def\PY@tc##1{\textcolor[rgb]{0.63,0.00,0.00}{##1}}}
\expandafter\def\csname PY@tok@gu\endcsname{\let\PY@bf=\textbf\def\PY@tc##1{\textcolor[rgb]{0.50,0.00,0.50}{##1}}}
\expandafter\def\csname PY@tok@gt\endcsname{\def\PY@tc##1{\textcolor[rgb]{0.00,0.25,0.82}{##1}}}
\expandafter\def\csname PY@tok@gs\endcsname{\let\PY@bf=\textbf}
\expandafter\def\csname PY@tok@gr\endcsname{\def\PY@tc##1{\textcolor[rgb]{1.00,0.00,0.00}{##1}}}
\expandafter\def\csname PY@tok@cm\endcsname{\let\PY@it=\textit\def\PY@tc##1{\textcolor[rgb]{0.25,0.50,0.50}{##1}}}
\expandafter\def\csname PY@tok@vg\endcsname{\def\PY@tc##1{\textcolor[rgb]{0.10,0.09,0.49}{##1}}}
\expandafter\def\csname PY@tok@m\endcsname{\def\PY@tc##1{\textcolor[rgb]{0.40,0.40,0.40}{##1}}}
\expandafter\def\csname PY@tok@mh\endcsname{\def\PY@tc##1{\textcolor[rgb]{0.40,0.40,0.40}{##1}}}
\expandafter\def\csname PY@tok@go\endcsname{\def\PY@tc##1{\textcolor[rgb]{0.50,0.50,0.50}{##1}}}
\expandafter\def\csname PY@tok@ge\endcsname{\let\PY@it=\textit}
\expandafter\def\csname PY@tok@vc\endcsname{\def\PY@tc##1{\textcolor[rgb]{0.10,0.09,0.49}{##1}}}
\expandafter\def\csname PY@tok@il\endcsname{\def\PY@tc##1{\textcolor[rgb]{0.40,0.40,0.40}{##1}}}
\expandafter\def\csname PY@tok@cs\endcsname{\let\PY@it=\textit\def\PY@tc##1{\textcolor[rgb]{0.25,0.50,0.50}{##1}}}
\expandafter\def\csname PY@tok@cp\endcsname{\def\PY@tc##1{\textcolor[rgb]{0.74,0.48,0.00}{##1}}}
\expandafter\def\csname PY@tok@gi\endcsname{\def\PY@tc##1{\textcolor[rgb]{0.00,0.63,0.00}{##1}}}
\expandafter\def\csname PY@tok@gh\endcsname{\let\PY@bf=\textbf\def\PY@tc##1{\textcolor[rgb]{0.00,0.00,0.50}{##1}}}
\expandafter\def\csname PY@tok@ni\endcsname{\let\PY@bf=\textbf\def\PY@tc##1{\textcolor[rgb]{0.60,0.60,0.60}{##1}}}
\expandafter\def\csname PY@tok@nl\endcsname{\def\PY@tc##1{\textcolor[rgb]{0.63,0.63,0.00}{##1}}}
\expandafter\def\csname PY@tok@nn\endcsname{\let\PY@bf=\textbf\def\PY@tc##1{\textcolor[rgb]{0.00,0.00,1.00}{##1}}}
\expandafter\def\csname PY@tok@no\endcsname{\def\PY@tc##1{\textcolor[rgb]{0.53,0.00,0.00}{##1}}}
\expandafter\def\csname PY@tok@na\endcsname{\def\PY@tc##1{\textcolor[rgb]{0.49,0.56,0.16}{##1}}}
\expandafter\def\csname PY@tok@nb\endcsname{\def\PY@tc##1{\textcolor[rgb]{0.00,0.50,0.00}{##1}}}
\expandafter\def\csname PY@tok@nc\endcsname{\let\PY@bf=\textbf\def\PY@tc##1{\textcolor[rgb]{0.00,0.00,1.00}{##1}}}
\expandafter\def\csname PY@tok@nd\endcsname{\def\PY@tc##1{\textcolor[rgb]{0.67,0.13,1.00}{##1}}}
\expandafter\def\csname PY@tok@ne\endcsname{\let\PY@bf=\textbf\def\PY@tc##1{\textcolor[rgb]{0.82,0.25,0.23}{##1}}}
\expandafter\def\csname PY@tok@nf\endcsname{\def\PY@tc##1{\textcolor[rgb]{0.00,0.00,1.00}{##1}}}
\expandafter\def\csname PY@tok@si\endcsname{\let\PY@bf=\textbf\def\PY@tc##1{\textcolor[rgb]{0.73,0.40,0.53}{##1}}}
\expandafter\def\csname PY@tok@s2\endcsname{\def\PY@tc##1{\textcolor[rgb]{0.73,0.13,0.13}{##1}}}
\expandafter\def\csname PY@tok@vi\endcsname{\def\PY@tc##1{\textcolor[rgb]{0.10,0.09,0.49}{##1}}}
\expandafter\def\csname PY@tok@nt\endcsname{\let\PY@bf=\textbf\def\PY@tc##1{\textcolor[rgb]{0.00,0.50,0.00}{##1}}}
\expandafter\def\csname PY@tok@nv\endcsname{\def\PY@tc##1{\textcolor[rgb]{0.10,0.09,0.49}{##1}}}
\expandafter\def\csname PY@tok@s1\endcsname{\def\PY@tc##1{\textcolor[rgb]{0.73,0.13,0.13}{##1}}}
\expandafter\def\csname PY@tok@sh\endcsname{\def\PY@tc##1{\textcolor[rgb]{0.73,0.13,0.13}{##1}}}
\expandafter\def\csname PY@tok@sc\endcsname{\def\PY@tc##1{\textcolor[rgb]{0.73,0.13,0.13}{##1}}}
\expandafter\def\csname PY@tok@sx\endcsname{\def\PY@tc##1{\textcolor[rgb]{0.00,0.50,0.00}{##1}}}
\expandafter\def\csname PY@tok@bp\endcsname{\def\PY@tc##1{\textcolor[rgb]{0.00,0.50,0.00}{##1}}}
\expandafter\def\csname PY@tok@c1\endcsname{\let\PY@it=\textit\def\PY@tc##1{\textcolor[rgb]{0.25,0.50,0.50}{##1}}}
\expandafter\def\csname PY@tok@kc\endcsname{\let\PY@bf=\textbf\def\PY@tc##1{\textcolor[rgb]{0.00,0.50,0.00}{##1}}}
\expandafter\def\csname PY@tok@c\endcsname{\let\PY@it=\textit\def\PY@tc##1{\textcolor[rgb]{0.25,0.50,0.50}{##1}}}
\expandafter\def\csname PY@tok@mf\endcsname{\def\PY@tc##1{\textcolor[rgb]{0.40,0.40,0.40}{##1}}}
\expandafter\def\csname PY@tok@err\endcsname{\def\PY@bc##1{\setlength{\fboxsep}{0pt}\fcolorbox[rgb]{1.00,0.00,0.00}{1,1,1}{\strut ##1}}}
\expandafter\def\csname PY@tok@kd\endcsname{\let\PY@bf=\textbf\def\PY@tc##1{\textcolor[rgb]{0.00,0.50,0.00}{##1}}}
\expandafter\def\csname PY@tok@ss\endcsname{\def\PY@tc##1{\textcolor[rgb]{0.10,0.09,0.49}{##1}}}
\expandafter\def\csname PY@tok@sr\endcsname{\def\PY@tc##1{\textcolor[rgb]{0.73,0.40,0.53}{##1}}}
\expandafter\def\csname PY@tok@mo\endcsname{\def\PY@tc##1{\textcolor[rgb]{0.40,0.40,0.40}{##1}}}
\expandafter\def\csname PY@tok@kn\endcsname{\let\PY@bf=\textbf\def\PY@tc##1{\textcolor[rgb]{0.00,0.50,0.00}{##1}}}
\expandafter\def\csname PY@tok@mi\endcsname{\def\PY@tc##1{\textcolor[rgb]{0.40,0.40,0.40}{##1}}}
\expandafter\def\csname PY@tok@gp\endcsname{\let\PY@bf=\textbf\def\PY@tc##1{\textcolor[rgb]{0.00,0.00,0.50}{##1}}}
\expandafter\def\csname PY@tok@o\endcsname{\def\PY@tc##1{\textcolor[rgb]{0.40,0.40,0.40}{##1}}}
\expandafter\def\csname PY@tok@kr\endcsname{\let\PY@bf=\textbf\def\PY@tc##1{\textcolor[rgb]{0.00,0.50,0.00}{##1}}}
\expandafter\def\csname PY@tok@s\endcsname{\def\PY@tc##1{\textcolor[rgb]{0.73,0.13,0.13}{##1}}}
\expandafter\def\csname PY@tok@kp\endcsname{\def\PY@tc##1{\textcolor[rgb]{0.00,0.50,0.00}{##1}}}
\expandafter\def\csname PY@tok@w\endcsname{\def\PY@tc##1{\textcolor[rgb]{0.73,0.73,0.73}{##1}}}
\expandafter\def\csname PY@tok@kt\endcsname{\def\PY@tc##1{\textcolor[rgb]{0.69,0.00,0.25}{##1}}}
\expandafter\def\csname PY@tok@ow\endcsname{\let\PY@bf=\textbf\def\PY@tc##1{\textcolor[rgb]{0.67,0.13,1.00}{##1}}}
\expandafter\def\csname PY@tok@sb\endcsname{\def\PY@tc##1{\textcolor[rgb]{0.73,0.13,0.13}{##1}}}
\expandafter\def\csname PY@tok@k\endcsname{\let\PY@bf=\textbf\def\PY@tc##1{\textcolor[rgb]{0.00,0.50,0.00}{##1}}}
\expandafter\def\csname PY@tok@se\endcsname{\let\PY@bf=\textbf\def\PY@tc##1{\textcolor[rgb]{0.73,0.40,0.13}{##1}}}
\expandafter\def\csname PY@tok@sd\endcsname{\let\PY@it=\textit\def\PY@tc##1{\textcolor[rgb]{0.73,0.13,0.13}{##1}}}

\def\PYZbs{\char`\\}
\def\PYZus{\char`\_}
\def\PYZob{\char`\{}
\def\PYZcb{\char`\}}
\def\PYZca{\char`\^}
\def\PYZam{\char`\&}
\def\PYZlt{\char`\<}
\def\PYZgt{\char`\>}
\def\PYZsh{\char`\#}
\def\PYZpc{\char`\%}
\def\PYZdl{\char`\$}
\def\PYZti{\char`\~}
% for compatibility with earlier versions
\def\PYZat{@}
\def\PYZlb{[}
\def\PYZrb{]}
\makeatother


\subsection{Apéndice A: Enunciado del TP}


{\bf Introducci\'on}

El reconocimiento \'optico de caracteres (OCR, por sus siglas en ingl\'es) es el proceso por el cual se traducen o convierten im\'agenes de d\'igitos o caracteres (sean \'estos manuscritos o de alguna tipograf\'ia especial) a un formato representable en nuestra computadora (por ejemplo, ASCII). Esta tarea puede ser m\'as sencilla (por ejemplo, cuando tratamos de determinar el texto escrito en una versi\'on escaneada a buena resoluci\'on de un libro) o tornarse casi imposible (recetas indescifrables de m\'edicos, algunos parciales manuscritos de alumnos de m\'etodos num\'ericos, etc).

El objetivo del trabajo pr\'actico es implementar un m\'etodo de reconocimiento de d\'igitos manuscritos basado en la descomposici\'on en valores singulares, y analizar emp\'iricamente los par\'ametros principales del m\'etodo.

Como instancias de entrenamiento, se tiene un conjunto de $n$ im\'agenes de d\'igitos ma\-nus\-cri\-tos en escala de grises del mismo tama\~no y resoluci\'on (varias im\'agenes de cada d\'igito). Cada una de estas im\'agenes sabemos a qu\'e d\'igito se corresponde.
En este trabajo consideraremos la popular base de datos MNIST, utilizada como referencia en esta \'area de investigaci\'on\footnote{\texttt{http://yann.lecun.com/exdb/mnist/}}. 

Para $i = 1,\ldots, n$, sea $x_i \in \real^{m}$ la $i$-\'esima imagen de nuestra base de datos almacenada por filas en un vector, y sea $\mu = (x_1 + \ldots + x_n)/n$ el promedio de las im\'agenes. Definimos $X\in\real^{n\times m}$ como la matriz que contiene en la $i$-\'esima fila al vector $(x_i - \mu)^{t}/\sqrt{n-1}$, y $$X=U \Sigma V^t$$ a su descomposici\'on en valores singulares, con $U\in\real^{n\times n}$ y $V\in\real^{m\times m}$ matrices ortogonales, y $\Sigma\in\real^{n\times m}$ la matriz diagonal conteniendo en la posici\'on $(i,i)$ al $i$-\'esimo valor singular $\sigma_i$.
Siendo $v_i$ la columna $i$ de $V$, definimos para $i = 1,\ldots,n$ la \textsl{transformaci\'on caracter\'istica} del d\'igito $x_{i}$ como el vector $\mathbf{tc}(x_i) = (v_1^t\, x_i, v_2^t\, x_i,\ldots,v_k^t\, x_i) \in\real^k$, donde $k \in\{1,\ldots,m\}$ es un par\'ametro de la implementaci\'on. Este proceso corresponde a extraer las $k$ primeras \textit{componentes principales} de cada imagen. La intenci\'on es que $\mathbf{tc}(x_i)$ resuma la informaci\'on m\'as relevante de la imagen, descartando los detalles o las zonas que no aportan rasgos distintivos.


Dada una nueva imagen $x$ de un d\'igito manuscrito, que no se encuentra en el conjunto inicial de im\'agenes de entrenamiento, el problema de reconocimiento consiste en determinar a qu\'e d\'igito corresponde. Para esto, se calcula $\mathbf{tc}(x)$ y se compara con $\mathbf{tc}(x_i)$, para $i = 1,\ldots, n$.


{\bf Enunciado}

Se pide implementar un programa que lea desde archivos las im\'agenes de entrenamiento de distintos d\'igitos manuscritos y que, utilizando la descomposici\'on en valores singulares, se calcule la transformaci\'on caracter\'istica de acuerdo con la descripci\'on anterior. Para ello se deber\'a implementar alg\'un m\'etodo de estimaci\'on de autovalores/autovectores. Dada una nueva imagen de un d\'igito manuscrito, el programa deber\'a determinar a qu\'e d\'igito co\-rres\-pon\-de.
El formato de los archivos de entrada y salida queda a elecci\'on del grupo. Si no usan un entorno de desarrollo que incluya bibliotecas para la lectura de archivos de im\'agenes, sugerimos que utilicen im\'agenes en formato \textsc{Raw}. 


Se deber\'an realizar experimentos para medir la efectividad del reconocimiento, analizando tanto la influencia de la cantidad $k$ de componentes principales seleccionadas como la influencia de la precisi\'on en el c\'alculo de los autovalores.





\subsection{Apéndice B}

\subsubsection{algorithms.cpp}
\begin{Verbatim}[commandchars=\\\{\}]
\PY{c+cp}{\PYZsh{}}\PY{c+cp}{include \PYZlt{}cmath\PYZgt{}}
\PY{c+cp}{\PYZsh{}}\PY{c+cp}{include \PYZlt{}vector\PYZgt{}}
\PY{c+cp}{\PYZsh{}}\PY{c+cp}{include \PYZlt{}algorithm\PYZgt{}}
\PY{c+cp}{\PYZsh{}}\PY{c+cp}{include \PYZlt{}iostream\PYZgt{}}
\PY{k}{using} \PY{k}{namespace} \PY{n}{std}\PY{p}{;}

\PY{c+cp}{\PYZsh{}}\PY{c+cp}{include "..}\PY{c+cp}{/}\PY{c+cp}{lib}\PY{c+cp}{/}\PY{c+cp}{commons.h"}
\PY{c+cp}{\PYZsh{}}\PY{c+cp}{include "algorithms.h"}
\PY{c+cp}{\PYZsh{}}\PY{c+cp}{include "mmatrix.h"}

\PY{c+cp}{\PYZsh{}}\PY{c+cp}{define	CONVERGENCE\PYZus{}NOT\PYZus{}ATTAINED\PYZus{}POWER\PYZus{}MTH(it, drch, dlt)		\PYZbs{}}
\PY{c+cp}{("El método de la potencia no convergió después de la máxima cantidad de iteraciones (" + int2str(it) + "). Luego de la última iteración, el cambio en dirección de la estimación era de " + double2str(drch) + ", ante un valor máximo aceptado de: " + double2str(dlt))}

\PY{c+cp}{\PYZsh{}}\PY{c+cp}{define MAX\PYZus{}ITERATIONS	10000}

\PY{c+cp}{\PYZsh{}}\PY{c+cp}{define	NUM\PYZus{}DIGITS	10}

\PY{c+c1}{//	//	//	//}

\PY{n}{MMatrix} \PY{n}{compute\PYZus{}mean\PYZus{}row}\PY{p}{(}\PY{n}{MMatrix}\PY{o}{\PYZam{}} \PY{n}{mat}\PY{p}{)}\PY{p}{;}
\PY{k+kt}{void} \PY{n}{extended\PYZus{}power\PYZus{}method}\PY{p}{(}\PY{n}{MMatrix}\PY{o}{\PYZam{}} \PY{n}{A}\PY{p}{,} \PY{k+kt}{int} \PY{n}{k}\PY{p}{,} \PY{k+kt}{double} \PY{n}{delta}\PY{p}{,} \PY{n}{MMatrix}\PY{o}{\PYZam{}} \PY{n}{V}\PY{p}{)}\PY{p}{;}
\PY{k+kt}{void} \PY{n}{power\PYZus{}method}\PY{p}{(}\PY{n}{MMatrix}\PY{o}{\PYZam{}} \PY{n}{A}\PY{p}{,} \PY{k+kt}{double} \PY{n}{delta}\PY{p}{,} \PY{n}{MMatrix}\PY{o}{\PYZam{}} \PY{n}{v}\PY{p}{)}\PY{p}{;}
\PY{k+kt}{void} \PY{n}{sort\PYZus{}eigenvectors}\PY{p}{(}\PY{n}{MMatrix}\PY{o}{\PYZam{}} \PY{n}{V}\PY{p}{,} \PY{n}{vector}\PY{o}{\PYZlt{}}\PY{k+kt}{double}\PY{o}{\PYZgt{}}\PY{o}{\PYZam{}} \PY{n}{eigenvalues}\PY{p}{)}\PY{p}{;}
\PY{k+kt}{double} \PY{n}{compute\PYZus{}raleygh\PYZus{}quotient}\PY{p}{(}\PY{n}{MMatrix}\PY{o}{\PYZam{}} \PY{n}{v}\PY{p}{,} \PY{n}{MMatrix}\PY{o}{\PYZam{}} \PY{n}{A}\PY{p}{)}\PY{p}{;}
\PY{k+kt}{double} \PY{n}{norm}\PY{p}{(}\PY{n}{MMatrix}\PY{o}{\PYZam{}} \PY{n}{m}\PY{p}{)}\PY{p}{;}

\PY{c+c1}{//	//	//	//}

\PY{n}{MMatrix} \PY{n}{normalize\PYZus{}variables}\PY{p}{(}\PY{n}{MMatrix}\PY{o}{\PYZam{}} \PY{n}{mat}\PY{p}{)}
\PY{p}{\PYZob{}}
	\PY{n}{MMatrix} \PY{n}{mean\PYZus{}row} \PY{o}{=} \PY{n}{compute\PYZus{}mean\PYZus{}row}\PY{p}{(}\PY{n}{mat}\PY{p}{)}\PY{p}{;}

	\PY{n}{MMatrix} \PY{n}{norm\PYZus{}mat}\PY{p}{(}\PY{n}{mat}\PY{p}{.}\PY{n}{rows}\PY{p}{(}\PY{p}{)}\PY{p}{,} \PY{n}{mat}\PY{p}{.}\PY{n}{cols}\PY{p}{(}\PY{p}{)}\PY{p}{)}\PY{p}{;}
	\PY{n}{foreach\PYZus{}a\PYZus{}ij}\PY{p}{(}\PY{n}{norm\PYZus{}mat}\PY{p}{,} \PY{n}{a\PYZus{}ij} \PY{o}{=} \PY{n}{mat}\PY{p}{(}\PY{n}{i}\PY{p}{,}\PY{n}{j}\PY{p}{)} \PY{o}{-} \PY{n}{mean\PYZus{}row}\PY{p}{(}\PY{n}{j}\PY{p}{)}\PY{p}{)}\PY{p}{;}

	\PY{k}{return} \PY{n}{norm\PYZus{}mat}\PY{p}{;}
\PY{p}{\PYZcb{}}

\PY{k+kt}{void} \PY{n}{compute\PYZus{}covariance\PYZus{}matrix}\PY{p}{(}\PY{n}{MMatrix}\PY{o}{\PYZam{}} \PY{n}{mat}\PY{p}{,} \PY{n}{MMatrix}\PY{o}{\PYZam{}} \PY{n}{cov\PYZus{}mat}\PY{p}{)}
\PY{p}{\PYZob{}}
	\PY{n}{MMatrix} \PY{n}{norm\PYZus{}mat} \PY{o}{=} \PY{n}{normalize\PYZus{}variables}\PY{p}{(}\PY{n}{mat}\PY{p}{)}\PY{p}{;}

	\PY{n}{cov\PYZus{}mat}\PY{p}{.}\PY{n}{set\PYZus{}size}\PY{p}{(}\PY{n}{norm\PYZus{}mat}\PY{p}{.}\PY{n}{cols}\PY{p}{(}\PY{p}{)}\PY{p}{,}\PY{n}{norm\PYZus{}mat}\PY{p}{.}\PY{n}{cols}\PY{p}{(}\PY{p}{)}\PY{p}{)}\PY{p}{;}
	\PY{k+kt}{double} \PY{n}{denominator} \PY{o}{=} \PY{l+m+mf}{1.0}\PY{o}{/}\PY{p}{(}\PY{n}{norm\PYZus{}mat}\PY{p}{.}\PY{n}{rows}\PY{p}{(}\PY{p}{)} \PY{o}{-} \PY{l+m+mi}{1}\PY{p}{)}\PY{p}{;}

	\PY{n}{foreach\PYZus{}a\PYZus{}ij\PYZus{}lower\PYZus{}triangular}\PY{p}{(}\PY{n}{cov\PYZus{}mat}\PY{p}{,}\PY{p}{\PYZob{}}
		\PY{n}{cov\PYZus{}mat}\PY{p}{(}\PY{n}{i}\PY{p}{,}\PY{n}{j}\PY{p}{)} \PY{o}{=} \PY{n}{cov\PYZus{}mat}\PY{p}{(}\PY{n}{j}\PY{p}{,}\PY{n}{i}\PY{p}{)} \PY{o}{=} \PY{p}{(}\PY{n}{MMatrix}\PY{o}{:}\PY{o}{:}\PY{n}{dot\PYZus{}col\PYZus{}col}\PY{p}{(}\PY{n}{norm\PYZus{}mat}\PY{p}{,} \PY{n}{i}\PY{p}{,} \PY{n}{norm\PYZus{}mat}\PY{p}{,} \PY{n}{j}\PY{p}{)} \PYZbs{}
		\PY{o}{*} \PY{n}{denominator}\PY{p}{)}\PY{p}{;}
	\PY{p}{\PYZcb{}}\PY{p}{)}\PY{p}{;}
\PY{p}{\PYZcb{}}

\PY{n}{MMatrix} \PY{n}{compute\PYZus{}transformation\PYZus{}matrix}\PY{p}{(}\PY{n}{MMatrix} \PY{n}{A}\PY{p}{,} \PY{k+kt}{int} \PY{n}{num\PYZus{}eigenvectors}\PY{p}{,} \PY{k+kt}{double} \PY{n}{delta}\PY{p}{,} \PY{k+kt}{bool} \PY{n}{verbose}\PY{p}{)}
\PY{p}{\PYZob{}}
	\PY{n}{MMatrix} \PY{n}{V}\PY{p}{(}\PY{n}{A}\PY{p}{.}\PY{n}{rows}\PY{p}{(}\PY{p}{)}\PY{p}{,} \PY{n}{num\PYZus{}eigenvectors}\PY{p}{)}\PY{p}{;}
	\PY{n}{vector}\PY{o}{\PYZlt{}}\PY{k+kt}{double}\PY{o}{\PYZgt{}} \PY{n}{eigenvalues}\PY{p}{;}
	\PY{k}{for} \PY{p}{(}\PY{k+kt}{int} \PY{n}{k} \PY{o}{=} \PY{l+m+mi}{0}\PY{p}{;} \PY{n}{k} \PY{o}{\PYZlt{}} \PY{n}{num\PYZus{}eigenvectors}\PY{p}{;} \PY{o}{+}\PY{o}{+}\PY{n}{k}\PY{p}{)}
	\PY{p}{\PYZob{}}
		\PY{n}{BEGIN\PYZus{}TIMER}\PY{p}{(}\PY{p}{)}\PY{p}{;}

		\PY{n}{MMatrix} \PY{n}{v} \PY{o}{=} \PY{n}{power\PYZus{}method}\PY{p}{(}\PY{n}{A}\PY{p}{,} \PY{n}{delta}\PY{p}{)}\PY{p}{;}
		\PY{k+kt}{double} \PY{n}{lambda} \PY{o}{=} \PY{n}{compute\PYZus{}raleygh\PYZus{}quotient}\PY{p}{(}\PY{n}{v}\PY{p}{,} \PY{n}{A}\PY{p}{)}\PY{p}{;}
		\PY{n}{eigenvalues}\PY{p}{.}\PY{n}{push\PYZus{}back}\PY{p}{(}\PY{n}{lambda}\PY{p}{)}\PY{p}{;}

		\PY{c+cm}{/* deflation */}
		\PY{n}{foreach\PYZus{}a\PYZus{}ij}\PY{p}{(}\PY{n}{A}\PY{p}{,} \PY{n}{a\PYZus{}ij} \PY{o}{=} \PY{n}{a\PYZus{}ij} \PY{o}{-} \PY{n}{lambda} \PY{o}{*} \PY{n}{v}\PY{p}{(}\PY{n}{i}\PY{p}{)} \PY{o}{*} \PY{n}{v}\PY{p}{(}\PY{n}{j}\PY{p}{)} \PY{p}{)}\PY{p}{;}

		\PY{n}{foreach\PYZus{}v\PYZus{}i}\PY{p}{(}\PY{n}{v}\PY{p}{,}\PY{p}{\PYZob{}}
			\PY{n}{V}\PY{p}{(}\PY{n}{i}\PY{p}{,}\PY{n}{k}\PY{p}{)} \PY{o}{=} \PY{n}{v\PYZus{}i}\PY{p}{;}
		\PY{p}{\PYZcb{}}\PY{p}{)}\PY{p}{;}

		\PY{n}{PRINT\PYZus{}ON\PYZus{}VERBOSE}\PY{p}{(}\PY{l+s}{"}\PY{l+s}{Autovector número }\PY{l+s}{"} \PY{o}{+} \PY{n}{int2str}\PY{p}{(}\PY{n}{k}\PY{p}{)} \PY{o}{+} \PY{l+s}{"}\PY{l+s}{ calculado (}\PY{l+s}{"} \PY{o}{+} \PY{n}{int2str}\PY{p}{(}\PY{n}{MSECS\PYZus{}ELAPSED}\PY{p}{(}\PY{p}{)}\PY{p}{)} \PY{o}{+} \PY{l+s}{"}\PY{l+s}{ ms).}\PY{l+s}{"}\PY{p}{,} \PY{n}{verbose}\PY{p}{)}
	\PY{p}{\PYZcb{}}

	\PY{n}{sort\PYZus{}eigenvectors}\PY{p}{(}\PY{n}{V}\PY{p}{,} \PY{n}{eigenvalues}\PY{p}{)}\PY{p}{;}

	\PY{k}{return} \PY{n}{V}\PY{p}{;}
\PY{p}{\PYZcb{}}

\PY{n}{MMatrix} \PY{n}{power\PYZus{}method}\PY{p}{(}\PY{n}{MMatrix}\PY{o}{\PYZam{}} \PY{n}{A}\PY{p}{,} \PY{k+kt}{double} \PY{n}{delta}\PY{p}{)}
\PY{p}{\PYZob{}}
	\PY{n}{MMatrix} \PY{n}{v}\PY{p}{(}\PY{n}{A}\PY{p}{.}\PY{n}{rows}\PY{p}{(}\PY{p}{)}\PY{p}{,}\PY{l+m+mi}{1}\PY{p}{)}\PY{p}{;}
	\PY{n}{foreach\PYZus{}v\PYZus{}i}\PY{p}{(}\PY{n}{v}\PY{p}{,} \PY{n}{v\PYZus{}i} \PY{o}{=} \PY{p}{(}\PY{p}{(}\PY{k+kt}{double}\PY{p}{)}\PY{n}{rand}\PY{p}{(}\PY{p}{)}\PY{p}{)}\PY{o}{/}\PY{n}{RAND\PYZus{}MAX}\PY{p}{)}\PY{p}{;}
	\PY{n}{v} \PY{o}{/}\PY{o}{=} \PY{n}{norm}\PY{p}{(}\PY{n}{v}\PY{p}{)}\PY{p}{;}

	\PY{k+kt}{int} \PY{n}{iteration\PYZus{}count}\PY{p}{;}
	\PY{k+kt}{double} \PY{n}{direction\PYZus{}rate\PYZus{}of\PYZus{}change}\PY{p}{;}
	\PY{k}{for} \PY{p}{(}\PY{n}{iteration\PYZus{}count} \PY{o}{=} \PY{l+m+mi}{0}\PY{p}{;} \PY{n}{iteration\PYZus{}count} \PY{o}{\PYZlt{}} \PY{n}{MAX\PYZus{}ITERATIONS}\PY{p}{;} \PY{o}{+}\PY{o}{+}\PY{n}{iteration\PYZus{}count}\PY{p}{)}
	\PY{p}{\PYZob{}}
		\PY{n}{MMatrix} \PY{n}{y} \PY{o}{=} \PY{n}{A} \PY{o}{*} \PY{n}{v}\PY{p}{;}
		\PY{n}{y} \PY{o}{/}\PY{o}{=} \PY{n}{norm}\PY{p}{(}\PY{n}{y}\PY{p}{)}\PY{p}{;}

		\PY{n}{direction\PYZus{}rate\PYZus{}of\PYZus{}change} \PY{o}{=} \PY{l+m+mi}{1} \PY{o}{-} \PY{n}{abs}\PY{p}{(}\PY{n}{MMatrix}\PY{o}{:}\PY{o}{:}\PY{n}{dot}\PY{p}{(}\PY{n}{v}\PY{p}{,}\PY{n}{y}\PY{p}{)}\PY{p}{)}\PY{p}{;}
		\PY{k}{if}\PY{p}{(}\PY{n}{direction\PYZus{}rate\PYZus{}of\PYZus{}change} \PY{o}{\PYZlt{}}\PY{o}{=} \PY{n}{delta}\PY{p}{)} \PY{k}{break}\PY{p}{;}

		\PY{n}{v} \PY{o}{=} \PY{n}{y}\PY{p}{;}
	\PY{p}{\PYZcb{}}

	\PY{k}{if}\PY{p}{(}\PY{n}{iteration\PYZus{}count} \PY{o}{=}\PY{o}{=} \PY{n}{MAX\PYZus{}ITERATIONS}\PY{p}{)}
		\PY{n}{DISPLAY\PYZus{}ERROR\PYZus{}AND\PYZus{}EXIT}\PY{p}{(}\PY{n}{CONVERGENCE\PYZus{}NOT\PYZus{}ATTAINED\PYZus{}POWER\PYZus{}MTH}\PY{p}{(}\PY{n}{iteration\PYZus{}count}\PY{p}{,} \PY{n}{direction\PYZus{}rate\PYZus{}of\PYZus{}change}\PY{p}{,} \PY{n}{delta}\PY{p}{)}\PY{p}{)}\PY{p}{;}

	\PY{k}{return} \PY{n}{v}\PY{p}{;}
\PY{p}{\PYZcb{}}

\PY{k+kt}{bool} \PY{n}{compare\PYZus{}index\PYZus{}eigenvalue\PYZus{}pair}\PY{p}{(}\PY{n}{pair}\PY{o}{\PYZlt{}}\PY{k+kt}{int}\PY{p}{,}\PY{k+kt}{double}\PY{o}{\PYZgt{}} \PY{n}{a}\PY{p}{,} \PY{n}{pair}\PY{o}{\PYZlt{}}\PY{k+kt}{int}\PY{p}{,}\PY{k+kt}{double}\PY{o}{\PYZgt{}} \PY{n}{b}\PY{p}{)}
\PY{p}{\PYZob{}}
	\PY{k}{return} \PY{n}{a}\PY{p}{.}\PY{n}{second} \PY{o}{\PYZgt{}} \PY{n}{b}\PY{p}{.}\PY{n}{second}\PY{p}{;}
\PY{p}{\PYZcb{}}

\PY{k+kt}{void} \PY{n}{sort\PYZus{}eigenvectors}\PY{p}{(}\PY{n}{MMatrix}\PY{o}{\PYZam{}} \PY{n}{V}\PY{p}{,} \PY{n}{vector}\PY{o}{\PYZlt{}}\PY{k+kt}{double}\PY{o}{\PYZgt{}}\PY{o}{\PYZam{}} \PY{n}{eigenvalues}\PY{p}{)}
\PY{p}{\PYZob{}}
	\PY{n}{vector}\PY{o}{\PYZlt{}}\PY{n}{pair}\PY{o}{\PYZlt{}}\PY{k+kt}{int}\PY{p}{,}\PY{k+kt}{double}\PY{o}{\PYZgt{}} \PY{o}{\PYZgt{}} \PY{n}{index\PYZus{}eigenvalue\PYZus{}pairs}\PY{p}{;}
	\PY{k}{for} \PY{p}{(}\PY{k+kt}{int} \PY{n}{i} \PY{o}{=} \PY{l+m+mi}{0}\PY{p}{;} \PY{n}{i} \PY{o}{\PYZlt{}} \PY{n}{eigenvalues}\PY{p}{.}\PY{n}{size}\PY{p}{(}\PY{p}{)}\PY{p}{;} \PY{o}{+}\PY{o}{+}\PY{n}{i}\PY{p}{)}
		\PY{n}{index\PYZus{}eigenvalue\PYZus{}pairs}\PY{p}{.}\PY{n}{push\PYZus{}back}\PY{p}{(}\PY{n}{pair}\PY{o}{\PYZlt{}}\PY{k+kt}{int}\PY{p}{,}\PY{k+kt}{double}\PY{o}{\PYZgt{}}\PY{p}{(}\PY{n}{i}\PY{p}{,} \PY{n}{eigenvalues}\PY{p}{.}\PY{n}{at}\PY{p}{(}\PY{n}{i}\PY{p}{)}\PY{p}{)}\PY{p}{)}\PY{p}{;}

	\PY{n}{sort}\PY{p}{(}\PY{n}{index\PYZus{}eigenvalue\PYZus{}pairs}\PY{p}{.}\PY{n}{begin}\PY{p}{(}\PY{p}{)}\PY{p}{,} \PY{n}{index\PYZus{}eigenvalue\PYZus{}pairs}\PY{p}{.}\PY{n}{end}\PY{p}{(}\PY{p}{)}\PY{p}{,} \PY{n}{compare\PYZus{}index\PYZus{}eigenvalue\PYZus{}pair}\PY{p}{)}\PY{p}{;}

	\PY{n}{MMatrix} \PY{n}{sorted\PYZus{}V}\PY{p}{(}\PY{n}{V}\PY{p}{.}\PY{n}{rows}\PY{p}{(}\PY{p}{)}\PY{p}{,} \PY{n}{V}\PY{p}{.}\PY{n}{cols}\PY{p}{(}\PY{p}{)}\PY{p}{)}\PY{p}{;}
	\PY{k}{for}\PY{p}{(}\PY{k+kt}{int} \PY{n}{j} \PY{o}{=} \PY{l+m+mi}{0}\PY{p}{;} \PY{n}{j} \PY{o}{\PYZlt{}} \PY{n}{V}\PY{p}{.}\PY{n}{cols}\PY{p}{(}\PY{p}{)}\PY{p}{;} \PY{o}{+}\PY{o}{+}\PY{n}{j}\PY{p}{)}
	\PY{p}{\PYZob{}}
		\PY{k+kt}{int} \PY{n}{prev\PYZus{}row} \PY{o}{=} \PY{n}{index\PYZus{}eigenvalue\PYZus{}pairs}\PY{p}{.}\PY{n}{at}\PY{p}{(}\PY{n}{j}\PY{p}{)}\PY{p}{.}\PY{n}{first}\PY{p}{;}
		\PY{k}{for} \PY{p}{(}\PY{k+kt}{int} \PY{n}{i} \PY{o}{=} \PY{l+m+mi}{0}\PY{p}{;} \PY{n}{i} \PY{o}{\PYZlt{}} \PY{n}{V}\PY{p}{.}\PY{n}{rows}\PY{p}{(}\PY{p}{)}\PY{p}{;} \PY{o}{+}\PY{o}{+}\PY{n}{i}\PY{p}{)}
			\PY{n}{sorted\PYZus{}V}\PY{p}{(}\PY{n}{i}\PY{p}{,}\PY{n}{j}\PY{p}{)} \PY{o}{=} \PY{n}{V}\PY{p}{(}\PY{n}{i}\PY{p}{,} \PY{n}{prev\PYZus{}row}\PY{p}{)}\PY{p}{;}
	\PY{p}{\PYZcb{}}

	\PY{n}{V} \PY{o}{=} \PY{n}{sorted\PYZus{}V}\PY{p}{;}
\PY{p}{\PYZcb{}}

\PY{c+cm}{/* no dimension check, only unary vectors*/}
\PY{k+kt}{double} \PY{n}{compute\PYZus{}raleygh\PYZus{}quotient}\PY{p}{(}\PY{n}{MMatrix}\PY{o}{\PYZam{}} \PY{n}{v}\PY{p}{,} \PY{n}{MMatrix}\PY{o}{\PYZam{}} \PY{n}{A}\PY{p}{)}
\PY{p}{\PYZob{}}
	\PY{k+kt}{double} \PY{n}{res} \PY{o}{=} \PY{l+m+mi}{1}\PY{p}{;}
	\PY{n}{foreach\PYZus{}a\PYZus{}ij}\PY{p}{(}\PY{n}{A}\PY{p}{,}\PY{p}{\PYZob{}}
		\PY{n}{res} \PY{o}{+}\PY{o}{=} \PY{n}{a\PYZus{}ij} \PY{o}{*} \PY{n}{v}\PY{p}{(}\PY{n}{i}\PY{p}{)} \PY{o}{*} \PY{n}{v}\PY{p}{(}\PY{n}{j}\PY{p}{)}\PY{p}{;}
	\PY{p}{\PYZcb{}}\PY{p}{)}\PY{p}{;}

	\PY{k}{return} \PY{n}{res}\PY{p}{;}
\PY{p}{\PYZcb{}}

\PY{c+cm}{/* no dimension check, only vectors*/}
\PY{k+kt}{double} \PY{n}{norm}\PY{p}{(}\PY{n}{MMatrix}\PY{o}{\PYZam{}} \PY{n}{v}\PY{p}{)}
\PY{p}{\PYZob{}}
	\PY{k+kt}{double} \PY{n}{res} \PY{o}{=} \PY{l+m+mi}{0}\PY{p}{;}
	\PY{n}{foreach\PYZus{}v\PYZus{}i}\PY{p}{(}\PY{n}{v}\PY{p}{,} \PY{p}{\PYZob{}}
		\PY{n}{res} \PY{o}{+}\PY{o}{=} \PY{n}{v\PYZus{}i} \PY{o}{*} \PY{n}{v\PYZus{}i}\PY{p}{;}
	\PY{p}{\PYZcb{}}\PY{p}{)}\PY{p}{;}

	\PY{k}{return} \PY{n}{sqrt}\PY{p}{(}\PY{n}{res}\PY{p}{)}\PY{p}{;}
\PY{p}{\PYZcb{}}

\PY{n}{MMatrix} \PY{n}{transform\PYZus{}images}\PY{p}{(}\PY{n}{MMatrix}\PY{o}{\PYZam{}} \PY{n}{images}\PY{p}{,} \PY{n}{MMatrix}\PY{o}{\PYZam{}} \PY{n}{V}\PY{p}{)}
\PY{p}{\PYZob{}}
	\PY{k}{return} \PY{n}{images} \PY{o}{*} \PY{n}{V}\PY{p}{;}
\PY{p}{\PYZcb{}}

\PY{n}{MMatrix} \PY{n}{compute\PYZus{}average\PYZus{}by\PYZus{}digit}\PY{p}{(}\PY{n}{MMatrix}\PY{o}{\PYZam{}} \PY{n}{transf\PYZus{}images}\PY{p}{,} \PY{n}{vector}\PY{o}{\PYZlt{}}\PY{k+kt}{int}\PY{o}{\PYZgt{}}\PY{o}{\PYZam{}} \PY{n}{labels}\PY{p}{)}
\PY{p}{\PYZob{}}
	\PY{k+kt}{int} \PY{n}{images\PYZus{}per\PYZus{}digit}\PY{p}{[}\PY{n}{NUM\PYZus{}DIGITS}\PY{p}{]} \PY{o}{=} \PY{p}{\PYZob{}}\PY{l+m+mi}{0}\PY{p}{\PYZcb{}}\PY{p}{;}
	\PY{n}{MMatrix} \PY{n}{avgs}\PY{p}{(}\PY{n}{NUM\PYZus{}DIGITS}\PY{p}{,} \PY{n}{transf\PYZus{}images}\PY{p}{.}\PY{n}{cols}\PY{p}{(}\PY{p}{)}\PY{p}{,} \PY{l+m+mf}{0.0}\PY{p}{)}\PY{p}{;}

	\PY{k}{for} \PY{p}{(}\PY{k+kt}{int} \PY{n}{i} \PY{o}{=} \PY{l+m+mi}{0}\PY{p}{;} \PY{n}{i} \PY{o}{\PYZlt{}} \PY{n}{transf\PYZus{}images}\PY{p}{.}\PY{n}{rows}\PY{p}{(}\PY{p}{)}\PY{p}{;} \PY{o}{+}\PY{o}{+}\PY{n}{i}\PY{p}{)}
	\PY{p}{\PYZob{}}
		\PY{k+kt}{int} \PY{n}{digit\PYZus{}index} \PY{o}{=} \PY{n}{labels}\PY{p}{.}\PY{n}{at}\PY{p}{(}\PY{n}{i}\PY{p}{)}\PY{p}{;}
		\PY{o}{+}\PY{o}{+}\PY{n}{images\PYZus{}per\PYZus{}digit}\PY{p}{[}\PY{n}{digit\PYZus{}index}\PY{p}{]}\PY{p}{;}

		\PY{k}{for} \PY{p}{(}\PY{k+kt}{int} \PY{n}{j} \PY{o}{=} \PY{l+m+mi}{0}\PY{p}{;} \PY{n}{j} \PY{o}{\PYZlt{}} \PY{n}{transf\PYZus{}images}\PY{p}{.}\PY{n}{cols}\PY{p}{(}\PY{p}{)}\PY{p}{;} \PY{o}{+}\PY{o}{+}\PY{n}{j}\PY{p}{)}
			\PY{n}{avgs}\PY{p}{(}\PY{n}{digit\PYZus{}index}\PY{p}{,} \PY{n}{j}\PY{p}{)} \PY{o}{+}\PY{o}{=} \PY{n}{transf\PYZus{}images}\PY{p}{(}\PY{n}{i}\PY{p}{,}\PY{n}{j}\PY{p}{)}\PY{p}{;}
	\PY{p}{\PYZcb{}}

	\PY{n}{foreach\PYZus{}a\PYZus{}ij}\PY{p}{(}\PY{n}{avgs}\PY{p}{,} \PY{n}{a\PYZus{}ij} \PY{o}{=} \PY{p}{(}\PY{n}{a\PYZus{}ij} \PY{o}{/} \PY{n}{images\PYZus{}per\PYZus{}digit}\PY{p}{[}\PY{n}{i}\PY{p}{]}\PY{p}{)}\PY{p}{)}\PY{p}{;}

	\PY{k}{return} \PY{n}{avgs}\PY{p}{;}
\PY{p}{\PYZcb{}}

\PY{n}{MMatrix} \PY{n}{compute\PYZus{}mean\PYZus{}row}\PY{p}{(}\PY{n}{MMatrix}\PY{o}{\PYZam{}} \PY{n}{mat}\PY{p}{)}
\PY{p}{\PYZob{}}
	\PY{n}{MMatrix} \PY{n}{mean\PYZus{}row}\PY{p}{(}\PY{l+m+mi}{1}\PY{p}{,} \PY{n}{mat}\PY{p}{.}\PY{n}{cols}\PY{p}{(}\PY{p}{)}\PY{p}{,} \PY{l+m+mf}{0.0}\PY{p}{)}\PY{p}{;}
	\PY{n}{foreach\PYZus{}a\PYZus{}ij}\PY{p}{(}\PY{n}{mat}\PY{p}{,} \PY{p}{\PYZob{}}
		\PY{n}{mean\PYZus{}row}\PY{p}{(}\PY{n}{j}\PY{p}{)} \PY{o}{+}\PY{o}{=} \PY{n}{a\PYZus{}ij}\PY{p}{;}
	\PY{p}{\PYZcb{}}\PY{p}{)}\PY{p}{;}

	\PY{n}{mean\PYZus{}row} \PY{o}{/}\PY{o}{=} \PY{n}{mat}\PY{p}{.}\PY{n}{rows}\PY{p}{(}\PY{p}{)}\PY{p}{;}

	\PY{k}{return} \PY{n}{mean\PYZus{}row}\PY{p}{;}
\PY{p}{\PYZcb{}}

\PY{k+kt}{int} \PY{n}{classify\PYZus{}image}\PY{p}{(}\PY{n}{MMatrix}\PY{o}{\PYZam{}} \PY{n}{transf\PYZus{}image}\PY{p}{,} \PY{n}{MMatrix}\PY{o}{\PYZam{}} \PY{n}{V}\PY{p}{,} \PY{n}{MMatrix}\PY{o}{\PYZam{}} \PY{n}{avgs}\PY{p}{,} \PY{k+kt}{int} \PY{n}{k}\PY{p}{)}
\PY{p}{\PYZob{}}
	\PY{k+kt}{double} \PY{n}{min\PYZus{}dist} \PY{o}{=} \PY{o}{+}\PY{n}{INFINITY}\PY{p}{,} \PY{n}{dist}\PY{p}{;}
	\PY{k+kt}{int} \PY{n}{digit} \PY{o}{=} \PY{o}{-}\PY{l+m+mi}{1}\PY{p}{;}

	\PY{k}{for} \PY{p}{(}\PY{k+kt}{int} \PY{n}{d} \PY{o}{=} \PY{l+m+mi}{0}\PY{p}{;} \PY{n}{d} \PY{o}{\PYZlt{}} \PY{n}{NUM\PYZus{}DIGITS}\PY{p}{;} \PY{o}{+}\PY{o}{+}\PY{n}{d}\PY{p}{)}
	\PY{p}{\PYZob{}}
		\PY{n}{MMatrix} \PY{n}{diff}\PY{p}{(}\PY{l+m+mi}{1}\PY{p}{,} \PY{n}{k}\PY{p}{)}\PY{p}{;}
		\PY{n}{foreach\PYZus{}v\PYZus{}i}\PY{p}{(}\PY{n}{diff}\PY{p}{,} \PY{n}{v\PYZus{}i} \PY{o}{=} \PY{n}{transf\PYZus{}image}\PY{p}{(}\PY{n}{i}\PY{p}{)} \PY{o}{-} \PY{n}{avgs}\PY{p}{(}\PY{n}{d}\PY{p}{,}\PY{n}{i}\PY{p}{)}\PY{p}{)}\PY{p}{;}

		\PY{n}{dist} \PY{o}{=} \PY{n}{norm}\PY{p}{(}\PY{n}{diff}\PY{p}{)}\PY{p}{;}
		\PY{k}{if}\PY{p}{(} \PY{n}{dist} \PY{o}{\PYZlt{}} \PY{n}{min\PYZus{}dist} \PY{p}{)}
		\PY{p}{\PYZob{}}
			\PY{n}{min\PYZus{}dist} \PY{o}{=} \PY{n}{dist}\PY{p}{;}
			\PY{n}{digit} \PY{o}{=} \PY{n}{d}\PY{p}{;}
		\PY{p}{\PYZcb{}}
	\PY{p}{\PYZcb{}}

	\PY{k}{return} \PY{n}{digit}\PY{p}{;}
\PY{p}{\PYZcb{}}

\PY{k+kt}{int} \PY{n}{classify\PYZus{}images}\PY{p}{(}\PY{n}{MMatrix}\PY{o}{\PYZam{}} \PY{n}{images}\PY{p}{,} \PY{n}{vector}\PY{o}{\PYZlt{}}\PY{k+kt}{int}\PY{o}{\PYZgt{}}\PY{o}{\PYZam{}} \PY{n}{labels}\PY{p}{,} \PY{n}{MMatrix}\PY{o}{\PYZam{}} \PY{n}{V}\PY{p}{,} \PY{n}{MMatrix}\PY{o}{\PYZam{}} \PY{n}{avgs}\PY{p}{,} \PY{k+kt}{int} \PY{n}{k}\PY{p}{)}
\PY{p}{\PYZob{}}
	\PY{k+kt}{int} \PY{n}{hits} \PY{o}{=} \PY{l+m+mi}{0}\PY{p}{;}

	\PY{n}{MMatrix} \PY{n}{transf\PYZus{}images} \PY{o}{=} \PY{n}{transform\PYZus{}images}\PY{p}{(}\PY{n}{images}\PY{p}{,} \PY{n}{V}\PY{p}{)}\PY{p}{;}

	\PY{k}{for} \PY{p}{(}\PY{k+kt}{int} \PY{n}{i} \PY{o}{=} \PY{l+m+mi}{0}\PY{p}{;} \PY{n}{i} \PY{o}{\PYZlt{}} \PY{n}{transf\PYZus{}images}\PY{p}{.}\PY{n}{rows}\PY{p}{(}\PY{p}{)}\PY{p}{;} \PY{o}{+}\PY{o}{+}\PY{n}{i}\PY{p}{)}
	\PY{p}{\PYZob{}}
		\PY{n}{MMatrix} \PY{n}{transf\PYZus{}image}\PY{p}{;}
		\PY{n}{transf\PYZus{}images}\PY{p}{.}\PY{n}{copy\PYZus{}row}\PY{p}{(}\PY{n}{i}\PY{p}{,} \PY{n}{transf\PYZus{}image}\PY{p}{)}\PY{p}{;}

		\PY{k+kt}{int} \PY{n}{digit} \PY{o}{=} \PY{n}{classify\PYZus{}image}\PY{p}{(}\PY{n}{transf\PYZus{}image}\PY{p}{,} \PY{n}{V}\PY{p}{,} \PY{n}{avgs}\PY{p}{,} \PY{n}{k}\PY{p}{)}\PY{p}{;}
		\PY{k}{if}\PY{p}{(}\PY{n}{digit} \PY{o}{=}\PY{o}{=} \PY{n}{labels}\PY{p}{.}\PY{n}{at}\PY{p}{(}\PY{n}{i}\PY{p}{)}\PY{p}{)} \PY{o}{+}\PY{o}{+}\PY{n}{hits}\PY{p}{;}
	\PY{p}{\PYZcb{}}

	\PY{k}{return} \PY{n}{hits}\PY{p}{;}
\PY{p}{\PYZcb{}}
\end{Verbatim}


\subsubsection{data-io.cpp}
\begin{Verbatim}[commandchars=\\\{\}]
\PY{c+cp}{\PYZsh{}}\PY{c+cp}{include \PYZlt{}iostream\PYZgt{}}
\PY{c+cp}{\PYZsh{}}\PY{c+cp}{include \PYZlt{}vector\PYZgt{}}
\PY{c+cp}{\PYZsh{}}\PY{c+cp}{include \PYZlt{}fstream\PYZgt{}}
\PY{k}{using} \PY{k}{namespace} \PY{n}{std}\PY{p}{;}

\PY{c+cp}{\PYZsh{}}\PY{c+cp}{include "..}\PY{c+cp}{/}\PY{c+cp}{lib}\PY{c+cp}{/}\PY{c+cp}{commons.h"}

\PY{c+cp}{\PYZsh{}}\PY{c+cp}{include "data-io.h"}
\PY{c+cp}{\PYZsh{}}\PY{c+cp}{include "mmatrix.h"}

\PY{c+cp}{\PYZsh{}}\PY{c+cp}{define	 INVALID\PYZus{}FILE\PYZus{}FORMAT(fn)					("El formato del archivo " + fn + " es incorrecto" )}
\PY{c+cp}{\PYZsh{}}\PY{c+cp}{define	 IMAGES\PYZus{}LABELS\PYZus{}INCONSISTENCY(igs,lbs)	("Los archivos \PYZbs{}"" + igs + "\PYZbs{}" y \PYZbs{}"" + lbs + "\PYZbs{}" no contienen la misma cantidad de elementos" )}
\PY{c+cp}{\PYZsh{}}\PY{c+cp}{define	 FILE\PYZus{}NOT\PYZus{}FOUND(filename)				("El archivo \PYZbs{}"" + filename + "\PYZbs{}" no existe o se encuentra inutilizable")}
\PY{c+cp}{\PYZsh{}}\PY{c+cp}{define	 FILE\PYZus{}NOT\PYZus{}CREATED(fn)					("No se pudo crear el archivo: " + fn)}

\PY{c+cp}{\PYZsh{}}\PY{c+cp}{define	 IMAGE\PYZus{}HEIGHT\PYZus{}PXS	28}
\PY{c+cp}{\PYZsh{}}\PY{c+cp}{define	 IMAGE\PYZus{}WIDTH\PYZus{}PXS		28}

\PY{c+cp}{\PYZsh{}}\PY{c+cp}{define	 BYTE\PYZus{}2\PYZus{}INT(buff)		 	((int)(0xFF \PYZam{} ((unsigned char)*(buff))))}
\PY{c+cp}{\PYZsh{}}\PY{c+cp}{define BYTE\PYZus{}ARRAY\PYZus{}2\PYZus{}INT(buff) 		((BYTE\PYZus{}2\PYZus{}INT(buff) \PYZlt{}\PYZlt{} 24) + (BYTE\PYZus{}2\PYZus{}INT(buff+1) \PYZlt{}\PYZlt{} 16) + (BYTE\PYZus{}2\PYZus{}INT(buff+2) \PYZlt{}\PYZlt{} 8)  + (BYTE\PYZus{}2\PYZus{}INT(buff+3) \PYZlt{}\PYZlt{} 0))}

\PY{c+cp}{\PYZsh{}}\PY{c+cp}{define LIMIT	100000}

\PY{c+cp}{\PYZsh{}}\PY{c+cp}{ifdef \PYZus{}WIN64}
	\PY{c+cp}{\PYZsh{}}\PY{c+cp}{define		DIRECTORY\PYZus{}SEPARATOR		('\PYZbs{}\PYZbs{}')}
\PY{c+cp}{\PYZsh{}}\PY{c+cp}{elif \PYZus{}WIN32}
	\PY{c+cp}{\PYZsh{}}\PY{c+cp}{define		DIRECTORY\PYZus{}SEPARATOR		('\PYZbs{}\PYZbs{}')}
\PY{c+cp}{\PYZsh{}}\PY{c+cp}{elif \PYZus{}\PYZus{}linux}
	\PY{c+cp}{\PYZsh{}}\PY{c+cp}{define		DIRECTORY\PYZus{}SEPARATOR		('}\PY{c+cp}{/}\PY{c+cp}{')}
\PY{c+cp}{\PYZsh{}}\PY{c+cp}{endif}

\PY{c+cp}{\PYZsh{}}\PY{c+cp}{define		BYTE\PYZus{}ARRAY\PYZus{}DOUBLE\PYZus{}LEN		(sizeof(double))}
\PY{c+cp}{\PYZsh{}}\PY{c+cp}{define		BYTE\PYZus{}ARRAY\PYZus{}INT\PYZus{}LEN		(sizeof(int))}

\PY{k}{typedef} \PY{k}{union} \PY{p}{\PYZob{}}
    \PY{k+kt}{char} \PY{n}{bytes}\PY{p}{[}\PY{n}{BYTE\PYZus{}ARRAY\PYZus{}INT\PYZus{}LEN}\PY{p}{]}\PY{p}{;}
    \PY{k+kt}{int} \PY{n}{value}\PY{p}{;}
\PY{p}{\PYZcb{}} \PY{n}{ByteArrayIntConverter}\PY{p}{;}

\PY{k}{typedef} \PY{k}{union} \PY{p}{\PYZob{}}
    \PY{k+kt}{char} \PY{n}{bytes}\PY{p}{[}\PY{n}{BYTE\PYZus{}ARRAY\PYZus{}DOUBLE\PYZus{}LEN}\PY{p}{]}\PY{p}{;}
    \PY{k+kt}{double} \PY{n}{value}\PY{p}{;}
\PY{p}{\PYZcb{}} \PY{n}{ByteArrayDoubleConverter}\PY{p}{;}

\PY{n}{string} \PY{n}{get\PYZus{}file\PYZus{}basename}\PY{p}{(} \PY{n}{string} \PY{k}{const}\PY{o}{\PYZam{}} \PY{n}{path} \PY{p}{)}\PY{p}{;}

\PY{c+c1}{//	//	//	//}

\PY{k+kt}{void} \PY{n}{load\PYZus{}ubyte\PYZus{}images}\PY{p}{(}\PY{n}{string} \PY{n}{filename}\PY{p}{,} \PY{n}{MMatrix}\PY{o}{\PYZam{}} \PY{n}{images}\PY{p}{)}
\PY{p}{\PYZob{}}

	\PY{n}{ifstream} \PY{n}{file} \PY{p}{(}\PY{n}{filename}\PY{p}{.}\PY{n}{c\PYZus{}str}\PY{p}{(}\PY{p}{)}\PY{p}{,} \PY{n}{ios}\PY{o}{:}\PY{o}{:}\PY{n}{in} \PY{o}{|} \PY{n}{ios}\PY{o}{:}\PY{o}{:}\PY{n}{binary}\PY{p}{)}\PY{p}{;}
	\PY{k}{if}\PY{p}{(}\PY{o}{!}\PY{n}{file}\PY{p}{.}\PY{n}{is\PYZus{}open}\PY{p}{(}\PY{p}{)}\PY{p}{)}
		\PY{n}{DISPLAY\PYZus{}ERROR\PYZus{}AND\PYZus{}EXIT}\PY{p}{(}\PY{n}{FILE\PYZus{}NOT\PYZus{}FOUND}\PY{p}{(}\PY{n}{filename}\PY{p}{)}\PY{p}{)}\PY{p}{;}

	\PY{k+kt}{char} \PY{n}{buffer}\PY{p}{[}\PY{l+m+mi}{4}\PY{p}{]}\PY{p}{;}

	\PY{n}{file}\PY{p}{.}\PY{n}{read}\PY{p}{(}\PY{n}{buffer}\PY{p}{,} \PY{l+m+mi}{4}\PY{p}{)}\PY{p}{;}
	\PY{k}{if}\PY{p}{(}\PY{o}{!}\PY{n}{file}\PY{p}{)} \PY{n}{DISPLAY\PYZus{}ERROR\PYZus{}AND\PYZus{}EXIT}\PY{p}{(}\PY{n}{INVALID\PYZus{}FILE\PYZus{}FORMAT}\PY{p}{(}\PY{n}{filename}\PY{p}{)}\PY{p}{)}\PY{p}{;}
	\PY{k+kt}{int} \PY{n}{magic\PYZus{}number} \PY{o}{=} \PY{n}{BYTE\PYZus{}ARRAY\PYZus{}2\PYZus{}INT}\PY{p}{(}\PY{n}{buffer}\PY{p}{)}\PY{p}{;}
	
	\PY{n}{file}\PY{p}{.}\PY{n}{read}\PY{p}{(}\PY{n}{buffer}\PY{p}{,} \PY{l+m+mi}{4}\PY{p}{)}\PY{p}{;}
	\PY{k}{if}\PY{p}{(}\PY{o}{!}\PY{n}{file}\PY{p}{)} \PY{n}{DISPLAY\PYZus{}ERROR\PYZus{}AND\PYZus{}EXIT}\PY{p}{(}\PY{n}{INVALID\PYZus{}FILE\PYZus{}FORMAT}\PY{p}{(}\PY{n}{filename}\PY{p}{)}\PY{p}{)}\PY{p}{;}
	\PY{k+kt}{int} \PY{n}{number\PYZus{}of\PYZus{}images} \PY{o}{=} \PY{n}{MIN}\PY{p}{(}\PY{n}{LIMIT}\PY{p}{,} \PY{n}{BYTE\PYZus{}ARRAY\PYZus{}2\PYZus{}INT}\PY{p}{(}\PY{n}{buffer}\PY{p}{)}\PY{p}{)}\PY{p}{;}
	
	\PY{n}{file}\PY{p}{.}\PY{n}{read}\PY{p}{(}\PY{n}{buffer}\PY{p}{,} \PY{l+m+mi}{4}\PY{p}{)}\PY{p}{;}
	\PY{k}{if}\PY{p}{(}\PY{o}{!}\PY{n}{file}\PY{p}{)} \PY{n}{DISPLAY\PYZus{}ERROR\PYZus{}AND\PYZus{}EXIT}\PY{p}{(}\PY{n}{INVALID\PYZus{}FILE\PYZus{}FORMAT}\PY{p}{(}\PY{n}{filename}\PY{p}{)}\PY{p}{)}\PY{p}{;}
	\PY{k+kt}{int} \PY{n}{number\PYZus{}of\PYZus{}rows} \PY{o}{=} \PY{n}{BYTE\PYZus{}ARRAY\PYZus{}2\PYZus{}INT}\PY{p}{(}\PY{n}{buffer}\PY{p}{)}\PY{p}{;}
	
	\PY{n}{file}\PY{p}{.}\PY{n}{read}\PY{p}{(}\PY{n}{buffer}\PY{p}{,} \PY{l+m+mi}{4}\PY{p}{)}\PY{p}{;}
	\PY{k}{if}\PY{p}{(}\PY{o}{!}\PY{n}{file}\PY{p}{)} \PY{n}{DISPLAY\PYZus{}ERROR\PYZus{}AND\PYZus{}EXIT}\PY{p}{(}\PY{n}{INVALID\PYZus{}FILE\PYZus{}FORMAT}\PY{p}{(}\PY{n}{filename}\PY{p}{)}\PY{p}{)}\PY{p}{;}
	\PY{k+kt}{int} \PY{n}{number\PYZus{}of\PYZus{}cols} \PY{o}{=} \PY{n}{BYTE\PYZus{}ARRAY\PYZus{}2\PYZus{}INT}\PY{p}{(}\PY{n}{buffer}\PY{p}{)}\PY{p}{;}
	
	\PY{k}{if}\PY{p}{(} \PY{n}{number\PYZus{}of\PYZus{}rows} \PY{o}{!}\PY{o}{=} \PY{n}{IMAGE\PYZus{}HEIGHT\PYZus{}PXS} \PY{o}{|}\PY{o}{|} \PY{n}{number\PYZus{}of\PYZus{}cols} \PY{o}{!}\PY{o}{=} \PY{n}{IMAGE\PYZus{}WIDTH\PYZus{}PXS} \PY{p}{)}
		\PY{n}{DISPLAY\PYZus{}ERROR\PYZus{}AND\PYZus{}EXIT}\PY{p}{(}\PY{n}{INVALID\PYZus{}FILE\PYZus{}FORMAT}\PY{p}{(}\PY{n}{filename}\PY{p}{)}\PY{p}{)}\PY{p}{;}

	\PY{n}{images}\PY{p}{.}\PY{n}{set\PYZus{}size}\PY{p}{(}\PY{n}{number\PYZus{}of\PYZus{}images}\PY{p}{,} \PY{n}{number\PYZus{}of\PYZus{}rows} \PY{o}{*} \PY{n}{number\PYZus{}of\PYZus{}cols}\PY{p}{)}\PY{p}{;}
	\PY{k}{for} \PY{p}{(}\PY{k+kt}{int} \PY{n}{im} \PY{o}{=} \PY{l+m+mi}{0}\PY{p}{;} \PY{n}{im} \PY{o}{\PYZlt{}} \PY{n}{number\PYZus{}of\PYZus{}images}\PY{p}{;} \PY{o}{+}\PY{o}{+}\PY{n}{im}\PY{p}{)}
		\PY{k}{for} \PY{p}{(}\PY{k+kt}{int} \PY{n}{j} \PY{o}{=} \PY{l+m+mi}{0}\PY{p}{;} \PY{n}{j} \PY{o}{\PYZlt{}} \PY{n}{number\PYZus{}of\PYZus{}cols}\PY{p}{;} \PY{o}{+}\PY{o}{+}\PY{n}{j}\PY{p}{)}
			\PY{k}{for} \PY{p}{(}\PY{k+kt}{int} \PY{n}{i} \PY{o}{=} \PY{l+m+mi}{0}\PY{p}{;} \PY{n}{i} \PY{o}{\PYZlt{}} \PY{n}{number\PYZus{}of\PYZus{}rows}\PY{p}{;} \PY{o}{+}\PY{o}{+}\PY{n}{i}\PY{p}{)}
			\PY{p}{\PYZob{}}
				\PY{n}{file}\PY{p}{.}\PY{n}{read}\PY{p}{(}\PY{n}{buffer}\PY{p}{,} \PY{l+m+mi}{1}\PY{p}{)}\PY{p}{;}
				\PY{k}{if}\PY{p}{(}\PY{o}{!}\PY{n}{file}\PY{p}{)}
					\PY{n}{DISPLAY\PYZus{}ERROR\PYZus{}AND\PYZus{}EXIT}\PY{p}{(}\PY{n}{INVALID\PYZus{}FILE\PYZus{}FORMAT}\PY{p}{(}\PY{n}{filename}\PY{p}{)}\PY{p}{)}\PY{p}{;}

				\PY{k+kt}{int} \PY{n}{n} \PY{o}{=} \PY{n}{i} \PY{o}{*} \PY{n}{number\PYZus{}of\PYZus{}cols} \PY{o}{+} \PY{n}{j}\PY{p}{;}
				\PY{n}{images}\PY{p}{(}\PY{n}{im}\PY{p}{,} \PY{n}{n}\PY{p}{)} \PY{o}{=} \PY{p}{(}\PY{p}{(}\PY{k+kt}{double}\PY{p}{)}\PY{n}{BYTE\PYZus{}2\PYZus{}INT}\PY{p}{(}\PY{n}{buffer}\PY{p}{)}\PY{p}{)}\PY{p}{;}
			\PY{p}{\PYZcb{}}

	\PY{n}{file}\PY{p}{.}\PY{n}{close}\PY{p}{(}\PY{p}{)}\PY{p}{;}
\PY{p}{\PYZcb{}}

\PY{k+kt}{void} \PY{n}{load\PYZus{}ubyte\PYZus{}labels}\PY{p}{(}\PY{n}{string} \PY{n}{filename}\PY{p}{,} \PY{n}{vector}\PY{o}{\PYZlt{}}\PY{k+kt}{int}\PY{o}{\PYZgt{}}\PY{o}{\PYZam{}} \PY{n}{labels}\PY{p}{)}
\PY{p}{\PYZob{}}
	\PY{n}{ifstream} \PY{n}{file} \PY{p}{(}\PY{n}{filename}\PY{p}{.}\PY{n}{c\PYZus{}str}\PY{p}{(}\PY{p}{)}\PY{p}{,} \PY{n}{ios}\PY{o}{:}\PY{o}{:}\PY{n}{in} \PY{o}{|} \PY{n}{ios}\PY{o}{:}\PY{o}{:}\PY{n}{binary}\PY{p}{)}\PY{p}{;}
	\PY{k}{if}\PY{p}{(}\PY{o}{!}\PY{n}{file}\PY{p}{.}\PY{n}{is\PYZus{}open}\PY{p}{(}\PY{p}{)}\PY{p}{)}
		\PY{n}{DISPLAY\PYZus{}ERROR\PYZus{}AND\PYZus{}EXIT}\PY{p}{(}\PY{n}{FILE\PYZus{}NOT\PYZus{}FOUND}\PY{p}{(}\PY{n}{filename}\PY{p}{)}\PY{p}{)}\PY{p}{;}
	
	\PY{k+kt}{char} \PY{n}{buffer}\PY{p}{[}\PY{l+m+mi}{4}\PY{p}{]}\PY{p}{;}

	\PY{n}{file}\PY{p}{.}\PY{n}{read}\PY{p}{(}\PY{n}{buffer}\PY{p}{,} \PY{l+m+mi}{4}\PY{p}{)}\PY{p}{;}
	\PY{k}{if}\PY{p}{(}\PY{o}{!}\PY{n}{file}\PY{p}{)} \PY{n}{DISPLAY\PYZus{}ERROR\PYZus{}AND\PYZus{}EXIT}\PY{p}{(}\PY{n}{INVALID\PYZus{}FILE\PYZus{}FORMAT}\PY{p}{(}\PY{n}{filename}\PY{p}{)}\PY{p}{)}\PY{p}{;}
	\PY{k+kt}{int} \PY{n}{magic\PYZus{}number} \PY{o}{=} \PY{n}{BYTE\PYZus{}ARRAY\PYZus{}2\PYZus{}INT}\PY{p}{(}\PY{n}{buffer}\PY{p}{)}\PY{p}{;}

	\PY{n}{file}\PY{p}{.}\PY{n}{read}\PY{p}{(}\PY{n}{buffer}\PY{p}{,} \PY{l+m+mi}{4}\PY{p}{)}\PY{p}{;}
	\PY{k}{if}\PY{p}{(}\PY{o}{!}\PY{n}{file}\PY{p}{)} \PY{n}{DISPLAY\PYZus{}ERROR\PYZus{}AND\PYZus{}EXIT}\PY{p}{(}\PY{n}{INVALID\PYZus{}FILE\PYZus{}FORMAT}\PY{p}{(}\PY{n}{filename}\PY{p}{)}\PY{p}{)}\PY{p}{;}
	\PY{k+kt}{int} \PY{n}{number\PYZus{}of\PYZus{}items} \PY{o}{=} \PY{n}{MIN}\PY{p}{(}\PY{n}{LIMIT}\PY{p}{,} \PY{n}{BYTE\PYZus{}ARRAY\PYZus{}2\PYZus{}INT}\PY{p}{(}\PY{n}{buffer}\PY{p}{)}\PY{p}{)}\PY{p}{;}

	\PY{k}{for} \PY{p}{(}\PY{k+kt}{int} \PY{n}{i} \PY{o}{=} \PY{l+m+mi}{0}\PY{p}{;} \PY{n}{i} \PY{o}{\PYZlt{}} \PY{n}{number\PYZus{}of\PYZus{}items}\PY{p}{;} \PY{o}{+}\PY{o}{+}\PY{n}{i}\PY{p}{)}
	\PY{p}{\PYZob{}}
		\PY{n}{file}\PY{p}{.}\PY{n}{read}\PY{p}{(}\PY{n}{buffer}\PY{p}{,} \PY{l+m+mi}{1}\PY{p}{)}\PY{p}{;}
		\PY{k}{if}\PY{p}{(}\PY{o}{!}\PY{n}{file}\PY{p}{)}
			\PY{n}{DISPLAY\PYZus{}ERROR\PYZus{}AND\PYZus{}EXIT}\PY{p}{(}\PY{n}{INVALID\PYZus{}FILE\PYZus{}FORMAT}\PY{p}{(}\PY{n}{filename}\PY{p}{)}\PY{p}{)}\PY{p}{;}

		\PY{n}{labels}\PY{p}{.}\PY{n}{push\PYZus{}back}\PY{p}{(}\PY{n}{BYTE\PYZus{}2\PYZus{}INT}\PY{p}{(}\PY{n}{buffer}\PY{p}{)}\PY{p}{)}\PY{p}{;}
	\PY{p}{\PYZcb{}}

	\PY{n}{file}\PY{p}{.}\PY{n}{close}\PY{p}{(}\PY{p}{)}\PY{p}{;}
\PY{p}{\PYZcb{}}

\PY{k+kt}{void} \PY{n}{load\PYZus{}mnist\PYZus{}data}\PY{p}{(}\PY{n}{string} \PY{n}{images\PYZus{}filename}\PY{p}{,} \PY{n}{string} \PY{n}{labels\PYZus{}filename}\PY{p}{,} \PY{n}{MMatrix}\PY{o}{\PYZam{}} \PY{n}{images}\PY{p}{,} \PY{n}{vector}\PY{o}{\PYZlt{}}\PY{k+kt}{int}\PY{o}{\PYZgt{}}\PY{o}{\PYZam{}} \PY{n}{labels}\PY{p}{)}
\PY{p}{\PYZob{}}
    \PY{n}{load\PYZus{}ubyte\PYZus{}images}\PY{p}{(}\PY{n}{images\PYZus{}filename}\PY{p}{,} \PY{n}{images}\PY{p}{)}\PY{p}{;}
    \PY{n}{load\PYZus{}ubyte\PYZus{}labels}\PY{p}{(}\PY{n}{labels\PYZus{}filename}\PY{p}{,} \PY{n}{labels}\PY{p}{)}\PY{p}{;}
    \PY{k}{if}\PY{p}{(}\PY{n}{images}\PY{p}{.}\PY{n}{rows}\PY{p}{(}\PY{p}{)} \PY{o}{!}\PY{o}{=} \PY{n}{labels}\PY{p}{.}\PY{n}{size}\PY{p}{(}\PY{p}{)}\PY{p}{)}
        \PY{n}{DISPLAY\PYZus{}ERROR\PYZus{}AND\PYZus{}EXIT}\PY{p}{(}\PY{n}{IMAGES\PYZus{}LABELS\PYZus{}INCONSISTENCY}\PY{p}{(}\PY{n}{images\PYZus{}filename}\PY{p}{,} \PY{n}{labels\PYZus{}filename}\PY{p}{)}\PY{p}{)}\PY{p}{;}
\PY{p}{\PYZcb{}}

\PY{n}{string} \PY{n}{write\PYZus{}covariance\PYZus{}matrix\PYZus{}to\PYZus{}file}\PY{p}{(}\PY{n}{string} \PY{n}{images\PYZus{}filename}\PY{p}{,} \PY{n}{MMatrix}\PY{o}{\PYZam{}} \PY{n}{cov\PYZus{}mat}\PY{p}{)}
\PY{p}{\PYZob{}}
	\PY{n}{string} \PY{n}{filename} \PY{o}{=} \PY{n}{get\PYZus{}file\PYZus{}basename}\PY{p}{(}\PY{n}{images\PYZus{}filename}\PY{p}{)} \PY{o}{+} \PY{l+s}{"}\PY{l+s}{\PYZus{}covmat.mdat}\PY{l+s}{"}\PY{p}{;}
	\PY{n}{ofstream} \PY{n}{file} \PY{p}{(}\PY{n}{filename}\PY{p}{.}\PY{n}{c\PYZus{}str}\PY{p}{(}\PY{p}{)}\PY{p}{,} \PY{n}{ios}\PY{o}{:}\PY{o}{:}\PY{n}{out} \PY{o}{|} \PY{n}{ios}\PY{o}{:}\PY{o}{:}\PY{n}{binary}\PY{p}{)}\PY{p}{;}
	\PY{k}{if}\PY{p}{(}\PY{o}{!}\PY{n}{file}\PY{p}{.}\PY{n}{is\PYZus{}open}\PY{p}{(}\PY{p}{)}\PY{p}{)}
		\PY{n}{DISPLAY\PYZus{}ERROR\PYZus{}AND\PYZus{}EXIT}\PY{p}{(}\PY{n}{FILE\PYZus{}NOT\PYZus{}CREATED}\PY{p}{(}\PY{n}{filename}\PY{p}{)}\PY{p}{)}\PY{p}{;}

	\PY{n}{ByteArrayIntConverter} \PY{n}{int\PYZus{}converter}\PY{p}{;}
	\PY{n}{ByteArrayDoubleConverter} \PY{n}{double\PYZus{}converter}\PY{p}{;}

    \PY{n}{int\PYZus{}converter}\PY{p}{.}\PY{n}{value} \PY{o}{=} \PY{n}{cov\PYZus{}mat}\PY{p}{.}\PY{n}{rows}\PY{p}{(}\PY{p}{)}\PY{p}{;}
    \PY{n}{file}\PY{p}{.}\PY{n}{write}\PY{p}{(}\PY{n}{int\PYZus{}converter}\PY{p}{.}\PY{n}{bytes}\PY{p}{,} \PY{n}{BYTE\PYZus{}ARRAY\PYZus{}INT\PYZus{}LEN}\PY{p}{)}\PY{p}{;}

    \PY{n}{int\PYZus{}converter}\PY{p}{.}\PY{n}{value} \PY{o}{=} \PY{n}{cov\PYZus{}mat}\PY{p}{.}\PY{n}{cols}\PY{p}{(}\PY{p}{)}\PY{p}{;}
    \PY{n}{file}\PY{p}{.}\PY{n}{write}\PY{p}{(}\PY{n}{int\PYZus{}converter}\PY{p}{.}\PY{n}{bytes}\PY{p}{,} \PY{n}{BYTE\PYZus{}ARRAY\PYZus{}INT\PYZus{}LEN}\PY{p}{)}\PY{p}{;}

    \PY{k}{for} \PY{p}{(}\PY{k+kt}{int} \PY{n}{i} \PY{o}{=} \PY{l+m+mi}{0}\PY{p}{;} \PY{n}{i} \PY{o}{\PYZlt{}} \PY{n}{cov\PYZus{}mat}\PY{p}{.}\PY{n}{rows}\PY{p}{(}\PY{p}{)}\PY{p}{;} \PY{o}{+}\PY{o}{+}\PY{n}{i}\PY{p}{)}
    \PY{p}{\PYZob{}}
    	\PY{k}{for} \PY{p}{(}\PY{k+kt}{int} \PY{n}{j} \PY{o}{=} \PY{l+m+mi}{0}\PY{p}{;} \PY{n}{j} \PY{o}{\PYZlt{}} \PY{n}{cov\PYZus{}mat}\PY{p}{.}\PY{n}{cols}\PY{p}{(}\PY{p}{)}\PY{p}{;} \PY{o}{+}\PY{o}{+}\PY{n}{j}\PY{p}{)}
    	\PY{p}{\PYZob{}}
    		\PY{n}{double\PYZus{}converter}\PY{p}{.}\PY{n}{value} \PY{o}{=} \PY{n}{cov\PYZus{}mat}\PY{p}{(}\PY{n}{i}\PY{p}{,}\PY{n}{j}\PY{p}{)}\PY{p}{;}
    		\PY{n}{file}\PY{p}{.}\PY{n}{write}\PY{p}{(}\PY{n}{double\PYZus{}converter}\PY{p}{.}\PY{n}{bytes}\PY{p}{,} \PY{n}{BYTE\PYZus{}ARRAY\PYZus{}DOUBLE\PYZus{}LEN}\PY{p}{)}\PY{p}{;}
    	\PY{p}{\PYZcb{}}
    \PY{p}{\PYZcb{}}

    \PY{n}{file}\PY{p}{.}\PY{n}{close}\PY{p}{(}\PY{p}{)}\PY{p}{;}

    \PY{k}{return} \PY{n}{filename}\PY{p}{;}
\PY{p}{\PYZcb{}}

\PY{k+kt}{void} \PY{n}{load\PYZus{}covariance\PYZus{}matrix}\PY{p}{(}\PY{n}{string} \PY{n}{filename}\PY{p}{,} \PY{n}{MMatrix}\PY{o}{\PYZam{}} \PY{n}{cov\PYZus{}mat}\PY{p}{)}
\PY{p}{\PYZob{}}
	\PY{n}{ifstream} \PY{n}{file} \PY{p}{(}\PY{n}{filename}\PY{p}{.}\PY{n}{c\PYZus{}str}\PY{p}{(}\PY{p}{)}\PY{p}{,} \PY{n}{ios}\PY{o}{:}\PY{o}{:}\PY{n}{in} \PY{o}{|} \PY{n}{ios}\PY{o}{:}\PY{o}{:}\PY{n}{binary}\PY{p}{)}\PY{p}{;}
	\PY{k}{if}\PY{p}{(}\PY{o}{!}\PY{n}{file}\PY{p}{.}\PY{n}{is\PYZus{}open}\PY{p}{(}\PY{p}{)}\PY{p}{)}
		\PY{n}{DISPLAY\PYZus{}ERROR\PYZus{}AND\PYZus{}EXIT}\PY{p}{(}\PY{n}{FILE\PYZus{}NOT\PYZus{}FOUND}\PY{p}{(}\PY{n}{filename}\PY{p}{)}\PY{p}{)}\PY{p}{;}

	\PY{n}{ByteArrayIntConverter} \PY{n}{int\PYZus{}converter}\PY{p}{;}
	\PY{n}{ByteArrayDoubleConverter} \PY{n}{double\PYZus{}converter}\PY{p}{;}

    \PY{n}{file}\PY{p}{.}\PY{n}{read}\PY{p}{(}\PY{n}{int\PYZus{}converter}\PY{p}{.}\PY{n}{bytes}\PY{p}{,} \PY{n}{BYTE\PYZus{}ARRAY\PYZus{}INT\PYZus{}LEN}\PY{p}{)}\PY{p}{;}
    \PY{k}{if}\PY{p}{(}\PY{o}{!}\PY{n}{file}\PY{p}{)} \PY{n}{DISPLAY\PYZus{}ERROR\PYZus{}AND\PYZus{}EXIT}\PY{p}{(}\PY{n}{INVALID\PYZus{}FILE\PYZus{}FORMAT}\PY{p}{(}\PY{n}{filename}\PY{p}{)}\PY{p}{)}\PY{p}{;}
    \PY{k+kt}{int} \PY{n}{rows} \PY{o}{=} \PY{n}{int\PYZus{}converter}\PY{p}{.}\PY{n}{value}\PY{p}{;}

    \PY{n}{file}\PY{p}{.}\PY{n}{read}\PY{p}{(}\PY{n}{int\PYZus{}converter}\PY{p}{.}\PY{n}{bytes}\PY{p}{,} \PY{n}{BYTE\PYZus{}ARRAY\PYZus{}INT\PYZus{}LEN}\PY{p}{)}\PY{p}{;}
    \PY{k}{if}\PY{p}{(}\PY{o}{!}\PY{n}{file}\PY{p}{)} \PY{n}{DISPLAY\PYZus{}ERROR\PYZus{}AND\PYZus{}EXIT}\PY{p}{(}\PY{n}{INVALID\PYZus{}FILE\PYZus{}FORMAT}\PY{p}{(}\PY{n}{filename}\PY{p}{)}\PY{p}{)}\PY{p}{;}
    \PY{k+kt}{int} \PY{n}{cols} \PY{o}{=} \PY{n}{int\PYZus{}converter}\PY{p}{.}\PY{n}{value}\PY{p}{;}

    \PY{n}{cov\PYZus{}mat}\PY{p}{.}\PY{n}{set\PYZus{}size}\PY{p}{(}\PY{n}{rows}\PY{p}{,} \PY{n}{cols}\PY{p}{)}\PY{p}{;}
    \PY{k}{for} \PY{p}{(}\PY{k+kt}{int} \PY{n}{i} \PY{o}{=} \PY{l+m+mi}{0}\PY{p}{;} \PY{n}{i} \PY{o}{\PYZlt{}} \PY{n}{rows}\PY{p}{;} \PY{o}{+}\PY{o}{+}\PY{n}{i}\PY{p}{)}
    \PY{p}{\PYZob{}}
    	\PY{k}{for} \PY{p}{(}\PY{k+kt}{int} \PY{n}{j} \PY{o}{=} \PY{l+m+mi}{0}\PY{p}{;} \PY{n}{j} \PY{o}{\PYZlt{}} \PY{n}{cols}\PY{p}{;} \PY{o}{+}\PY{o}{+}\PY{n}{j}\PY{p}{)}
    	\PY{p}{\PYZob{}}
    		\PY{n}{file}\PY{p}{.}\PY{n}{read}\PY{p}{(}\PY{n}{double\PYZus{}converter}\PY{p}{.}\PY{n}{bytes}\PY{p}{,} \PY{n}{BYTE\PYZus{}ARRAY\PYZus{}DOUBLE\PYZus{}LEN}\PY{p}{)}\PY{p}{;}
    		\PY{k}{if}\PY{p}{(}\PY{o}{!}\PY{n}{file}\PY{p}{)} \PY{n}{DISPLAY\PYZus{}ERROR\PYZus{}AND\PYZus{}EXIT}\PY{p}{(}\PY{n}{INVALID\PYZus{}FILE\PYZus{}FORMAT}\PY{p}{(}\PY{n}{filename}\PY{p}{)}\PY{p}{)}\PY{p}{;}
    		\PY{n}{cov\PYZus{}mat}\PY{p}{(}\PY{n}{i}\PY{p}{,}\PY{n}{j}\PY{p}{)} \PY{o}{=} \PY{n}{double\PYZus{}converter}\PY{p}{.}\PY{n}{value}\PY{p}{;}
    	\PY{p}{\PYZcb{}}
    \PY{p}{\PYZcb{}}

    \PY{n}{file}\PY{p}{.}\PY{n}{close}\PY{p}{(}\PY{p}{)}\PY{p}{;}
\PY{p}{\PYZcb{}}

\PY{n}{string} \PY{n}{write\PYZus{}data\PYZus{}file}\PY{p}{(}\PY{k+kt}{double} \PY{n}{delta}\PY{p}{,} \PY{n}{MMatrix}\PY{o}{\PYZam{}} \PY{n}{V}\PY{p}{,} \PY{n}{MMatrix}\PY{o}{\PYZam{}} \PY{n}{avgs}\PY{p}{)}
\PY{p}{\PYZob{}}
	\PY{n}{string} \PY{n}{filename} \PY{o}{=} \PY{l+s}{"}\PY{l+s}{tp3\PYZus{}data\PYZus{}delta\PYZus{}}\PY{l+s}{"} \PY{o}{+} \PY{n}{double2str}\PY{p}{(}\PY{n}{delta}\PY{p}{)} \PY{o}{+} \PY{l+s}{"}\PY{l+s}{.mdat}\PY{l+s}{"}\PY{p}{;}
	\PY{n}{ofstream} \PY{n}{file} \PY{p}{(}\PY{n}{filename}\PY{p}{.}\PY{n}{c\PYZus{}str}\PY{p}{(}\PY{p}{)}\PY{p}{,} \PY{n}{ios}\PY{o}{:}\PY{o}{:}\PY{n}{out} \PY{o}{|} \PY{n}{ios}\PY{o}{:}\PY{o}{:}\PY{n}{binary}\PY{p}{)}\PY{p}{;}
	\PY{k}{if}\PY{p}{(}\PY{o}{!}\PY{n}{file}\PY{p}{.}\PY{n}{is\PYZus{}open}\PY{p}{(}\PY{p}{)}\PY{p}{)}
		\PY{n}{DISPLAY\PYZus{}ERROR\PYZus{}AND\PYZus{}EXIT}\PY{p}{(}\PY{n}{FILE\PYZus{}NOT\PYZus{}CREATED}\PY{p}{(}\PY{n}{filename}\PY{p}{)}\PY{p}{)}\PY{p}{;}

	\PY{n}{ByteArrayIntConverter} \PY{n}{int\PYZus{}converter}\PY{p}{;}
	\PY{n}{ByteArrayDoubleConverter} \PY{n}{double\PYZus{}converter}\PY{p}{;}

    \PY{n}{double\PYZus{}converter}\PY{p}{.}\PY{n}{value} \PY{o}{=} \PY{n}{delta}\PY{p}{;}
    \PY{n}{file}\PY{p}{.}\PY{n}{write}\PY{p}{(}\PY{n}{double\PYZus{}converter}\PY{p}{.}\PY{n}{bytes}\PY{p}{,} \PY{n}{BYTE\PYZus{}ARRAY\PYZus{}DOUBLE\PYZus{}LEN}\PY{p}{)}\PY{p}{;}

    \PY{n}{int\PYZus{}converter}\PY{p}{.}\PY{n}{value} \PY{o}{=} \PY{n}{V}\PY{p}{.}\PY{n}{rows}\PY{p}{(}\PY{p}{)}\PY{p}{;}
    \PY{n}{file}\PY{p}{.}\PY{n}{write}\PY{p}{(}\PY{n}{int\PYZus{}converter}\PY{p}{.}\PY{n}{bytes}\PY{p}{,} \PY{n}{BYTE\PYZus{}ARRAY\PYZus{}INT\PYZus{}LEN}\PY{p}{)}\PY{p}{;}

    \PY{n}{int\PYZus{}converter}\PY{p}{.}\PY{n}{value} \PY{o}{=} \PY{n}{V}\PY{p}{.}\PY{n}{cols}\PY{p}{(}\PY{p}{)}\PY{p}{;}
    \PY{n}{file}\PY{p}{.}\PY{n}{write}\PY{p}{(}\PY{n}{int\PYZus{}converter}\PY{p}{.}\PY{n}{bytes}\PY{p}{,} \PY{n}{BYTE\PYZus{}ARRAY\PYZus{}INT\PYZus{}LEN}\PY{p}{)}\PY{p}{;}

    \PY{k}{for} \PY{p}{(}\PY{k+kt}{int} \PY{n}{i} \PY{o}{=} \PY{l+m+mi}{0}\PY{p}{;} \PY{n}{i} \PY{o}{\PYZlt{}} \PY{n}{V}\PY{p}{.}\PY{n}{rows}\PY{p}{(}\PY{p}{)}\PY{p}{;} \PY{o}{+}\PY{o}{+}\PY{n}{i}\PY{p}{)}
    \PY{p}{\PYZob{}}
    	\PY{k}{for} \PY{p}{(}\PY{k+kt}{int} \PY{n}{j} \PY{o}{=} \PY{l+m+mi}{0}\PY{p}{;} \PY{n}{j} \PY{o}{\PYZlt{}} \PY{n}{V}\PY{p}{.}\PY{n}{cols}\PY{p}{(}\PY{p}{)}\PY{p}{;} \PY{o}{+}\PY{o}{+}\PY{n}{j}\PY{p}{)}
    	\PY{p}{\PYZob{}}
    		\PY{n}{double\PYZus{}converter}\PY{p}{.}\PY{n}{value} \PY{o}{=} \PY{n}{V}\PY{p}{(}\PY{n}{i}\PY{p}{,}\PY{n}{j}\PY{p}{)}\PY{p}{;}
    		\PY{n}{file}\PY{p}{.}\PY{n}{write}\PY{p}{(}\PY{n}{double\PYZus{}converter}\PY{p}{.}\PY{n}{bytes}\PY{p}{,} \PY{n}{BYTE\PYZus{}ARRAY\PYZus{}DOUBLE\PYZus{}LEN}\PY{p}{)}\PY{p}{;}
    	\PY{p}{\PYZcb{}}
    \PY{p}{\PYZcb{}}

    \PY{n}{int\PYZus{}converter}\PY{p}{.}\PY{n}{value} \PY{o}{=} \PY{n}{avgs}\PY{p}{.}\PY{n}{rows}\PY{p}{(}\PY{p}{)}\PY{p}{;}
    \PY{n}{file}\PY{p}{.}\PY{n}{write}\PY{p}{(}\PY{n}{int\PYZus{}converter}\PY{p}{.}\PY{n}{bytes}\PY{p}{,} \PY{n}{BYTE\PYZus{}ARRAY\PYZus{}INT\PYZus{}LEN}\PY{p}{)}\PY{p}{;}

    \PY{n}{int\PYZus{}converter}\PY{p}{.}\PY{n}{value} \PY{o}{=} \PY{n}{avgs}\PY{p}{.}\PY{n}{cols}\PY{p}{(}\PY{p}{)}\PY{p}{;}
    \PY{n}{file}\PY{p}{.}\PY{n}{write}\PY{p}{(}\PY{n}{int\PYZus{}converter}\PY{p}{.}\PY{n}{bytes}\PY{p}{,} \PY{n}{BYTE\PYZus{}ARRAY\PYZus{}INT\PYZus{}LEN}\PY{p}{)}\PY{p}{;}

    \PY{k}{for} \PY{p}{(}\PY{k+kt}{int} \PY{n}{i} \PY{o}{=} \PY{l+m+mi}{0}\PY{p}{;} \PY{n}{i} \PY{o}{\PYZlt{}} \PY{n}{avgs}\PY{p}{.}\PY{n}{rows}\PY{p}{(}\PY{p}{)}\PY{p}{;} \PY{o}{+}\PY{o}{+}\PY{n}{i}\PY{p}{)}
    \PY{p}{\PYZob{}}
    	\PY{k}{for} \PY{p}{(}\PY{k+kt}{int} \PY{n}{j} \PY{o}{=} \PY{l+m+mi}{0}\PY{p}{;} \PY{n}{j} \PY{o}{\PYZlt{}} \PY{n}{avgs}\PY{p}{.}\PY{n}{cols}\PY{p}{(}\PY{p}{)}\PY{p}{;} \PY{o}{+}\PY{o}{+}\PY{n}{j}\PY{p}{)}
    	\PY{p}{\PYZob{}}
    		\PY{n}{double\PYZus{}converter}\PY{p}{.}\PY{n}{value} \PY{o}{=} \PY{n}{avgs}\PY{p}{(}\PY{n}{i}\PY{p}{,}\PY{n}{j}\PY{p}{)}\PY{p}{;}
    		\PY{n}{file}\PY{p}{.}\PY{n}{write}\PY{p}{(}\PY{n}{double\PYZus{}converter}\PY{p}{.}\PY{n}{bytes}\PY{p}{,} \PY{n}{BYTE\PYZus{}ARRAY\PYZus{}DOUBLE\PYZus{}LEN}\PY{p}{)}\PY{p}{;}
    	\PY{p}{\PYZcb{}}
    \PY{p}{\PYZcb{}}

	\PY{n}{file}\PY{p}{.}\PY{n}{close}\PY{p}{(}\PY{p}{)}\PY{p}{;}

    \PY{k}{return} \PY{n}{filename}\PY{p}{;}
\PY{p}{\PYZcb{}}

\PY{k+kt}{void} \PY{n}{load\PYZus{}data\PYZus{}file}\PY{p}{(}\PY{n}{string} \PY{n}{filename}\PY{p}{,} \PY{k+kt}{double}\PY{o}{\PYZam{}} \PY{n}{delta}\PY{p}{,} \PY{n}{MMatrix}\PY{o}{\PYZam{}} \PY{n}{V}\PY{p}{,} \PY{n}{MMatrix}\PY{o}{\PYZam{}} \PY{n}{avgs}\PY{p}{)}
\PY{p}{\PYZob{}}
	\PY{n}{ifstream} \PY{n}{file} \PY{p}{(}\PY{n}{filename}\PY{p}{.}\PY{n}{c\PYZus{}str}\PY{p}{(}\PY{p}{)}\PY{p}{,} \PY{n}{ios}\PY{o}{:}\PY{o}{:}\PY{n}{in} \PY{o}{|} \PY{n}{ios}\PY{o}{:}\PY{o}{:}\PY{n}{binary}\PY{p}{)}\PY{p}{;}
	\PY{k}{if}\PY{p}{(}\PY{o}{!}\PY{n}{file}\PY{p}{.}\PY{n}{is\PYZus{}open}\PY{p}{(}\PY{p}{)}\PY{p}{)}
		\PY{n}{DISPLAY\PYZus{}ERROR\PYZus{}AND\PYZus{}EXIT}\PY{p}{(}\PY{n}{FILE\PYZus{}NOT\PYZus{}FOUND}\PY{p}{(}\PY{n}{filename}\PY{p}{)}\PY{p}{)}\PY{p}{;}

	\PY{n}{ByteArrayIntConverter} \PY{n}{int\PYZus{}converter}\PY{p}{;}
	\PY{n}{ByteArrayDoubleConverter} \PY{n}{double\PYZus{}converter}\PY{p}{;}

    \PY{n}{file}\PY{p}{.}\PY{n}{read}\PY{p}{(}\PY{n}{double\PYZus{}converter}\PY{p}{.}\PY{n}{bytes}\PY{p}{,} \PY{n}{BYTE\PYZus{}ARRAY\PYZus{}DOUBLE\PYZus{}LEN}\PY{p}{)}\PY{p}{;}
    \PY{k}{if}\PY{p}{(}\PY{o}{!}\PY{n}{file}\PY{p}{)} \PY{n}{DISPLAY\PYZus{}ERROR\PYZus{}AND\PYZus{}EXIT}\PY{p}{(}\PY{n}{INVALID\PYZus{}FILE\PYZus{}FORMAT}\PY{p}{(}\PY{n}{filename}\PY{p}{)}\PY{p}{)}\PY{p}{;}
    \PY{n}{delta} \PY{o}{=} \PY{n}{double\PYZus{}converter}\PY{p}{.}\PY{n}{value}\PY{p}{;}

    \PY{n}{file}\PY{p}{.}\PY{n}{read}\PY{p}{(}\PY{n}{int\PYZus{}converter}\PY{p}{.}\PY{n}{bytes}\PY{p}{,} \PY{n}{BYTE\PYZus{}ARRAY\PYZus{}INT\PYZus{}LEN}\PY{p}{)}\PY{p}{;}
    \PY{k}{if}\PY{p}{(}\PY{o}{!}\PY{n}{file}\PY{p}{)} \PY{n}{DISPLAY\PYZus{}ERROR\PYZus{}AND\PYZus{}EXIT}\PY{p}{(}\PY{n}{INVALID\PYZus{}FILE\PYZus{}FORMAT}\PY{p}{(}\PY{n}{filename}\PY{p}{)}\PY{p}{)}\PY{p}{;}
    \PY{k+kt}{int} \PY{n}{rows} \PY{o}{=} \PY{n}{int\PYZus{}converter}\PY{p}{.}\PY{n}{value}\PY{p}{;}

    \PY{n}{file}\PY{p}{.}\PY{n}{read}\PY{p}{(}\PY{n}{int\PYZus{}converter}\PY{p}{.}\PY{n}{bytes}\PY{p}{,} \PY{n}{BYTE\PYZus{}ARRAY\PYZus{}INT\PYZus{}LEN}\PY{p}{)}\PY{p}{;}
    \PY{k}{if}\PY{p}{(}\PY{o}{!}\PY{n}{file}\PY{p}{)} \PY{n}{DISPLAY\PYZus{}ERROR\PYZus{}AND\PYZus{}EXIT}\PY{p}{(}\PY{n}{INVALID\PYZus{}FILE\PYZus{}FORMAT}\PY{p}{(}\PY{n}{filename}\PY{p}{)}\PY{p}{)}\PY{p}{;}
    \PY{k+kt}{int} \PY{n}{cols} \PY{o}{=} \PY{n}{int\PYZus{}converter}\PY{p}{.}\PY{n}{value}\PY{p}{;}

    \PY{n}{V}\PY{p}{.}\PY{n}{set\PYZus{}size}\PY{p}{(}\PY{n}{rows}\PY{p}{,} \PY{n}{cols}\PY{p}{)}\PY{p}{;}
    \PY{k}{for} \PY{p}{(}\PY{k+kt}{int} \PY{n}{i} \PY{o}{=} \PY{l+m+mi}{0}\PY{p}{;} \PY{n}{i} \PY{o}{\PYZlt{}} \PY{n}{rows}\PY{p}{;} \PY{o}{+}\PY{o}{+}\PY{n}{i}\PY{p}{)}
    \PY{p}{\PYZob{}}
    	\PY{k}{for} \PY{p}{(}\PY{k+kt}{int} \PY{n}{j} \PY{o}{=} \PY{l+m+mi}{0}\PY{p}{;} \PY{n}{j} \PY{o}{\PYZlt{}} \PY{n}{cols}\PY{p}{;} \PY{o}{+}\PY{o}{+}\PY{n}{j}\PY{p}{)}
    	\PY{p}{\PYZob{}}
    		\PY{n}{file}\PY{p}{.}\PY{n}{read}\PY{p}{(}\PY{n}{double\PYZus{}converter}\PY{p}{.}\PY{n}{bytes}\PY{p}{,} \PY{n}{BYTE\PYZus{}ARRAY\PYZus{}DOUBLE\PYZus{}LEN}\PY{p}{)}\PY{p}{;}
    		\PY{k}{if}\PY{p}{(}\PY{o}{!}\PY{n}{file}\PY{p}{)} \PY{n}{DISPLAY\PYZus{}ERROR\PYZus{}AND\PYZus{}EXIT}\PY{p}{(}\PY{n}{INVALID\PYZus{}FILE\PYZus{}FORMAT}\PY{p}{(}\PY{n}{filename}\PY{p}{)}\PY{p}{)}\PY{p}{;}
    		\PY{n}{V}\PY{p}{(}\PY{n}{i}\PY{p}{,}\PY{n}{j}\PY{p}{)} \PY{o}{=} \PY{n}{double\PYZus{}converter}\PY{p}{.}\PY{n}{value}\PY{p}{;}
    	\PY{p}{\PYZcb{}}
    \PY{p}{\PYZcb{}}

    \PY{n}{file}\PY{p}{.}\PY{n}{read}\PY{p}{(}\PY{n}{int\PYZus{}converter}\PY{p}{.}\PY{n}{bytes}\PY{p}{,} \PY{n}{BYTE\PYZus{}ARRAY\PYZus{}INT\PYZus{}LEN}\PY{p}{)}\PY{p}{;}
    \PY{k}{if}\PY{p}{(}\PY{o}{!}\PY{n}{file}\PY{p}{)} \PY{n}{DISPLAY\PYZus{}ERROR\PYZus{}AND\PYZus{}EXIT}\PY{p}{(}\PY{n}{INVALID\PYZus{}FILE\PYZus{}FORMAT}\PY{p}{(}\PY{n}{filename}\PY{p}{)}\PY{p}{)}\PY{p}{;}
    \PY{n}{rows} \PY{o}{=} \PY{n}{int\PYZus{}converter}\PY{p}{.}\PY{n}{value}\PY{p}{;}

    \PY{n}{file}\PY{p}{.}\PY{n}{read}\PY{p}{(}\PY{n}{int\PYZus{}converter}\PY{p}{.}\PY{n}{bytes}\PY{p}{,} \PY{n}{BYTE\PYZus{}ARRAY\PYZus{}INT\PYZus{}LEN}\PY{p}{)}\PY{p}{;}
    \PY{k}{if}\PY{p}{(}\PY{o}{!}\PY{n}{file}\PY{p}{)} \PY{n}{DISPLAY\PYZus{}ERROR\PYZus{}AND\PYZus{}EXIT}\PY{p}{(}\PY{n}{INVALID\PYZus{}FILE\PYZus{}FORMAT}\PY{p}{(}\PY{n}{filename}\PY{p}{)}\PY{p}{)}\PY{p}{;}
    \PY{n}{cols} \PY{o}{=} \PY{n}{int\PYZus{}converter}\PY{p}{.}\PY{n}{value}\PY{p}{;}

    \PY{n}{avgs}\PY{p}{.}\PY{n}{set\PYZus{}size}\PY{p}{(}\PY{n}{rows}\PY{p}{,} \PY{n}{cols}\PY{p}{)}\PY{p}{;}
    \PY{k}{for} \PY{p}{(}\PY{k+kt}{int} \PY{n}{i} \PY{o}{=} \PY{l+m+mi}{0}\PY{p}{;} \PY{n}{i} \PY{o}{\PYZlt{}} \PY{n}{rows}\PY{p}{;} \PY{o}{+}\PY{o}{+}\PY{n}{i}\PY{p}{)}
    \PY{p}{\PYZob{}}
    	\PY{k}{for} \PY{p}{(}\PY{k+kt}{int} \PY{n}{j} \PY{o}{=} \PY{l+m+mi}{0}\PY{p}{;} \PY{n}{j} \PY{o}{\PYZlt{}} \PY{n}{cols}\PY{p}{;} \PY{o}{+}\PY{o}{+}\PY{n}{j}\PY{p}{)}
    	\PY{p}{\PYZob{}}
    		\PY{n}{file}\PY{p}{.}\PY{n}{read}\PY{p}{(}\PY{n}{double\PYZus{}converter}\PY{p}{.}\PY{n}{bytes}\PY{p}{,} \PY{n}{BYTE\PYZus{}ARRAY\PYZus{}DOUBLE\PYZus{}LEN}\PY{p}{)}\PY{p}{;}
    		\PY{k}{if}\PY{p}{(}\PY{o}{!}\PY{n}{file}\PY{p}{)} \PY{n}{DISPLAY\PYZus{}ERROR\PYZus{}AND\PYZus{}EXIT}\PY{p}{(}\PY{n}{INVALID\PYZus{}FILE\PYZus{}FORMAT}\PY{p}{(}\PY{n}{filename}\PY{p}{)}\PY{p}{)}\PY{p}{;}
    		\PY{n}{avgs}\PY{p}{(}\PY{n}{i}\PY{p}{,}\PY{n}{j}\PY{p}{)} \PY{o}{=} \PY{n}{double\PYZus{}converter}\PY{p}{.}\PY{n}{value}\PY{p}{;}
    	\PY{p}{\PYZcb{}}
    \PY{p}{\PYZcb{}}

	\PY{n}{file}\PY{p}{.}\PY{n}{close}\PY{p}{(}\PY{p}{)}\PY{p}{;}
\PY{p}{\PYZcb{}}

\PY{k+kt}{void} \PY{n}{open\PYZus{}output\PYZus{}file}\PY{p}{(}\PY{n}{string} \PY{n}{filename}\PY{p}{,} \PY{n}{ofstream}\PY{o}{\PYZam{}} \PY{n}{file}\PY{p}{)}
\PY{p}{\PYZob{}}
	\PY{n}{file}\PY{p}{.}\PY{n}{open}\PY{p}{(}\PY{n}{filename}\PY{p}{.}\PY{n}{c\PYZus{}str}\PY{p}{(}\PY{p}{)}\PY{p}{)}\PY{p}{;}
	\PY{k}{if}\PY{p}{(}\PY{o}{!}\PY{n}{file}\PY{p}{.}\PY{n}{is\PYZus{}open}\PY{p}{(}\PY{p}{)}\PY{p}{)}
		\PY{n}{DISPLAY\PYZus{}ERROR\PYZus{}AND\PYZus{}EXIT}\PY{p}{(}\PY{n}{FILE\PYZus{}NOT\PYZus{}CREATED}\PY{p}{(}\PY{n}{filename}\PY{p}{)}\PY{p}{)}\PY{p}{;}

	\PY{n}{file} \PY{o}{\PYZlt{}}\PY{o}{\PYZlt{}} \PY{l+s}{"}\PY{l+s}{\PYZpc{} delta, k, archivo de imágenes clasificadas, cantidad de imágenes clasificadas, cantidad de aciertos}\PY{l+s}{"} \PY{o}{\PYZlt{}}\PY{o}{\PYZlt{}} \PY{n}{endl}\PY{p}{;}
	\PY{n}{file} \PY{o}{\PYZlt{}}\PY{o}{\PYZlt{}} \PY{n}{get\PYZus{}file\PYZus{}basename}\PY{p}{(}\PY{n}{filename}\PY{p}{)} \PY{o}{\PYZlt{}}\PY{o}{\PYZlt{}} \PY{l+s}{"}\PY{l+s}{ = \PYZob{} }\PY{l+s}{"} \PY{o}{\PYZlt{}}\PY{o}{\PYZlt{}} \PY{n}{endl}\PY{p}{;}
\PY{p}{\PYZcb{}}

\PY{k+kt}{void} \PY{n}{write\PYZus{}results}\PY{p}{(}\PY{n}{ofstream}\PY{o}{\PYZam{}} \PY{n}{file}\PY{p}{,} \PY{k+kt}{double} \PY{n}{delta}\PY{p}{,} \PY{k+kt}{int} \PY{n}{k}\PY{p}{,} \PY{n}{string} \PY{n}{images\PYZus{}filename}\PY{p}{,} \PY{k+kt}{int} \PY{n}{total\PYZus{}images}\PY{p}{,} \PY{k+kt}{int} \PY{n}{hits}\PY{p}{)}
\PY{p}{\PYZob{}}
	\PY{n}{file} \PY{o}{\PYZlt{}}\PY{o}{\PYZlt{}} \PY{l+s}{"}\PY{l+s+se}{\PYZbs{}t}\PY{l+s}{\PYZob{}}\PY{l+s}{"}\PY{p}{;}
	\PY{n}{file} \PY{o}{\PYZlt{}}\PY{o}{\PYZlt{}} \PY{n}{delta} \PY{o}{\PYZlt{}}\PY{o}{\PYZlt{}} \PY{l+s}{"}\PY{l+s}{, }\PY{l+s}{"}\PY{p}{;}
	\PY{n}{file} \PY{o}{\PYZlt{}}\PY{o}{\PYZlt{}} \PY{n}{k} \PY{o}{\PYZlt{}}\PY{o}{\PYZlt{}} \PY{l+s}{"}\PY{l+s}{, }\PY{l+s}{"}\PY{p}{;}
	\PY{n}{file} \PY{o}{\PYZlt{}}\PY{o}{\PYZlt{}} \PY{l+s}{"}\PY{l+s}{'}\PY{l+s}{"} \PY{o}{\PYZlt{}}\PY{o}{\PYZlt{}} \PY{n}{images\PYZus{}filename} \PY{o}{\PYZlt{}}\PY{o}{\PYZlt{}} \PY{l+s}{"}\PY{l+s}{', }\PY{l+s}{"}\PY{p}{;}
	\PY{n}{file} \PY{o}{\PYZlt{}}\PY{o}{\PYZlt{}} \PY{n}{total\PYZus{}images} \PY{o}{\PYZlt{}}\PY{o}{\PYZlt{}} \PY{l+s}{"}\PY{l+s}{, }\PY{l+s}{"}\PY{p}{;}
	\PY{n}{file} \PY{o}{\PYZlt{}}\PY{o}{\PYZlt{}} \PY{n}{hits}\PY{p}{;}
	\PY{n}{file} \PY{o}{\PYZlt{}}\PY{o}{\PYZlt{}} \PY{l+s}{"}\PY{l+s}{\PYZcb{},}\PY{l+s}{"} \PY{o}{\PYZlt{}}\PY{o}{\PYZlt{}} \PY{n}{endl}\PY{p}{;}
\PY{p}{\PYZcb{}}

\PY{k+kt}{void} \PY{n}{close\PYZus{}output\PYZus{}file}\PY{p}{(}\PY{n}{ofstream}\PY{o}{\PYZam{}} \PY{n}{file}\PY{p}{)}
\PY{p}{\PYZob{}}
	\PY{n}{file} \PY{o}{\PYZlt{}}\PY{o}{\PYZlt{}} \PY{l+s}{"}\PY{l+s}{\PYZcb{};}\PY{l+s}{"}\PY{p}{;}
	\PY{n}{file}\PY{p}{.}\PY{n}{close}\PY{p}{(}\PY{p}{)}\PY{p}{;}
\PY{p}{\PYZcb{}}

\PY{n}{string} \PY{n}{get\PYZus{}file\PYZus{}basename}\PY{p}{(} \PY{n}{string} \PY{k}{const}\PY{o}{\PYZam{}} \PY{n}{path} \PY{p}{)}
\PY{p}{\PYZob{}}
    \PY{n}{string} \PY{n}{filename} \PY{o}{=} \PY{n}{path}\PY{p}{.}\PY{n}{substr}\PY{p}{(} \PY{n}{path}\PY{p}{.}\PY{n}{find\PYZus{}last\PYZus{}of}\PY{p}{(} \PY{n}{DIRECTORY\PYZus{}SEPARATOR} \PY{p}{)} \PY{o}{+} \PY{l+m+mi}{1} \PY{p}{)}\PY{p}{;}
    \PY{n}{string} \PY{n}{basename} \PY{o}{=} \PY{n}{filename}\PY{p}{.}\PY{n}{substr}\PY{p}{(} \PY{l+m+mi}{0}\PY{p}{,} \PY{n}{filename}\PY{p}{.}\PY{n}{find\PYZus{}last\PYZus{}of}\PY{p}{(} \PY{l+s+sc}{'.'} \PY{p}{)} \PY{p}{)}\PY{p}{;}

    \PY{k}{return} \PY{n}{basename}\PY{p}{;}
\PY{p}{\PYZcb{}}
\end{Verbatim}


\subsubsection{mmatrix.cpp}
\begin{Verbatim}[commandchars=\\\{\}]
\PY{c+cp}{\PYZsh{}}\PY{c+cp}{include \PYZlt{}cmath\PYZgt{}}
\PY{c+cp}{\PYZsh{}}\PY{c+cp}{include \PYZlt{}iomanip\PYZgt{}}
\PY{c+cp}{\PYZsh{}}\PY{c+cp}{include \PYZlt{}iostream\PYZgt{}}
\PY{k}{using} \PY{k}{namespace} \PY{n}{std}\PY{p}{;}

\PY{c+cp}{\PYZsh{}}\PY{c+cp}{include "..}\PY{c+cp}{/}\PY{c+cp}{lib}\PY{c+cp}{/}\PY{c+cp}{commons.h"}

\PY{c+cp}{\PYZsh{}}\PY{c+cp}{include "mmatrix.h"}

\PY{c+cp}{\PYZsh{}}\PY{c+cp}{define INVALID\PYZus{}MATRIX\PYZus{}SIZE(rows,cols)			\PYZbs{}}
\PY{c+cp}{		("Tamaño de matriz inválido; filas: " + int2str(rows) + ", columnas: " + int2str(cols))}
\PY{c+cp}{\PYZsh{}}\PY{c+cp}{define OUT\PYZus{}OF\PYZus{}BOUNDS(i,j)						\PYZbs{}}
\PY{c+cp}{		("Índices fuera de rango; i: " + int2str(i) + ", j: " + int2str(j))}
\PY{c+cp}{\PYZsh{}}\PY{c+cp}{define OUT\PYZus{}OF\PYZus{}BOUNDS\PYZus{}LINEAR(n)					\PYZbs{}}
\PY{c+cp}{		("Índice lineal fuera de rango; i: " + int2str(n))}
\PY{c+cp}{\PYZsh{}}\PY{c+cp}{define DIMENSIONS\PYZus{}MISMATCH(r1,r2,c1,c2)		\PYZbs{}}
\PY{c+cp}{		("Las dimensiones no concuerdan; el lado izquierdo es de (" + int2str(r1) + ", " + int2str(c1) + "), y el lado derecho de (" + int2str(r2) + ", " + int2str(c2) + ")")}
\PY{c+cp}{\PYZsh{}}\PY{c+cp}{define	 DIMENSIONS\PYZus{}MISMATCH\PYZus{}MULT\PYZus{}INPLACE(r1,r2,c1,c2)	\PYZbs{}}
\PY{c+cp}{		("Las dimensiones no concuerdan para la multiplicación in situ (el producto debe estar definido y la matriz derecha debe ser cuadrada); el lado izquierdo es de (" + int2str(r1) + ", " + int2str(c1) + "), y el lado derecho de (" + int2str(r2) + ", " + int2str(c2) + ")")}
\PY{c+cp}{\PYZsh{}}\PY{c+cp}{define	 NEAR\PYZus{}ZERO\PYZus{}DIVISION(denom)	\PYZbs{}}
\PY{c+cp}{		("Divisón por un valor cercano a cero: " + double2str(denom))}
\PY{c+cp}{\PYZsh{}}\PY{c+cp}{define	 DIMENSIONS\PYZus{}MISMATCH\PYZus{}TRANSP\PYZus{}INPLACE(rows,cols)	\PYZbs{}}
\PY{c+cp}{		("No se puede trasponer in situ porque la matriz no es cuadrada; filas: " + int2str(rows) + ", columnas: " + int2str(cols))}
\PY{c+cp}{\PYZsh{}}\PY{c+cp}{define	 IN\PYZus{}RANGE(a,x,b)				(((a) \PYZlt{}= (x)) \PYZam{}\PYZam{} ((x) \PYZlt{} (b)))}
\PY{c+cp}{\PYZsh{}}\PY{c+cp}{define	 NOT\PYZus{}IN\PYZus{}RANGE(a,x,b)			(!IN\PYZus{}RANGE((a),(x),(b)))}

\PY{c+c1}{//	//	//	//}

\PY{n}{MMatrix}\PY{o}{:}\PY{o}{:}\PY{n}{MMatrix}\PY{p}{(}\PY{p}{)}
\PY{p}{\PYZob{}}
	\PY{n}{initialize}\PY{p}{(}\PY{p}{)}\PY{p}{;}
\PY{p}{\PYZcb{}}

\PY{n}{MMatrix}\PY{o}{:}\PY{o}{:}\PY{n}{MMatrix}\PY{p}{(}\PY{k+kt}{int} \PY{n}{rows}\PY{p}{,} \PY{k+kt}{int} \PY{n}{cols}\PY{p}{)}
\PY{p}{\PYZob{}}
	\PY{n}{initialize}\PY{p}{(}\PY{p}{)}\PY{p}{;}
	\PY{n}{set\PYZus{}size}\PY{p}{(}\PY{n}{rows}\PY{p}{,} \PY{n}{cols}\PY{p}{)}\PY{p}{;}
\PY{p}{\PYZcb{}}

\PY{n}{MMatrix}\PY{o}{:}\PY{o}{:}\PY{n}{MMatrix}\PY{p}{(}\PY{k+kt}{int} \PY{n}{rows}\PY{p}{,} \PY{k+kt}{int} \PY{n}{cols}\PY{p}{,} \PY{k+kt}{double} \PY{n}{value}\PY{p}{)}
\PY{p}{\PYZob{}}
	\PY{n}{initialize}\PY{p}{(}\PY{p}{)}\PY{p}{;}
	\PY{n}{set\PYZus{}size}\PY{p}{(}\PY{n}{rows}\PY{p}{,} \PY{n}{cols}\PY{p}{)}\PY{p}{;}
	\PY{n}{foreach\PYZus{}a\PYZus{}ij}\PY{p}{(}\PY{o}{*}\PY{k}{this}\PY{p}{,} \PY{n}{a\PYZus{}ij} \PY{o}{=} \PY{n}{value}\PY{p}{)}\PY{p}{;}
\PY{p}{\PYZcb{}}

\PY{n}{MMatrix}\PY{o}{:}\PY{o}{:}\PY{n}{MMatrix}\PY{p}{(}\PY{k}{const} \PY{n}{MMatrix}\PY{o}{\PYZam{}} \PY{n}{mat}\PY{p}{)}
\PY{p}{\PYZob{}}
	\PY{n}{initialize}\PY{p}{(}\PY{p}{)}\PY{p}{;}
	\PY{n}{set\PYZus{}size}\PY{p}{(}\PY{n}{mat}\PY{p}{.}\PY{n}{rows}\PY{p}{(}\PY{p}{)}\PY{p}{,} \PY{n}{mat}\PY{p}{.}\PY{n}{cols}\PY{p}{(}\PY{p}{)}\PY{p}{)}\PY{p}{;}
	\PY{n}{foreach\PYZus{}a\PYZus{}ij}\PY{p}{(}\PY{o}{*}\PY{k}{this}\PY{p}{,} \PY{n}{a\PYZus{}ij} \PY{o}{=} \PY{n}{mat}\PY{p}{(}\PY{n}{i}\PY{p}{,}\PY{n}{j}\PY{p}{)}\PY{p}{)}\PY{p}{;}
\PY{p}{\PYZcb{}}

\PY{n}{MMatrix}\PY{o}{:}\PY{o}{:}\PY{o}{\PYZti{}}\PY{n}{MMatrix}\PY{p}{(}\PY{p}{)}
\PY{p}{\PYZob{}}
	\PY{n}{clean\PYZus{}up}\PY{p}{(}\PY{p}{)}\PY{p}{;}
\PY{p}{\PYZcb{}}

\PY{k+kt}{void} \PY{n}{MMatrix}\PY{o}{:}\PY{o}{:}\PY{n}{initialize}\PY{p}{(}\PY{p}{)}
\PY{p}{\PYZob{}}
	\PY{n}{\PYZus{}data} \PY{o}{=} \PY{l+m+mi}{0}\PY{p}{;}
	\PY{n}{\PYZus{}rows} \PY{o}{=} \PY{l+m+mi}{0}\PY{p}{;}
	\PY{n}{\PYZus{}cols} \PY{o}{=} \PY{l+m+mi}{0}\PY{p}{;}
\PY{p}{\PYZcb{}}

\PY{k+kt}{void} \PY{n}{MMatrix}\PY{o}{:}\PY{o}{:}\PY{n}{clean\PYZus{}up}\PY{p}{(}\PY{p}{)}
\PY{p}{\PYZob{}}
	\PY{k}{if}\PY{p}{(}\PY{n}{\PYZus{}data} \PY{o}{!}\PY{o}{=} \PY{l+m+mi}{0}\PY{p}{)}
		\PY{k}{delete}\PY{p}{[}\PY{p}{]} \PY{n}{\PYZus{}data}\PY{p}{;}
\PY{p}{\PYZcb{}}

\PY{k+kt}{void} \PY{n}{MMatrix}\PY{o}{:}\PY{o}{:}\PY{n}{set\PYZus{}size}\PY{p}{(}\PY{k+kt}{int} \PY{n}{rows}\PY{p}{,} \PY{k+kt}{int} \PY{n}{cols}\PY{p}{)}
\PY{p}{\PYZob{}}
	\PY{k}{if}\PY{p}{(}\PY{n}{rows} \PY{o}{\PYZlt{}}\PY{o}{=} \PY{l+m+mi}{0} \PY{o}{|}\PY{o}{|} \PY{n}{cols} \PY{o}{\PYZlt{}}\PY{o}{=} \PY{l+m+mi}{0}\PY{p}{)}
		\PY{n}{DISPLAY\PYZus{}ERROR\PYZus{}AND\PYZus{}EXIT}\PY{p}{(}\PY{n}{INVALID\PYZus{}MATRIX\PYZus{}SIZE}\PY{p}{(}\PY{n}{rows}\PY{p}{,}\PY{n}{cols}\PY{p}{)}\PY{p}{)}\PY{p}{;}

	\PY{n}{clean\PYZus{}up}\PY{p}{(}\PY{p}{)}\PY{p}{;}
	\PY{n}{\PYZus{}data} \PY{o}{=} \PY{k}{new} \PY{k+kt}{double}\PY{p}{[}\PY{n}{rows}\PY{o}{*}\PY{n}{cols}\PY{p}{]}\PY{p}{;}
	\PY{n}{\PYZus{}rows} \PY{o}{=} \PY{n}{rows}\PY{p}{;}
	\PY{n}{\PYZus{}cols} \PY{o}{=} \PY{n}{cols}\PY{p}{;}
\PY{p}{\PYZcb{}}

\PY{k+kt}{int} \PY{n}{MMatrix}\PY{o}{:}\PY{o}{:}\PY{n}{rows}\PY{p}{(}\PY{p}{)} \PY{k}{const}
\PY{p}{\PYZob{}}
	\PY{k}{return} \PY{n}{\PYZus{}rows}\PY{p}{;}
\PY{p}{\PYZcb{}}

\PY{k+kt}{int} \PY{n}{MMatrix}\PY{o}{:}\PY{o}{:}\PY{n}{cols}\PY{p}{(}\PY{p}{)} \PY{k}{const}
\PY{p}{\PYZob{}}
	\PY{k}{return} \PY{n}{\PYZus{}cols}\PY{p}{;}
\PY{p}{\PYZcb{}}

\PY{k+kt}{int} \PY{n}{MMatrix}\PY{o}{:}\PY{o}{:}\PY{n}{size}\PY{p}{(}\PY{p}{)} \PY{k}{const}
\PY{p}{\PYZob{}}
	\PY{k}{return} \PY{n}{MAX}\PY{p}{(}\PY{n}{\PYZus{}rows}\PY{p}{,}\PY{n}{\PYZus{}cols}\PY{p}{)}\PY{p}{;}
\PY{p}{\PYZcb{}}

\PY{n}{MMatrix} \PY{n}{MMatrix}\PY{o}{:}\PY{o}{:}\PY{n}{row}\PY{p}{(}\PY{k+kt}{int} \PY{n}{r}\PY{p}{)} \PY{k}{const}
\PY{p}{\PYZob{}}
	\PY{n}{MMatrix} \PY{n}{row}\PY{p}{(}\PY{l+m+mi}{1}\PY{p}{,} \PY{n}{\PYZus{}cols}\PY{p}{)}\PY{p}{;}
	\PY{n}{foreach\PYZus{}v\PYZus{}i}\PY{p}{(}\PY{n}{row}\PY{p}{,} \PY{n}{v\PYZus{}i} \PY{o}{=} \PY{k}{operator}\PY{p}{(}\PY{p}{)}\PY{p}{(}\PY{n}{r}\PY{p}{,} \PY{n}{i}\PY{p}{)}\PY{p}{)}\PY{p}{;}

	\PY{k}{return} \PY{n}{row}\PY{p}{;}
\PY{p}{\PYZcb{}}

\PY{k+kt}{void} \PY{n}{MMatrix}\PY{o}{:}\PY{o}{:}\PY{n}{copy\PYZus{}row}\PY{p}{(}\PY{k+kt}{int} \PY{n}{r}\PY{p}{,} \PY{n}{MMatrix}\PY{o}{\PYZam{}} \PY{n}{row}\PY{p}{)} \PY{k}{const}
\PY{p}{\PYZob{}}
	\PY{k}{if}\PY{p}{(}\PY{n}{row}\PY{p}{.}\PY{n}{rows}\PY{p}{(}\PY{p}{)} \PY{o}{!}\PY{o}{=} \PY{l+m+mi}{1} \PY{o}{|}\PY{o}{|} \PY{n}{row}\PY{p}{.}\PY{n}{cols}\PY{p}{(}\PY{p}{)} \PY{o}{!}\PY{o}{=} \PY{n}{\PYZus{}cols}\PY{p}{)}
		\PY{n}{row}\PY{p}{.}\PY{n}{set\PYZus{}size}\PY{p}{(}\PY{l+m+mi}{1}\PY{p}{,} \PY{n}{\PYZus{}cols}\PY{p}{)}\PY{p}{;}

	\PY{n}{foreach\PYZus{}v\PYZus{}i}\PY{p}{(}\PY{n}{row}\PY{p}{,} \PY{n}{v\PYZus{}i} \PY{o}{=} \PY{k}{operator}\PY{p}{(}\PY{p}{)}\PY{p}{(}\PY{n}{r}\PY{p}{,} \PY{n}{i}\PY{p}{)}\PY{p}{)}\PY{p}{;}
\PY{p}{\PYZcb{}}

\PY{n}{MMatrix} \PY{n}{MMatrix}\PY{o}{:}\PY{o}{:}\PY{n}{col}\PY{p}{(}\PY{k+kt}{int} \PY{n}{c}\PY{p}{)} \PY{k}{const}
\PY{p}{\PYZob{}}
	\PY{n}{MMatrix} \PY{n}{col}\PY{p}{(}\PY{n}{\PYZus{}rows}\PY{p}{,} \PY{l+m+mi}{1}\PY{p}{)}\PY{p}{;}
	\PY{n}{foreach\PYZus{}v\PYZus{}i}\PY{p}{(}\PY{n}{col}\PY{p}{,} \PY{n}{v\PYZus{}i} \PY{o}{=} \PY{k}{operator}\PY{p}{(}\PY{p}{)}\PY{p}{(}\PY{n}{i}\PY{p}{,} \PY{n}{c}\PY{p}{)}\PY{p}{)}\PY{p}{;}

	\PY{k}{return} \PY{n}{col}\PY{p}{;}
\PY{p}{\PYZcb{}}

\PY{k+kt}{void} \PY{n}{MMatrix}\PY{o}{:}\PY{o}{:}\PY{n}{copy\PYZus{}col}\PY{p}{(}\PY{k+kt}{int} \PY{n}{c}\PY{p}{,} \PY{n}{MMatrix}\PY{o}{\PYZam{}} \PY{n}{col}\PY{p}{)} \PY{k}{const}
\PY{p}{\PYZob{}}
	\PY{k}{if}\PY{p}{(}\PY{n}{col}\PY{p}{.}\PY{n}{rows}\PY{p}{(}\PY{p}{)} \PY{o}{!}\PY{o}{=} \PY{n}{\PYZus{}rows} \PY{o}{|}\PY{o}{|} \PY{n}{col}\PY{p}{.}\PY{n}{cols}\PY{p}{(}\PY{p}{)} \PY{o}{!}\PY{o}{=} \PY{l+m+mi}{1}\PY{p}{)}
		\PY{n}{col}\PY{p}{.}\PY{n}{set\PYZus{}size}\PY{p}{(}\PY{n}{\PYZus{}rows}\PY{p}{,} \PY{l+m+mi}{1}\PY{p}{)}\PY{p}{;}

	\PY{n}{foreach\PYZus{}v\PYZus{}i}\PY{p}{(}\PY{n}{col}\PY{p}{,} \PY{n}{v\PYZus{}i} \PY{o}{=} \PY{k}{operator}\PY{p}{(}\PY{p}{)}\PY{p}{(}\PY{n}{i}\PY{p}{,} \PY{n}{c}\PY{p}{)}\PY{p}{)}\PY{p}{;}
\PY{p}{\PYZcb{}}

\PY{n}{MMatrix}\PY{o}{\PYZam{}} \PY{n}{MMatrix}\PY{o}{:}\PY{o}{:}\PY{k}{operator}\PY{o}{=}\PY{p}{(}\PY{k}{const} \PY{n}{MMatrix}\PY{o}{\PYZam{}} \PY{n}{rvalue}\PY{p}{)}
\PY{p}{\PYZob{}}
	\PY{k}{if}\PY{p}{(}\PY{n}{\PYZus{}rows} \PY{o}{!}\PY{o}{=} \PY{n}{rvalue}\PY{p}{.}\PY{n}{rows}\PY{p}{(}\PY{p}{)} \PY{o}{|}\PY{o}{|} \PY{n}{\PYZus{}cols} \PY{o}{!}\PY{o}{=} \PY{n}{rvalue}\PY{p}{.}\PY{n}{cols}\PY{p}{(}\PY{p}{)}\PY{p}{)}
		\PY{n}{set\PYZus{}size}\PY{p}{(}\PY{n}{rvalue}\PY{p}{.}\PY{n}{rows}\PY{p}{(}\PY{p}{)}\PY{p}{,} \PY{n}{rvalue}\PY{p}{.}\PY{n}{cols}\PY{p}{(}\PY{p}{)}\PY{p}{)}\PY{p}{;}

	\PY{n}{foreach\PYZus{}a\PYZus{}ij}\PY{p}{(}\PY{o}{*}\PY{k}{this}\PY{p}{,} \PY{n}{a\PYZus{}ij} \PY{o}{=} \PY{n}{rvalue}\PY{p}{(}\PY{n}{i}\PY{p}{,}\PY{n}{j}\PY{p}{)}\PY{p}{)}\PY{p}{;}

	\PY{k}{return} \PY{o}{*}\PY{k}{this}\PY{p}{;}
\PY{p}{\PYZcb{}}

\PY{k+kt}{double}\PY{o}{\PYZam{}} \PY{n}{MMatrix}\PY{o}{:}\PY{o}{:}\PY{k}{operator}\PY{p}{(}\PY{p}{)}\PY{p}{(}\PY{k+kt}{int} \PY{n}{n}\PY{p}{)}
\PY{p}{\PYZob{}}
	\PY{k}{if}\PY{p}{(}\PY{n}{NOT\PYZus{}IN\PYZus{}RANGE}\PY{p}{(}\PY{l+m+mi}{0}\PY{p}{,} \PY{n}{n}\PY{p}{,} \PY{n}{\PYZus{}rows} \PY{o}{*} \PY{n}{\PYZus{}cols}\PY{p}{)}\PY{p}{)}
		\PY{n}{DISPLAY\PYZus{}ERROR\PYZus{}AND\PYZus{}EXIT}\PY{p}{(}\PY{n}{OUT\PYZus{}OF\PYZus{}BOUNDS\PYZus{}LINEAR}\PY{p}{(}\PY{n}{n}\PY{p}{)}\PY{p}{)}\PY{p}{;}

	\PY{k}{return} \PY{n}{\PYZus{}data}\PY{p}{[}\PY{n}{n}\PY{p}{]}\PY{p}{;}
\PY{p}{\PYZcb{}}

\PY{k+kt}{double} \PY{n}{MMatrix}\PY{o}{:}\PY{o}{:}\PY{k}{operator}\PY{p}{(}\PY{p}{)}\PY{p}{(}\PY{k+kt}{int} \PY{n}{n}\PY{p}{)} \PY{k}{const}
\PY{p}{\PYZob{}}
	\PY{k}{if}\PY{p}{(}\PY{n}{NOT\PYZus{}IN\PYZus{}RANGE}\PY{p}{(}\PY{l+m+mi}{0}\PY{p}{,} \PY{n}{n}\PY{p}{,} \PY{n}{\PYZus{}rows} \PY{o}{*} \PY{n}{\PYZus{}cols}\PY{p}{)}\PY{p}{)}
		\PY{n}{DISPLAY\PYZus{}ERROR\PYZus{}AND\PYZus{}EXIT}\PY{p}{(}\PY{n}{OUT\PYZus{}OF\PYZus{}BOUNDS\PYZus{}LINEAR}\PY{p}{(}\PY{n}{n}\PY{p}{)}\PY{p}{)}\PY{p}{;}

	\PY{k}{return} \PY{n}{\PYZus{}data}\PY{p}{[}\PY{n}{n}\PY{p}{]}\PY{p}{;}
\PY{p}{\PYZcb{}}

\PY{k+kt}{double}\PY{o}{\PYZam{}} \PY{n}{MMatrix}\PY{o}{:}\PY{o}{:}\PY{k}{operator}\PY{p}{(}\PY{p}{)}\PY{p}{(}\PY{k+kt}{int} \PY{n}{i}\PY{p}{,} \PY{k+kt}{int} \PY{n}{j}\PY{p}{)}
\PY{p}{\PYZob{}}
	\PY{k}{if}\PY{p}{(}\PY{n}{NOT\PYZus{}IN\PYZus{}RANGE}\PY{p}{(}\PY{l+m+mi}{0}\PY{p}{,}\PY{n}{i}\PY{p}{,}\PY{n}{\PYZus{}rows}\PY{p}{)} \PY{o}{|}\PY{o}{|} \PY{n}{NOT\PYZus{}IN\PYZus{}RANGE}\PY{p}{(}\PY{l+m+mi}{0}\PY{p}{,}\PY{n}{j}\PY{p}{,}\PY{n}{\PYZus{}cols}\PY{p}{)}\PY{p}{)}
		\PY{n}{DISPLAY\PYZus{}ERROR\PYZus{}AND\PYZus{}EXIT}\PY{p}{(}\PY{n}{OUT\PYZus{}OF\PYZus{}BOUNDS}\PY{p}{(}\PY{n}{i}\PY{p}{,}\PY{n}{j}\PY{p}{)}\PY{p}{)}\PY{p}{;}

	\PY{k}{return} \PY{n}{\PYZus{}data}\PY{p}{[}\PY{n}{i}\PY{o}{*}\PY{n}{\PYZus{}cols}\PY{o}{+}\PY{n}{j}\PY{p}{]}\PY{p}{;}
\PY{p}{\PYZcb{}}

\PY{k+kt}{double} \PY{n}{MMatrix}\PY{o}{:}\PY{o}{:}\PY{k}{operator}\PY{p}{(}\PY{p}{)}\PY{p}{(}\PY{k+kt}{int} \PY{n}{i}\PY{p}{,} \PY{k+kt}{int} \PY{n}{j}\PY{p}{)} \PY{k}{const}
\PY{p}{\PYZob{}}
	\PY{k}{if}\PY{p}{(}\PY{n}{NOT\PYZus{}IN\PYZus{}RANGE}\PY{p}{(}\PY{l+m+mi}{0}\PY{p}{,}\PY{n}{i}\PY{p}{,}\PY{n}{\PYZus{}rows}\PY{p}{)} \PY{o}{|}\PY{o}{|} \PY{n}{NOT\PYZus{}IN\PYZus{}RANGE}\PY{p}{(}\PY{l+m+mi}{0}\PY{p}{,}\PY{n}{j}\PY{p}{,}\PY{n}{\PYZus{}cols}\PY{p}{)}\PY{p}{)}
		\PY{n}{DISPLAY\PYZus{}ERROR\PYZus{}AND\PYZus{}EXIT}\PY{p}{(}\PY{n}{OUT\PYZus{}OF\PYZus{}BOUNDS}\PY{p}{(}\PY{n}{i}\PY{p}{,}\PY{n}{j}\PY{p}{)}\PY{p}{)}\PY{p}{;}
	
	\PY{k}{return} \PY{n}{\PYZus{}data}\PY{p}{[}\PY{n}{i}\PY{o}{*}\PY{n}{\PYZus{}cols}\PY{o}{+}\PY{n}{j}\PY{p}{]}\PY{p}{;}
\PY{p}{\PYZcb{}}

\PY{n}{MMatrix} \PY{n}{MMatrix}\PY{o}{:}\PY{o}{:}\PY{k}{operator}\PY{o}{-}\PY{p}{(}\PY{k}{const} \PY{n}{MMatrix}\PY{o}{\PYZam{}} \PY{n}{rhs}\PY{p}{)} \PY{k}{const}
\PY{p}{\PYZob{}}
	\PY{k}{if}\PY{p}{(}\PY{n}{\PYZus{}rows} \PY{o}{!}\PY{o}{=} \PY{n}{rhs}\PY{p}{.}\PY{n}{rows}\PY{p}{(}\PY{p}{)} \PY{o}{|}\PY{o}{|} \PY{n}{\PYZus{}cols} \PY{o}{!}\PY{o}{=} \PY{n}{rhs}\PY{p}{.}\PY{n}{cols}\PY{p}{(}\PY{p}{)}\PY{p}{)}
		\PY{n}{DISPLAY\PYZus{}ERROR\PYZus{}AND\PYZus{}EXIT}\PY{p}{(}\PY{n}{DIMENSIONS\PYZus{}MISMATCH}\PY{p}{(}\PY{n}{\PYZus{}rows}\PY{p}{,} \PY{n}{rhs}\PY{p}{.}\PY{n}{rows}\PY{p}{(}\PY{p}{)}\PY{p}{,} \PY{n}{\PYZus{}cols}\PY{p}{,} \PY{n}{rhs}\PY{p}{.}\PY{n}{cols}\PY{p}{(}\PY{p}{)}\PY{p}{)}\PY{p}{)}\PY{p}{;}

	\PY{n}{MMatrix} \PY{n}{res}\PY{p}{(}\PY{n}{\PYZus{}rows}\PY{p}{,} \PY{n}{\PYZus{}cols}\PY{p}{)}\PY{p}{;}
	\PY{n}{foreach\PYZus{}a\PYZus{}ij}\PY{p}{(}\PY{n}{res}\PY{p}{,} \PY{n}{a\PYZus{}ij} \PY{o}{=} \PY{k}{operator}\PY{p}{(}\PY{p}{)}\PY{p}{(}\PY{n}{i}\PY{p}{,}\PY{n}{j}\PY{p}{)} \PY{o}{-} \PY{n}{rhs}\PY{p}{(}\PY{n}{i}\PY{p}{,}\PY{n}{j}\PY{p}{)}\PY{p}{)}\PY{p}{;}

	\PY{k}{return} \PY{n}{res}\PY{p}{;}
\PY{p}{\PYZcb{}}

\PY{n}{MMatrix}\PY{o}{\PYZam{}} \PY{n}{MMatrix}\PY{o}{:}\PY{o}{:}\PY{k}{operator}\PY{o}{*}\PY{o}{=}\PY{p}{(}\PY{k+kt}{double} \PY{n}{rhs}\PY{p}{)}
\PY{p}{\PYZob{}}
	\PY{n}{foreach\PYZus{}a\PYZus{}ij}\PY{p}{(}\PY{o}{*}\PY{k}{this}\PY{p}{,} \PY{n}{a\PYZus{}ij} \PY{o}{=} \PY{k}{operator}\PY{p}{(}\PY{p}{)}\PY{p}{(}\PY{n}{i}\PY{p}{,}\PY{n}{j}\PY{p}{)} \PY{o}{*} \PY{n}{rhs}\PY{p}{)}\PY{p}{;}

	\PY{k}{return} \PY{o}{*}\PY{k}{this}\PY{p}{;}
\PY{p}{\PYZcb{}}

\PY{n}{MMatrix}\PY{o}{\PYZam{}} \PY{n}{MMatrix}\PY{o}{:}\PY{o}{:}\PY{k}{operator}\PY{o}{/}\PY{o}{=}\PY{p}{(}\PY{k+kt}{double} \PY{n}{rhs}\PY{p}{)}
\PY{p}{\PYZob{}}
	\PY{k}{if}\PY{p}{(} \PY{n}{abs}\PY{p}{(}\PY{n}{rhs}\PY{p}{)} \PY{o}{\PYZlt{}} \PY{n}{DBL\PYZus{}TOLERANCE\PYZus{}2\PYZus{}ZERO} \PY{p}{)}
		\PY{n}{DISPLAY\PYZus{}ERROR\PYZus{}AND\PYZus{}EXIT}\PY{p}{(}\PY{n}{NEAR\PYZus{}ZERO\PYZus{}DIVISION}\PY{p}{(}\PY{n}{rhs}\PY{p}{)}\PY{p}{)}\PY{p}{;}

	\PY{n}{foreach\PYZus{}a\PYZus{}ij}\PY{p}{(}\PY{o}{*}\PY{k}{this}\PY{p}{,} \PY{n}{a\PYZus{}ij} \PY{o}{=} \PY{k}{operator}\PY{p}{(}\PY{p}{)}\PY{p}{(}\PY{n}{i}\PY{p}{,}\PY{n}{j}\PY{p}{)} \PY{o}{/} \PY{n}{rhs}\PY{p}{)}\PY{p}{;}

	\PY{k}{return} \PY{o}{*}\PY{k}{this}\PY{p}{;}
\PY{p}{\PYZcb{}}

\PY{n}{MMatrix} \PY{n}{MMatrix}\PY{o}{:}\PY{o}{:}\PY{k}{operator}\PY{o}{*}\PY{p}{(}\PY{k}{const} \PY{n}{MMatrix}\PY{o}{\PYZam{}} \PY{n}{rhs}\PY{p}{)} \PY{k}{const}
\PY{p}{\PYZob{}}
	\PY{k}{if}\PY{p}{(}\PY{n}{\PYZus{}cols} \PY{o}{!}\PY{o}{=} \PY{n}{rhs}\PY{p}{.}\PY{n}{rows}\PY{p}{(}\PY{p}{)}\PY{p}{)}
		\PY{n}{DISPLAY\PYZus{}ERROR\PYZus{}AND\PYZus{}EXIT}\PY{p}{(}\PY{n}{DIMENSIONS\PYZus{}MISMATCH}\PY{p}{(}\PY{n}{\PYZus{}rows}\PY{p}{,} \PY{n}{rhs}\PY{p}{.}\PY{n}{rows}\PY{p}{(}\PY{p}{)}\PY{p}{,} \PY{n}{\PYZus{}cols}\PY{p}{,} \PY{n}{rhs}\PY{p}{.}\PY{n}{cols}\PY{p}{(}\PY{p}{)}\PY{p}{)}\PY{p}{)}\PY{p}{;}

	\PY{n}{MMatrix} \PY{n}{res}\PY{p}{(}\PY{n}{\PYZus{}rows}\PY{p}{,} \PY{n}{rhs}\PY{p}{.}\PY{n}{cols}\PY{p}{(}\PY{p}{)}\PY{p}{)}\PY{p}{;}
	\PY{n}{foreach\PYZus{}a\PYZus{}ij}\PY{p}{(}\PY{n}{res}\PY{p}{,} \PY{n}{a\PYZus{}ij} \PY{o}{=} \PY{n}{dot\PYZus{}row\PYZus{}col}\PY{p}{(}\PY{o}{*}\PY{k}{this}\PY{p}{,} \PY{n}{i}\PY{p}{,} \PY{n}{rhs}\PY{p}{,} \PY{n}{j}\PY{p}{)}\PY{p}{)}\PY{p}{;}

	\PY{k}{return} \PY{n}{res}\PY{p}{;}
\PY{p}{\PYZcb{}}

\PY{n}{MMatrix}\PY{o}{\PYZam{}} \PY{n}{MMatrix}\PY{o}{:}\PY{o}{:}\PY{k}{operator}\PY{o}{*}\PY{o}{=}\PY{p}{(}\PY{k}{const} \PY{n}{MMatrix}\PY{o}{\PYZam{}} \PY{n}{rhs}\PY{p}{)}
\PY{p}{\PYZob{}}
	\PY{k}{if}\PY{p}{(}\PY{n}{\PYZus{}cols} \PY{o}{!}\PY{o}{=} \PY{n}{rhs}\PY{p}{.}\PY{n}{rows}\PY{p}{(}\PY{p}{)} \PY{o}{|}\PY{o}{|} \PY{n}{rhs}\PY{p}{.}\PY{n}{rows}\PY{p}{(}\PY{p}{)} \PY{o}{!}\PY{o}{=} \PY{n}{rhs}\PY{p}{.}\PY{n}{cols}\PY{p}{(}\PY{p}{)}\PY{p}{)}
		\PY{n}{DISPLAY\PYZus{}ERROR\PYZus{}AND\PYZus{}EXIT}\PY{p}{(}\PY{n}{DIMENSIONS\PYZus{}MISMATCH\PYZus{}MULT\PYZus{}INPLACE}\PY{p}{(}\PY{n}{\PYZus{}rows}\PY{p}{,} \PY{n}{rhs}\PY{p}{.}\PY{n}{rows}\PY{p}{(}\PY{p}{)}\PY{p}{,} \PY{n}{\PYZus{}cols}\PY{p}{,} \PY{n}{rhs}\PY{p}{.}\PY{n}{cols}\PY{p}{(}\PY{p}{)}\PY{p}{)}\PY{p}{)}\PY{p}{;}

	\PY{n}{MMatrix} \PY{n}{aux\PYZus{}row}\PY{p}{;}
	\PY{n}{foreach\PYZus{}a\PYZus{}ij}\PY{p}{(}\PY{o}{*}\PY{k}{this}\PY{p}{,}\PY{p}{\PYZob{}}
		\PY{k}{if}\PY{p}{(}\PY{n}{j} \PY{o}{=}\PY{o}{=} \PY{l+m+mi}{0}\PY{p}{)} \PY{n}{copy\PYZus{}row}\PY{p}{(}\PY{n}{i}\PY{p}{,} \PY{n}{aux\PYZus{}row}\PY{p}{)}\PY{p}{;}
		\PY{n}{a\PYZus{}ij} \PY{o}{=} \PY{n}{dot\PYZus{}row\PYZus{}col}\PY{p}{(}\PY{n}{aux\PYZus{}row}\PY{p}{,} \PY{l+m+mi}{0}\PY{p}{,} \PY{n}{rhs}\PY{p}{,} \PY{n}{j}\PY{p}{)}\PY{p}{;}
	\PY{p}{\PYZcb{}}\PY{p}{)}\PY{p}{;}

	\PY{k}{return} \PY{o}{*}\PY{k}{this}\PY{p}{;}
\PY{p}{\PYZcb{}}

\PY{n}{MMatrix} \PY{n}{MMatrix}\PY{o}{:}\PY{o}{:}\PY{n}{t}\PY{p}{(}\PY{p}{)} \PY{k}{const}
\PY{p}{\PYZob{}}
	\PY{n}{MMatrix} \PY{n}{res}\PY{p}{(}\PY{n}{\PYZus{}cols}\PY{p}{,} \PY{n}{\PYZus{}rows}\PY{p}{)}\PY{p}{;}
	\PY{n}{foreach\PYZus{}a\PYZus{}ij}\PY{p}{(}\PY{n}{res}\PY{p}{,} \PY{n}{a\PYZus{}ij} \PY{o}{=} \PY{k}{operator}\PY{p}{(}\PY{p}{)}\PY{p}{(}\PY{n}{j}\PY{p}{,}\PY{n}{i}\PY{p}{)}\PY{p}{)}\PY{p}{;}

	\PY{k}{return} \PY{n}{res}\PY{p}{;}
\PY{p}{\PYZcb{}}

\PY{n}{MMatrix}\PY{o}{\PYZam{}} \PY{n}{MMatrix}\PY{o}{:}\PY{o}{:}\PY{n}{t\PYZus{}in\PYZus{}place}\PY{p}{(}\PY{p}{)}
\PY{p}{\PYZob{}}
	\PY{k}{if}\PY{p}{(}\PY{n}{\PYZus{}rows} \PY{o}{!}\PY{o}{=} \PY{n}{\PYZus{}cols}\PY{p}{)}
		\PY{n}{DISPLAY\PYZus{}ERROR\PYZus{}AND\PYZus{}EXIT}\PY{p}{(}\PY{n}{DIMENSIONS\PYZus{}MISMATCH\PYZus{}TRANSP\PYZus{}INPLACE}\PY{p}{(}\PY{n}{\PYZus{}rows}\PY{p}{,}\PY{n}{\PYZus{}cols}\PY{p}{)}\PY{p}{)}\PY{p}{;}

	\PY{n}{foreach\PYZus{}a\PYZus{}ij\PYZus{}lower\PYZus{}triangular}\PY{p}{(}\PY{o}{*}\PY{k}{this}\PY{p}{,}\PY{p}{\PYZob{}}
		\PY{n}{swap}\PY{p}{(}\PY{n}{a\PYZus{}ij}\PY{p}{,} \PY{k}{operator}\PY{p}{(}\PY{p}{)}\PY{p}{(}\PY{n}{j}\PY{p}{,}\PY{n}{i}\PY{p}{)}\PY{p}{)}\PY{p}{;}
	\PY{p}{\PYZcb{}}\PY{p}{)}\PY{p}{;}

	\PY{k}{return} \PY{o}{*}\PY{k}{this}\PY{p}{;}
\PY{p}{\PYZcb{}}

\PY{n}{MMatrix} \PY{n}{MMatrix}\PY{o}{:}\PY{o}{:}\PY{n}{identity\PYZus{}matrix}\PY{p}{(}\PY{k+kt}{int} \PY{n}{size}\PY{p}{)}
\PY{p}{\PYZob{}}
	\PY{n}{MMatrix} \PY{n}{res}\PY{p}{(}\PY{n}{size}\PY{p}{,} \PY{n}{size}\PY{p}{)}\PY{p}{;}
	\PY{n}{foreach\PYZus{}a\PYZus{}ij}\PY{p}{(}\PY{n}{res}\PY{p}{,} \PY{n}{a\PYZus{}ij} \PY{o}{=} \PY{p}{(}\PY{p}{(}\PY{n}{i} \PY{o}{=}\PY{o}{=} \PY{n}{j}\PY{p}{)}\PY{o}{?}\PY{p}{(}\PY{l+m+mi}{1}\PY{p}{)}\PY{o}{:}\PY{p}{(}\PY{l+m+mi}{0}\PY{p}{)}\PY{p}{)}\PY{p}{)}\PY{p}{;}

	\PY{k}{return} \PY{n}{res}\PY{p}{;}
\PY{p}{\PYZcb{}}

\PY{n}{MMatrix}\PY{o}{\PYZam{}} \PY{n}{MMatrix}\PY{o}{:}\PY{o}{:}\PY{n}{make\PYZus{}identity\PYZus{}matrix}\PY{p}{(}\PY{k+kt}{int} \PY{n}{size}\PY{p}{)}
\PY{p}{\PYZob{}}
	\PY{k}{if}\PY{p}{(}\PY{n}{\PYZus{}rows} \PY{o}{!}\PY{o}{=} \PY{n}{size} \PY{o}{|}\PY{o}{|} \PY{n}{\PYZus{}cols} \PY{o}{!}\PY{o}{=} \PY{n}{size}\PY{p}{)}
		\PY{n}{set\PYZus{}size}\PY{p}{(}\PY{n}{size}\PY{p}{,} \PY{n}{size}\PY{p}{)}\PY{p}{;}

	\PY{n}{foreach\PYZus{}a\PYZus{}ij}\PY{p}{(}\PY{o}{*}\PY{k}{this}\PY{p}{,} \PY{n}{a\PYZus{}ij} \PY{o}{=} \PY{p}{(}\PY{p}{(}\PY{n}{i} \PY{o}{=}\PY{o}{=} \PY{n}{j}\PY{p}{)}\PY{o}{?}\PY{p}{(}\PY{l+m+mi}{1}\PY{p}{)}\PY{o}{:}\PY{p}{(}\PY{l+m+mi}{0}\PY{p}{)}\PY{p}{)}\PY{p}{)}\PY{p}{;}

	\PY{k}{return} \PY{o}{*}\PY{k}{this}\PY{p}{;}
\PY{p}{\PYZcb{}}

\PY{n}{ostream}\PY{o}{\PYZam{}} \PY{k}{operator}\PY{o}{\PYZlt{}}\PY{o}{\PYZlt{}}\PY{p}{(}\PY{n}{ostream} \PY{o}{\PYZam{}}\PY{n}{os}\PY{p}{,} \PY{k}{const} \PY{n}{MMatrix} \PY{o}{\PYZam{}}\PY{n}{mat}\PY{p}{)}
\PY{p}{\PYZob{}}
	\PY{n}{os} \PY{o}{\PYZlt{}}\PY{o}{\PYZlt{}} \PY{l+s}{"}\PY{l+s}{[}\PY{l+s}{"}\PY{p}{;}
	\PY{k}{for} \PY{p}{(}\PY{k+kt}{int} \PY{n}{i} \PY{o}{=} \PY{l+m+mi}{0}\PY{p}{;} \PY{n}{i} \PY{o}{\PYZlt{}} \PY{n}{mat}\PY{p}{.}\PY{n}{rows}\PY{p}{(}\PY{p}{)}\PY{p}{;} \PY{o}{+}\PY{o}{+}\PY{n}{i}\PY{p}{)}
	\PY{p}{\PYZob{}}
		\PY{k}{for} \PY{p}{(}\PY{k+kt}{int} \PY{n}{j} \PY{o}{=} \PY{l+m+mi}{0}\PY{p}{;} \PY{n}{j} \PY{o}{\PYZlt{}} \PY{n}{mat}\PY{p}{.}\PY{n}{cols}\PY{p}{(}\PY{p}{)}\PY{p}{;} \PY{o}{+}\PY{o}{+}\PY{n}{j}\PY{p}{)}
		\PY{p}{\PYZob{}}
			\PY{k}{if}\PY{p}{(} \PY{l+m+mi}{0} \PY{o}{\PYZlt{}} \PY{n}{i} \PY{o}{\PYZam{}}\PY{o}{\PYZam{}} \PY{n}{j} \PY{o}{=}\PY{o}{=} \PY{l+m+mi}{0} \PY{p}{)}
				\PY{n}{os} \PY{o}{\PYZlt{}}\PY{o}{\PYZlt{}} \PY{l+s+sc}{'\PYZbs{}t'}\PY{p}{;}
			\PY{n}{os} \PY{o}{\PYZlt{}}\PY{o}{\PYZlt{}} \PY{n}{setw}\PY{p}{(}\PY{l+m+mi}{20}\PY{p}{)} \PY{o}{\PYZlt{}}\PY{o}{\PYZlt{}} \PY{n}{mat}\PY{p}{(}\PY{n}{i}\PY{p}{,}\PY{n}{j}\PY{p}{)}\PY{p}{;}
			\PY{k}{if}\PY{p}{(}\PY{n}{j} \PY{o}{+} \PY{l+m+mi}{1} \PY{o}{\PYZlt{}} \PY{n}{mat}\PY{p}{.}\PY{n}{cols}\PY{p}{(}\PY{p}{)}\PY{p}{)}
				\PY{n}{os} \PY{o}{\PYZlt{}}\PY{o}{\PYZlt{}} \PY{l+s}{"}\PY{l+s}{ }\PY{l+s}{"}\PY{p}{;}
		\PY{p}{\PYZcb{}}
		\PY{k}{if}\PY{p}{(}\PY{n}{i} \PY{o}{+} \PY{l+m+mi}{1} \PY{o}{\PYZlt{}} \PY{n}{mat}\PY{p}{.}\PY{n}{rows}\PY{p}{(}\PY{p}{)}\PY{p}{)}
			\PY{n}{os} \PY{o}{\PYZlt{}}\PY{o}{\PYZlt{}} \PY{n}{endl}\PY{p}{;}
	\PY{p}{\PYZcb{}}
	\PY{n}{os} \PY{o}{\PYZlt{}}\PY{o}{\PYZlt{}} \PY{l+s}{"}\PY{l+s}{]}\PY{l+s}{"}\PY{p}{;}

	\PY{k}{return} \PY{n}{os}\PY{p}{;}
\PY{p}{\PYZcb{}}
\end{Verbatim}


\subsubsection{tp3-classif.cpp}
\begin{Verbatim}[commandchars=\\\{\}]
\PY{c+cp}{\PYZsh{}}\PY{c+cp}{include \PYZlt{}fstream\PYZgt{}}
\PY{c+cp}{\PYZsh{}}\PY{c+cp}{include \PYZlt{}iostream\PYZgt{}}
\PY{c+cp}{\PYZsh{}}\PY{c+cp}{include \PYZlt{}vector\PYZgt{}}
\PY{k}{using} \PY{k}{namespace} \PY{n}{std}\PY{p}{;}

\PY{c+cp}{\PYZsh{}}\PY{c+cp}{include "..}\PY{c+cp}{/}\PY{c+cp}{lib}\PY{c+cp}{/}\PY{c+cp}{commons.h"}
\PY{c+cp}{\PYZsh{}}\PY{c+cp}{include "cmd-args.h"}
\PY{c+cp}{\PYZsh{}}\PY{c+cp}{include "mmatrix.h"}
\PY{c+cp}{\PYZsh{}}\PY{c+cp}{include "data-io.h"}
\PY{c+cp}{\PYZsh{}}\PY{c+cp}{include "algorithms.h"}

\PY{c+cp}{\PYZsh{}}\PY{c+cp}{define		NOT\PYZus{}ENOUGH\PYZus{}PRINC\PYZus{}COMPONENTS(amnt,k)		\PYZbs{}}
\PY{c+cp}{				("El archivo de datos solo contiene información sobre las primeras " + int2str(amnt) + " componentes principales, por lo que no se puede realizar la clasificación con un valor de k: " + int2str(k))}

\PY{c+cm}{/*}
\PY{c+cm}{- programa clasificador [recibe: test set image/label files, k's, archivos-binarios-de-datos | devuelve: archivo de resultados]}
\PY{c+cm}{	- parsea los archivos tipo ubyte de mnist, cargando las fotos en una matriz y las labels en un vector}
\PY{c+cm}{	- genero archivo de resultados}
\PY{c+cm}{	- para cada archivo-binario-de-datos}
\PY{c+cm}{		- cargo V y el promedio de cada clase (o dígito)}
\PY{c+cm}{		- para cada k}
\PY{c+cm}{			- clasifico todas las imágenes, comparando contra los promedios}
\PY{c+cm}{			- cuento los aciertos}
\PY{c+cm}{			- anoto resultados en el archivo de resultados}
\PY{c+cm}{*/}

\PY{k+kt}{int} \PY{n}{main}\PY{p}{(}\PY{k+kt}{int} \PY{n}{argc}\PY{p}{,} \PY{k+kt}{char}\PY{o}{*}\PY{o}{*} \PY{n}{argv}\PY{p}{)}
\PY{p}{\PYZob{}}
	\PY{n}{CmdArgsClassif} \PY{n}{args} \PY{o}{=} \PY{n}{parse\PYZus{}cmd\PYZus{}args\PYZus{}classif}\PY{p}{(}\PY{n}{argc}\PY{p}{,} \PY{n}{argv}\PY{p}{)}\PY{p}{;}

	\PY{n}{MMatrix} \PY{n}{images}\PY{p}{;}
	\PY{n}{vector}\PY{o}{\PYZlt{}}\PY{k+kt}{int}\PY{o}{\PYZgt{}} \PY{n}{labels}\PY{p}{;}
	
	\PY{n}{BEGIN\PYZus{}TIMER}\PY{p}{(}\PY{p}{)}\PY{p}{;}
	\PY{n}{load\PYZus{}mnist\PYZus{}data}\PY{p}{(}\PY{n}{args}\PY{p}{.}\PY{n}{images\PYZus{}filename}\PY{p}{,} \PY{n}{args}\PY{p}{.}\PY{n}{labels\PYZus{}filename}\PY{p}{,} \PY{n}{images}\PY{p}{,} \PY{n}{labels}\PY{p}{)}\PY{p}{;}
	\PY{n}{PRINT\PYZus{}ON\PYZus{}VERBOSE}\PY{p}{(}\PY{l+s}{"}\PY{l+s}{Imágenes y etiquetas cargadas correctamente; total de imágenes: }\PY{l+s}{"} \PY{o}{+} \PY{n}{int2str}\PY{p}{(}\PY{n}{images}\PY{p}{.}\PY{n}{rows}\PY{p}{(}\PY{p}{)}\PY{p}{)} \PY{o}{+} \PY{l+s}{"}\PY{l+s}{ (}\PY{l+s}{"} \PY{o}{+} \PY{n}{int2str}\PY{p}{(}\PY{n}{MSECS\PYZus{}ELAPSED}\PY{p}{(}\PY{p}{)}\PY{p}{)} \PY{o}{+} \PY{l+s}{"}\PY{l+s}{ ms).}\PY{l+s}{"}\PY{p}{,} \PY{n}{args}\PY{p}{.}\PY{n}{verbose}\PY{p}{)}\PY{p}{;}

	\PY{n}{ofstream} \PY{n}{output\PYZus{}file}\PY{p}{;}
	\PY{n}{open\PYZus{}output\PYZus{}file}\PY{p}{(}\PY{n}{args}\PY{p}{.}\PY{n}{output\PYZus{}filename}\PY{p}{,}\PY{n}{output\PYZus{}file}\PY{p}{)}\PY{p}{;}

	\PY{n}{MMatrix} \PY{n}{V}\PY{p}{,} \PY{n}{avgs}\PY{p}{;}
	\PY{n}{vector}\PY{o}{\PYZlt{}}\PY{n}{string}\PY{o}{\PYZgt{}}\PY{o}{:}\PY{o}{:}\PY{n}{const\PYZus{}iterator} \PY{n}{data\PYZus{}file}\PY{p}{;}
	\PY{k}{for} \PY{p}{(}\PY{n}{data\PYZus{}file} \PY{o}{=} \PY{n}{args}\PY{p}{.}\PY{n}{data\PYZus{}files}\PY{p}{.}\PY{n}{begin}\PY{p}{(}\PY{p}{)}\PY{p}{;} \PY{n}{data\PYZus{}file} \PY{o}{!}\PY{o}{=} \PY{n}{args}\PY{p}{.}\PY{n}{data\PYZus{}files}\PY{p}{.}\PY{n}{end}\PY{p}{(}\PY{p}{)}\PY{p}{;} \PY{o}{+}\PY{o}{+}\PY{n}{data\PYZus{}file}\PY{p}{)}
	\PY{p}{\PYZob{}}
		\PY{k+kt}{double} \PY{n}{delta}\PY{p}{;}

		\PY{n}{RESET\PYZus{}TIMER}\PY{p}{(}\PY{p}{)}\PY{p}{;}
		\PY{n}{load\PYZus{}data\PYZus{}file}\PY{p}{(}\PY{o}{*}\PY{n}{data\PYZus{}file}\PY{p}{,} \PY{n}{delta}\PY{p}{,} \PY{n}{V}\PY{p}{,} \PY{n}{avgs}\PY{p}{)}\PY{p}{;}
		\PY{n}{PRINT\PYZus{}ON\PYZus{}VERBOSE}\PY{p}{(}\PY{l+s}{"}\PY{l+s}{Archivo de datos cargado, delta: }\PY{l+s}{"} \PY{o}{+} \PY{n}{double2str}\PY{p}{(}\PY{n}{delta}\PY{p}{)} \PY{o}{+} \PY{l+s}{"}\PY{l+s}{ (}\PY{l+s}{"} \PY{o}{+} \PY{n}{int2str}\PY{p}{(}\PY{n}{MSECS\PYZus{}ELAPSED}\PY{p}{(}\PY{p}{)}\PY{p}{)} \PY{o}{+} \PY{l+s}{"}\PY{l+s}{ ms).}\PY{l+s}{"}\PY{p}{,} \PY{n}{args}\PY{p}{.}\PY{n}{verbose}\PY{p}{)}\PY{p}{;}

		\PY{n}{vector}\PY{o}{\PYZlt{}}\PY{k+kt}{int}\PY{o}{\PYZgt{}}\PY{o}{:}\PY{o}{:}\PY{n}{const\PYZus{}iterator} \PY{n}{k}\PY{p}{;}
		\PY{k}{for} \PY{p}{(}\PY{n}{k} \PY{o}{=} \PY{n}{args}\PY{p}{.}\PY{n}{k\PYZus{}values}\PY{p}{.}\PY{n}{begin}\PY{p}{(}\PY{p}{)}\PY{p}{;} \PY{n}{k} \PY{o}{!}\PY{o}{=} \PY{n}{args}\PY{p}{.}\PY{n}{k\PYZus{}values}\PY{p}{.}\PY{n}{end}\PY{p}{(}\PY{p}{)}\PY{p}{;} \PY{o}{+}\PY{o}{+}\PY{n}{k}\PY{p}{)}
		\PY{p}{\PYZob{}}
			\PY{k}{if}\PY{p}{(}\PY{n}{V}\PY{p}{.}\PY{n}{cols}\PY{p}{(}\PY{p}{)} \PY{o}{\PYZlt{}} \PY{o}{*}\PY{n}{k}\PY{p}{)}
				\PY{n}{DISPLAY\PYZus{}ERROR\PYZus{}AND\PYZus{}EXIT}\PY{p}{(}\PY{n}{NOT\PYZus{}ENOUGH\PYZus{}PRINC\PYZus{}COMPONENTS}\PY{p}{(}\PY{n}{V}\PY{p}{.}\PY{n}{cols}\PY{p}{(}\PY{p}{)}\PY{p}{,} \PY{o}{*}\PY{n}{k}\PY{p}{)}\PY{p}{)}\PY{p}{;}

			\PY{n}{PRINT\PYZus{}ON\PYZus{}VERBOSE}\PY{p}{(}\PY{l+s}{"}\PY{l+s}{Realizando clasificación con k: }\PY{l+s}{"} \PY{o}{+} \PY{n}{int2str}\PY{p}{(}\PY{o}{*}\PY{n}{k}\PY{p}{)}\PY{p}{,} \PY{n}{args}\PY{p}{.}\PY{n}{verbose}\PY{p}{)}\PY{p}{;}

			\PY{n}{RESET\PYZus{}TIMER}\PY{p}{(}\PY{p}{)}\PY{p}{;}
			\PY{k+kt}{int} \PY{n}{hits} \PY{o}{=} \PY{n}{classify\PYZus{}images}\PY{p}{(}\PY{n}{images}\PY{p}{,} \PY{n}{labels}\PY{p}{,} \PY{n}{V}\PY{p}{,} \PY{n}{avgs}\PY{p}{,} \PY{o}{*}\PY{n}{k}\PY{p}{)}\PY{p}{;}
			\PY{n}{PRINT\PYZus{}ON\PYZus{}VERBOSE}\PY{p}{(}\PY{l+s}{"}\PY{l+s}{Clasificaciones acertadas: }\PY{l+s}{"} \PY{o}{+} \PY{n}{int2str}\PY{p}{(}\PY{n}{hits}\PY{p}{)} \PY{o}{+} \PY{l+s}{"}\PY{l+s}{ (}\PY{l+s}{"} \PY{o}{+} \PY{n}{int2str}\PY{p}{(}\PY{n}{MSECS\PYZus{}ELAPSED}\PY{p}{(}\PY{p}{)}\PY{p}{)} \PY{o}{+} \PY{l+s}{"}\PY{l+s}{ ms).}\PY{l+s}{"}\PY{p}{,} \PY{n}{args}\PY{p}{.}\PY{n}{verbose}\PY{p}{)}\PY{p}{;}
			\PY{n}{PRINT\PYZus{}ON\PYZus{}VERBOSE}\PY{p}{(}\PY{l+s}{"}\PY{l+s}{Porcentaje de aciertos: }\PY{l+s}{"} \PY{o}{+} \PY{n}{double2str}\PY{p}{(}\PY{n}{hits}\PY{o}{*}\PY{l+m+mf}{100.0}\PY{o}{/}\PY{n}{images}\PY{p}{.}\PY{n}{rows}\PY{p}{(}\PY{p}{)}\PY{p}{)} \PY{o}{+} \PY{l+s}{"}\PY{l+s}{\PYZpc{}.}\PY{l+s}{"}\PY{p}{,} \PY{n}{args}\PY{p}{.}\PY{n}{verbose}\PY{p}{)}\PY{p}{;}

			\PY{n}{write\PYZus{}results}\PY{p}{(}\PY{n}{output\PYZus{}file}\PY{p}{,} \PY{n}{delta}\PY{p}{,} \PY{o}{*}\PY{n}{k}\PY{p}{,} \PY{n}{args}\PY{p}{.}\PY{n}{images\PYZus{}filename}\PY{p}{,} \PY{n}{images}\PY{p}{.}\PY{n}{rows}\PY{p}{(}\PY{p}{)}\PY{p}{,} \PY{n}{hits}\PY{p}{)}\PY{p}{;}
		\PY{p}{\PYZcb{}}
	\PY{p}{\PYZcb{}}

	\PY{n}{close\PYZus{}output\PYZus{}file}\PY{p}{(}\PY{n}{output\PYZus{}file}\PY{p}{)}\PY{p}{;}

	\PY{n}{PRINT\PYZus{}ON\PYZus{}VERBOSE}\PY{p}{(}\PY{l+s}{"}\PY{l+s}{Clasificación realizada para todos los archivos de datos y valores de k. Terminando la ejecución...}\PY{l+s+se}{\PYZbs{}n}\PY{l+s}{"}\PY{p}{,} \PY{n}{args}\PY{p}{.}\PY{n}{verbose}\PY{p}{)}\PY{p}{;}

	\PY{k}{return} \PY{l+m+mi}{0}\PY{p}{;}
\PY{p}{\PYZcb{}}
\end{Verbatim}


\subsubsection{tp3-gen.cpp}
\begin{Verbatim}[commandchars=\\\{\}]
\PY{c+cp}{\PYZsh{}}\PY{c+cp}{include \PYZlt{}algorithm\PYZgt{}}
\PY{c+cp}{\PYZsh{}}\PY{c+cp}{include \PYZlt{}iostream\PYZgt{}}
\PY{c+cp}{\PYZsh{}}\PY{c+cp}{include \PYZlt{}vector\PYZgt{}}
\PY{k}{using} \PY{k}{namespace} \PY{n}{std}\PY{p}{;}

\PY{c+cp}{\PYZsh{}}\PY{c+cp}{include "..}\PY{c+cp}{/}\PY{c+cp}{lib}\PY{c+cp}{/}\PY{c+cp}{commons.h"}

\PY{c+cp}{\PYZsh{}}\PY{c+cp}{include "cmd-args.h"}
\PY{c+cp}{\PYZsh{}}\PY{c+cp}{include "mmatrix.h"}
\PY{c+cp}{\PYZsh{}}\PY{c+cp}{include "data-io.h"}
\PY{c+cp}{\PYZsh{}}\PY{c+cp}{include "algorithms.h"}

\PY{c+cm}{/*}
\PY{c+cm}{* - programa generador [recibe: training set image/label files, delta's | devuelve: archivos-binarios-de-datos (uno por cada delta)]}
\PY{c+cm}{* 	- parsea el archivo tipo ubyte de mnist, cargando las fotos en una matriz}
\PY{c+cm}{* 	- genera la matriz 'X', y computa 'Y' = Xt * X}
\PY{c+cm}{* 	- para cada delta}
\PY{c+cm}{* 		- computa los autovectores/autovalores de 'Y' (algoritmo QR)}
\PY{c+cm}{* 		- ordena los autovectores segun orden decreciente de abs(autovalor) -\PYZgt{} matriz V}
\PY{c+cm}{* 		- transformo las imagenes usando V}
\PY{c+cm}{* 		- computo el promedio de cada clase (o dígito)}
\PY{c+cm}{* 		- guarda en un archivo binario a la matriz V y los promedios}
\PY{c+cm}{*/}

\PY{k+kt}{int} \PY{n}{main}\PY{p}{(}\PY{k+kt}{int} \PY{n}{argc}\PY{p}{,} \PY{k+kt}{char}\PY{o}{*}\PY{o}{*} \PY{n}{argv}\PY{p}{)}
\PY{p}{\PYZob{}}
	\PY{n}{CmdArgsGen} \PY{n}{args} \PY{o}{=} \PY{n}{parse\PYZus{}cmd\PYZus{}args\PYZus{}gen}\PY{p}{(}\PY{n}{argc}\PY{p}{,} \PY{n}{argv}\PY{p}{)}\PY{p}{;}

	\PY{n}{PRINT\PYZus{}ON\PYZus{}VERBOSE}\PY{p}{(}\PY{l+s}{"}\PY{l+s}{Comenzando ejecución del programa.}\PY{l+s}{"}\PY{p}{,} \PY{n}{args}\PY{p}{.}\PY{n}{verbose}\PY{p}{)}\PY{p}{;}

	\PY{n}{MMatrix} \PY{n}{images}\PY{p}{;}
	\PY{n}{vector}\PY{o}{\PYZlt{}}\PY{k+kt}{int}\PY{o}{\PYZgt{}} \PY{n}{labels}\PY{p}{;}
	\PY{n}{MMatrix} \PY{n}{cov\PYZus{}mat}\PY{p}{;}

	\PY{n}{BEGIN\PYZus{}TIMER}\PY{p}{(}\PY{p}{)}\PY{p}{;}
	\PY{n}{load\PYZus{}mnist\PYZus{}data}\PY{p}{(}\PY{n}{args}\PY{p}{.}\PY{n}{images\PYZus{}filename}\PY{p}{,} \PY{n}{args}\PY{p}{.}\PY{n}{labels\PYZus{}filename}\PY{p}{,} \PY{n}{images}\PY{p}{,} \PY{n}{labels}\PY{p}{)}\PY{p}{;}
	\PY{n}{PRINT\PYZus{}ON\PYZus{}VERBOSE}\PY{p}{(}\PY{l+s}{"}\PY{l+s}{Imágenes y etiquetas cargadas correctamente; total de imágenes: }\PY{l+s}{"} \PY{o}{+} \PY{n}{int2str}\PY{p}{(}\PY{n}{images}\PY{p}{.}\PY{n}{rows}\PY{p}{(}\PY{p}{)}\PY{p}{)} \PY{o}{+} \PY{l+s}{"}\PY{l+s}{, (}\PY{l+s}{"} \PY{o}{+} \PY{n}{int2str}\PY{p}{(}\PY{n}{MSECS\PYZus{}ELAPSED}\PY{p}{(}\PY{p}{)}\PY{p}{)} \PY{o}{+} \PY{l+s}{"}\PY{l+s}{ ms).}\PY{l+s}{"}\PY{p}{,} \PY{n}{args}\PY{p}{.}\PY{n}{verbose}\PY{p}{)}\PY{p}{;}

	\PY{k}{if}\PY{p}{(}\PY{n}{args}\PY{p}{.}\PY{n}{compute\PYZus{}cov\PYZus{}mat}\PY{p}{)}
	\PY{p}{\PYZob{}}
		\PY{n}{PRINT\PYZus{}ON\PYZus{}VERBOSE}\PY{p}{(}\PY{l+s}{"}\PY{l+s}{Comenzando el cómputo de la matriz de covarianza}\PY{l+s}{"}\PY{p}{,} \PY{n}{args}\PY{p}{.}\PY{n}{verbose}\PY{p}{)}\PY{p}{;}
		
		\PY{n}{RESET\PYZus{}TIMER}\PY{p}{(}\PY{p}{)}\PY{p}{;}
		\PY{n}{compute\PYZus{}covariance\PYZus{}matrix}\PY{p}{(}\PY{n}{images}\PY{p}{,} \PY{n}{cov\PYZus{}mat}\PY{p}{)}\PY{p}{;}
		\PY{n}{PRINT\PYZus{}ON\PYZus{}VERBOSE}\PY{p}{(}\PY{l+s}{"}\PY{l+s}{Matriz de covarianza computada (}\PY{l+s}{"} \PY{o}{+} \PY{n}{int2str}\PY{p}{(}\PY{n}{MSECS\PYZus{}ELAPSED}\PY{p}{(}\PY{p}{)}\PY{p}{)} \PY{o}{+} \PY{l+s}{"}\PY{l+s}{ ms).}\PY{l+s}{"}\PY{p}{,} \PY{n}{args}\PY{p}{.}\PY{n}{verbose}\PY{p}{)}\PY{p}{;}

		\PY{n}{RESET\PYZus{}TIMER}\PY{p}{(}\PY{p}{)}\PY{p}{;}
		\PY{n}{string} \PY{n}{cov\PYZus{}mat\PYZus{}filename} \PY{o}{=} \PY{n}{write\PYZus{}covariance\PYZus{}matrix\PYZus{}to\PYZus{}file}\PY{p}{(}\PY{n}{args}\PY{p}{.}\PY{n}{images\PYZus{}filename}\PY{p}{,} \PY{n}{cov\PYZus{}mat}\PY{p}{)}\PY{p}{;}
		\PY{n}{PRINT\PYZus{}ON\PYZus{}VERBOSE}\PY{p}{(}\PY{l+s}{"}\PY{l+s}{Matriz de covarianza exportada al archivo }\PY{l+s}{"} \PY{o}{+} \PY{n}{cov\PYZus{}mat\PYZus{}filename} \PY{o}{+} \PY{l+s}{"}\PY{l+s}{ (}\PY{l+s}{"} \PY{o}{+} \PY{n}{int2str}\PY{p}{(}\PY{n}{MSECS\PYZus{}ELAPSED}\PY{p}{(}\PY{p}{)}\PY{p}{)} \PY{o}{+} \PY{l+s}{"}\PY{l+s}{ ms).}\PY{l+s}{"}\PY{p}{,} \PY{n}{args}\PY{p}{.}\PY{n}{verbose}\PY{p}{)}\PY{p}{;}

	\PY{p}{\PYZcb{}} \PY{k}{else} \PY{p}{\PYZob{}}

		\PY{n}{RESET\PYZus{}TIMER}\PY{p}{(}\PY{p}{)}\PY{p}{;}
		\PY{n}{load\PYZus{}covariance\PYZus{}matrix}\PY{p}{(}\PY{n}{args}\PY{p}{.}\PY{n}{cov\PYZus{}mat\PYZus{}filename}\PY{p}{,} \PY{n}{cov\PYZus{}mat}\PY{p}{)}\PY{p}{;}
		\PY{n}{PRINT\PYZus{}ON\PYZus{}VERBOSE}\PY{p}{(}\PY{l+s}{"}\PY{l+s}{Matriz de covarianza cargada correctamente (}\PY{l+s}{"} \PY{o}{+} \PY{n}{int2str}\PY{p}{(}\PY{n}{MSECS\PYZus{}ELAPSED}\PY{p}{(}\PY{p}{)}\PY{p}{)} \PY{o}{+} \PY{l+s}{"}\PY{l+s}{ ms).}\PY{l+s}{"}\PY{p}{,} \PY{n}{args}\PY{p}{.}\PY{n}{verbose}\PY{p}{)}\PY{p}{;}
	\PY{p}{\PYZcb{}}

	\PY{n}{vector}\PY{o}{\PYZlt{}}\PY{k+kt}{double}\PY{o}{\PYZgt{}}\PY{o}{:}\PY{o}{:}\PY{n}{const\PYZus{}iterator} \PY{n}{delta}\PY{p}{;}
	\PY{k}{for} \PY{p}{(}\PY{n}{delta} \PY{o}{=} \PY{n}{args}\PY{p}{.}\PY{n}{delta\PYZus{}values}\PY{p}{.}\PY{n}{begin}\PY{p}{(}\PY{p}{)}\PY{p}{;} \PY{n}{delta} \PY{o}{!}\PY{o}{=} \PY{n}{args}\PY{p}{.}\PY{n}{delta\PYZus{}values}\PY{p}{.}\PY{n}{end}\PY{p}{(}\PY{p}{)}\PY{p}{;} \PY{o}{+}\PY{o}{+}\PY{n}{delta}\PY{p}{)}
	\PY{p}{\PYZob{}}
		\PY{n}{PRINT\PYZus{}ON\PYZus{}VERBOSE}\PY{p}{(}\PY{l+s}{"}\PY{l+s}{Comenzando a computar los datos para delta = }\PY{l+s}{"} \PY{o}{+} \PY{n}{double2str}\PY{p}{(}\PY{o}{*}\PY{n}{delta}\PY{p}{)} \PY{o}{+} \PY{l+s}{"}\PY{l+s}{.}\PY{l+s}{"}\PY{p}{,} \PY{n}{args}\PY{p}{.}\PY{n}{verbose}\PY{p}{)}\PY{p}{;}

		\PY{n}{RESET\PYZus{}TIMER}\PY{p}{(}\PY{p}{)}\PY{p}{;}
	 	\PY{n}{MMatrix} \PY{n}{V} \PY{o}{=} \PY{n}{compute\PYZus{}transformation\PYZus{}matrix}\PY{p}{(}\PY{n}{cov\PYZus{}mat}\PY{p}{,} \PY{n}{args}\PY{p}{.}\PY{n}{number\PYZus{}of\PYZus{}components}\PY{p}{,} \PY{o}{*}\PY{n}{delta}\PY{p}{,} \PY{n}{args}\PY{p}{.}\PY{n}{verbose}\PY{p}{)}\PY{p}{;}
		\PY{n}{PRINT\PYZus{}ON\PYZus{}VERBOSE}\PY{p}{(}\PY{l+s}{"}\PY{l+s}{Matriz de transformación computada; cantidad de autovectores: }\PY{l+s}{"} \PY{o}{+} \PY{n}{int2str}\PY{p}{(}\PY{n}{args}\PY{p}{.}\PY{n}{number\PYZus{}of\PYZus{}components}\PY{p}{)} \PY{o}{+} \PY{l+s}{"}\PY{l+s}{, (}\PY{l+s}{"} \PY{o}{+} \PY{n}{int2str}\PY{p}{(}\PY{n}{MSECS\PYZus{}ELAPSED}\PY{p}{(}\PY{p}{)}\PY{p}{)} \PY{o}{+} \PY{l+s}{"}\PY{l+s}{ ms).}\PY{l+s}{"}\PY{p}{,} \PY{n}{args}\PY{p}{.}\PY{n}{verbose}\PY{p}{)}\PY{p}{;}

		\PY{n}{RESET\PYZus{}TIMER}\PY{p}{(}\PY{p}{)}\PY{p}{;}
	 	\PY{n}{MMatrix} \PY{n}{transf\PYZus{}images} \PY{o}{=} \PY{n}{transform\PYZus{}images}\PY{p}{(}\PY{n}{images}\PY{p}{,} \PY{n}{V}\PY{p}{)}\PY{p}{;}
	 	\PY{n}{PRINT\PYZus{}ON\PYZus{}VERBOSE}\PY{p}{(}\PY{l+s}{"}\PY{l+s}{Imágenes transformadas (}\PY{l+s}{"} \PY{o}{+} \PY{n}{int2str}\PY{p}{(}\PY{n}{MSECS\PYZus{}ELAPSED}\PY{p}{(}\PY{p}{)}\PY{p}{)} \PY{o}{+} \PY{l+s}{"}\PY{l+s}{ ms).}\PY{l+s}{"}\PY{p}{,} \PY{n}{args}\PY{p}{.}\PY{n}{verbose}\PY{p}{)}\PY{p}{;}
		
		\PY{n}{RESET\PYZus{}TIMER}\PY{p}{(}\PY{p}{)}\PY{p}{;}
	 	\PY{n}{MMatrix} \PY{n}{avgs} \PY{o}{=} \PY{n}{compute\PYZus{}average\PYZus{}by\PYZus{}digit}\PY{p}{(}\PY{n}{transf\PYZus{}images}\PY{p}{,} \PY{n}{labels}\PY{p}{)}\PY{p}{;}
	 	\PY{n}{PRINT\PYZus{}ON\PYZus{}VERBOSE}\PY{p}{(}\PY{l+s}{"}\PY{l+s}{Promedios según dígito computados (}\PY{l+s}{"} \PY{o}{+} \PY{n}{int2str}\PY{p}{(}\PY{n}{MSECS\PYZus{}ELAPSED}\PY{p}{(}\PY{p}{)}\PY{p}{)} \PY{o}{+} \PY{l+s}{"}\PY{l+s}{ ms).}\PY{l+s}{"}\PY{p}{,} \PY{n}{args}\PY{p}{.}\PY{n}{verbose}\PY{p}{)}\PY{p}{;}

	 	\PY{n}{RESET\PYZus{}TIMER}\PY{p}{(}\PY{p}{)}\PY{p}{;}
	 	\PY{n}{string} \PY{n}{output\PYZus{}data\PYZus{}filename} \PY{o}{=} \PY{n}{write\PYZus{}data\PYZus{}file}\PY{p}{(}\PY{o}{*}\PY{n}{delta}\PY{p}{,} \PY{n}{V}\PY{p}{,} \PY{n}{avgs}\PY{p}{)}\PY{p}{;}
	 	\PY{n}{PRINT\PYZus{}ON\PYZus{}VERBOSE}\PY{p}{(}\PY{l+s}{"}\PY{l+s}{Datos exportados al archivo }\PY{l+s}{"} \PY{o}{+} \PY{n}{output\PYZus{}data\PYZus{}filename} \PY{o}{+} \PY{l+s}{"}\PY{l+s}{ (}\PY{l+s}{"} \PY{o}{+} \PY{n}{int2str}\PY{p}{(}\PY{n}{MSECS\PYZus{}ELAPSED}\PY{p}{(}\PY{p}{)}\PY{p}{)} \PY{o}{+} \PY{l+s}{"}\PY{l+s}{ ms).}\PY{l+s}{"}\PY{p}{,} \PY{n}{args}\PY{p}{.}\PY{n}{verbose}\PY{p}{)}\PY{p}{;}
	\PY{p}{\PYZcb{}}

	\PY{n}{PRINT\PYZus{}ON\PYZus{}VERBOSE}\PY{p}{(}\PY{l+s}{"}\PY{l+s}{Datos computados para todos los valores de delta. Terminando la ejecución...}\PY{l+s+se}{\PYZbs{}n}\PY{l+s}{"}\PY{p}{,} \PY{n}{args}\PY{p}{.}\PY{n}{verbose}\PY{p}{)}\PY{p}{;}

	\PY{k}{return} \PY{l+m+mi}{0}\PY{p}{;}
\PY{p}{\PYZcb{}}
\end{Verbatim}






