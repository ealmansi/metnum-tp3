
Los resultados obtenidos corroboran que si se descartan demasiados componentes principales el método se vuelve ineficaz. Esto a la vez 
depende del valor de delta escogido. Cuanto mayor es delta, mas empeoran los resultados al quitar muchas componentes.

Al calcular los autovectores con mucha precisión se requieren pocas componentes para obtener los 
mismo resultados que utilizando menos precisión con muchas componentes. 
En concreto utilizando delta 0.1 se requieren como mínimo unas 200 componentes para el nivel óptimo de aciertos, mientras que con delta 0.01
se requieren solo 60.

Esto es una relación de compromiso teniendo en cuenta que cuanto menor es delta, se requiere mayor tiempo de ejecución.
Relación que se puede desempatar cuando se observa que según el delta hay valores máximos de aciertos , y que es mayor en cuanto el delta es menor.
Como se observa en el gráfico la diferencia entre los niveles de aciertos son importantes para los delta 0.1 y 0.01.

Luego para deltas menores a 0.01 el nivel de acierto no se ve incrementado considerablemente y teniendo en cuenta que si se incrementa el tiempo de
ejecución, significa que no tiene mucho sentido utilizar $deltas < 0.01$
