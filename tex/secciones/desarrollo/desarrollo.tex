\subsection{Base de datos MNIST}

de dónde sacamos las imagenes
como se conforma la base de datos (training set 60k, test set 10k, con labels)
características de las imágenes (centradas, sin ruido, en blanco y negro, de 48*48)

\subsection{Aplicabilidad del Método de las potencias}

el método viene al pelo porque:
	- es rápido comparado con QR
	- necesitamos solo algunos autovectores (no todos) y siempre los de mayor autovalor (en valor absoluto)
	- si bien el error se propaga (porque la deflación usa el autovalor que acabás de calcular), no necesitamos que sea super exacto porque la degradación del rendimiento debería ser parsimoniosa respecto a la degradación en la precisión [si moves un poco las componentes principales la clasificación no puede cambiar TAANTO, no?]

\subsubsection{Criterio de parada}

si el producto interno entre dos aproximaciones sucesivas es muy chiquito, entonces la aproximación no cambió mucho de dirección y decimos que "convergió"

\subsection{Cantidad de componentes principales computadas}

Calculamos solo hasta 350 autovectores porque más del 99\% de la varianza queda comprendida ahí [hice la cuenta en matlab, sum(autovalores(1:350)) / sum(autovalores) > 0.99], y encima el método de las potencias se vuelve cada vez más lento

\subsection{Experimentación y criterio de clasificación}

cómo funciona todo una vez que ya tenes computado todo
cómo decidís a que clase pertenece una foto nueva -> (la transformás y comparás contra el promedio de las transformadas de cada dígito y te quedás con el más cercano en norma 2)