\documentclass[a4paper]{article}

\usepackage[spanish]{babel}
\usepackage[utf8]{inputenc}
\usepackage{amsmath, amssymb, graphics, setspace, framed, graphicx, listings}
\usepackage{caratula} %Para armar el cuadro de integrantes
\usepackage{booktabs} % To thicken table lines
\usepackage{xcolor}

\lstset { %
    language=C++,
    basicstyle=\footnotesize,% basic font setting
}
\usepackage{fancyvrb}
\usepackage{color}

\input{macros}

\newcommand{\real}{\mathbb{R}}
\newcommand{\nat}{\mathbb{N}}
\newcommand{\eme}{\mathcal{M}_X}
\newcommand{\emeh}{\mathcal{M}_X'}
\newcommand{\ere}{\mathcal{R}_X}
\newcommand{\ereh}{\mathcal{R}_X'}

\begin{document}


%---------------------

\integrante{Almansi, Emilio Guido}{674/12}{ealmansi@gmail.com}
\integrante{Vasapollo, Martin}{345/09}{martin.vsa@gmail.com}

\def\Materia{Métodos numéricos}
\cuatrimestre{1}
\anio{2013}
\def\Titulo{\LARGE Trabajo Práctico 3: OCR + SVD}
\def\Grupo{ (borrame) }

\def\Fecha{21 de junio de 2013}

%----- CARATULA -----%

\thispagestyle{empty}

\begin{center}
	\includegraphics[scale = 0.25]{imagenes/logo_uba.jpg}
\end{center}

\vspace{5mm}

\begin{center}
	{\textbf{\large UNIVERSIDAD DE BUENOS AIRES}}\\[1.5em]
	{\textbf{\large Departamento de Computaci\'{o}n}}\\[1.5em]
    {\textbf{\large Facultad de Ciencias Exactas y Naturales}}\\
    \vspace{20mm}
    {\LARGE\textbf{\Materia}}\\[1em]    
    \vspace{5mm}
    {\LARGE\textbf{\cuatrimestreLindo de \elanio}}\\
    \vspace{15mm}
    {\Large \textbf{\Titulo}}\\[1em]
    \vspace{15mm}
    {\textbf{\Large \Fecha}}\\ 
     \vspace{5mm}
   \textbf{\begin{center}\Large Resumen\end{center}}
   Se implementó un sistema de reconocimiento óptico de caracteres (OCR, según sus siglas en inglés), basado en el análisis de componentes principales como método de reducción de dimensionalidad.
   Los datos de entrenamiento y evaluación se tomaron de una base de datos con imágenes bien condicionadas y en escala de grises, conteniendo los dígitos del 0 al 9. La extracción de componentes principales del conjunto de datos se realizó por medio de la descomposición en valores singulares (SVD) de la matriz de covarianza, para lo cual se aproximaron los autovectores de dicha matriz mediante un método basado en el Método de las potencias.
   Luego, el reconocimiento de nuevas imágenes se realizó mediante su clasificación en el espacio de dimensión reducida determinado por las componentes principales. De especial interés fue el estudio del rendimiento del sistema ante distintas precisiones en el cálculo de autovectores, y la cantidad de componentes utilizada para la clasificación.     \\ \vspace{4mm}
   \textbf{Términos Clave :} OCR, SVD, reducción de la dimensión, Método de las potencias

   \vspace{10mm}
    \textbf{\tablaints}

    \end{center}



%-----------------------

\newpage

\tableofcontents

\medskip

\newpage
\graphicspath{{graficos/}}

\section{Introducción teórica}
\subsection{OCR y reducción de dimensionalidad}
\label{intro:ocr}

El reconocimiento óptico de caracteres (\emph{Optical Character Regonition}) es la interpretación automatizada de texto en formato de imágen, y la conversión mecánica de texto manuscrito o impreso a versión digital. Una variante más restringida del problema se enfoca en el reconocimiento de dígitos manuscritos, siendo ese el foco de este trabajo.

El enfoque utilizado para resolver el problema consiste, en primer lugar, en considerar a cada imagen de un dígito como una instancia de observación sobre \N variables, donde \N es la cantidad de píxeles que la componen. De esta forma, un conjunto de \M imágenes se puede interpretar como un conjunto de datos con \M muestras sobre \N variables.

Por ejemplo, si se tiene un conjunto de imágenes de tamaño $\n * \n = \N$ en escala de grises de 8 bits, se considera a cada píxel como una variable que toma valores entre 0 y 255, y cada una de las imágenes será una muestra con una observación para cada una de las \N variables.

Ahora bien, bajo esta interpretación, es evidente que no todas las variables tienen la misma relevancia a la hora de diferenciar una muestra de la otra; aquellos píxeles que no pertenezcan a la forma habitual de ninguno de los diez dígitos, tendrán valores similares o idénticos en todas las muestras. Por el contrario, algunos píxeles se activarán para ciertos dígitos y no para los demás (o lo que es igual, esas variables tomarán valores distintos en las muestras de unos u otros  dígitos), permitiendo caracterizar y distinguir distintas clases dentro de las observaciones.

El análisis de componentes principales extiende este concepto, permitiendo hallar una representación de los datos donde las distintas variables se organizan jerárquicamente según su relevancia. En concreto, permite hallar un sistema de coordenadas ortogonales formadas por combinaciones lineales de las originales, de forma tal que al ver los datos en este sistema, las nuevas variables (que ya no serán píxeles individuales, sino características comprendiendo a varios de ellos) queden ordenadas según la magnitud de sus varianzas. Los ejes de este sistema son lo que se conoce como \emph{componentes principales}.

Adicionalmente, es habitual hallar en datos reales con cierto grado de redundancia en su contenido, que la mayor parte de la varianza total del sistema se concentra en unas pocas componentes principales; esto permite descartar aquellas menos relevantes, reduciendo la dimensión de los datos con mínima pérdida de información, y exponiendo sus características más significativas.

\subsection{Descomposición SVD y relación con las componentes principales}

La descomposición en valores singulares (\emph{Singular Value Decomposition}) de una matriz \decMat{\X}{\M}{\N} esta compuesta por matrices \decMat{\U}{\M}{\M}, \decMat{\Sig}{\M}{\N} y \decMat{\V}{\N}{\N} cumpliendo:

$$\X = \U \Sig \V^{t}$$

donde \U y \V son ortogonales, \Sig es diagonal, y sus elementos $\sigma_i$ son no negativos y se encuentran ordenados decrecientemente.

Esta descomposición es útil en este caso ya que, de la teoría de del análisis de componentes principales, se desprende que el cambio de coordenadas entre el sistema original de los datos y el de sus componentes principales queda determinado por la matriz $\V^{t}$ de la descomposición en valores singulares de la matriz de covarianza de los datos.

En particular, como la matriz de covarianza es simétrica y semi-definida positiva por construcción \footnote{Si $A$ es la matriz de datos, $\A^{t} * \A = (\A^{t} * \A)^{t}$ y $\forall x \neq 0$, $x^t * A^t * A * x = \left \| A*x \right \|_2 \geq 0$}, también es diagonalizable y su descomposición SVD se puede tomar de forma equivalente a su descomposición $PDP^{-1}$.\footnote{Esta es la descomposición obtenida al diagonalizar la matriz.}

Por lo tanto, para obtener $\V^{t}$ es necesario calcular los autovectores de la matriz de covarianza de los datos. Sin embargo, si se quiere eliminar componentes que aporten poca información, reduciendo la dimensión de los datos a sus primeras $k$ componentes principales, es suficiente computar únicamente los $k$ autovectores dominantes; es decir, aquellos cuyos autovalores sean mayores en módulo.


\subsection{Método de las potencias y deflación}

EL algoritmo comienza con un vector de norma 1 aleatorio $b_0$. El método utiliza la siguiente iteración: 

$$b_{k+1} = \frac{Ab_k}{\|Ab_k\|}$$

donde A es la matriz de entrada.

Bajo la suposiciones: 
\begin{itemize}
  \item A tiene un autovalor que es estrictamente mayor en magnitud a los otro autovalores
  \item El vector tiene una componente distinta de cero en la dirección del autovector asociado al autovalor principal.
\end{itemize}

Entonces una subsucesión de $b_k$ converge al autovector asociado al autovalor dominante.
 
 
\subsection{Deflación}
Necesitamos encontrar $k$ autovectores, pero el método de la potencia solo permite encontrar el dominante. Para resolver este problema, 
luego de haber encontrado un un autovector, se realiza una transformación llamada deflación,
que elimina el autovalor dominante de forma que podamos encontrar el siguiente.
 
El algoritmo es el siguiente: \\

Para cada $i=1$ a $k$: \\
\indent \indent$v_i \leftarrow MetodoPotencia(A)$ \\
\indent \indent$\lambda_i \leftarrow (v_i^t A v_i) / ( v_i^t v_i)$\\
\indent \indent$A \leftarrow A - \lambda_i v_i v_i^t$ \\

 
\vspace{0.5cm}

\clearpage

\section{Desarrollo}
\subsection{Base de datos MNIST}

Para implementar y poner a prueba el sistema se utilizó la base de datos MNIST de dígitos manuscritos\footnote{Se puede obtener desde el sitio http://yann.lecun.com/exdb/mnist/.}. Esta consta de un conjunto de entrenamiento y un conjunto de prueba de 60.000 y 10.000 imágenes, respectivamente. Cada una de ellas viene apareada con una etiquetada indicando el dígito que contiene.

Las imágenes tienen un tamaño estandarizado de $28 \times 28$ píxeles, se encuentran libres de ruido, debidamente centradas, y tienen un formato color en escala de grises de 8 bits. Estas características permiten enfocar el estudio en el reconocimiento de los dígitos, sin tener que realizar preprocesamiento y condicionamiento de las imágenes, lo cual suele ser una etapa relevante en el problema de OCR.

Según el enfoque propuesto en la sección \ref{intro:ocr}, el conjunto de entrenamiento se puede considerar como una matriz de datos \decMat{\X}{\M}{\N} donde $\M = 60000$ y $\N = 28 * 28 = 784$. De esta forma, la matriz de covarianza será \decMat{\frac{1}{\N - 1} \X^t \X}{\N}{\N}.

\subsection{Aplicabilidad del Método de las potencias}

Para utilizar el Método de las potencias combinado con deflación fue necesario verificar primero que se cumplieran las hipótesis sobre las cuales se basan. Por un lado, la matriz de covarianza es simétrica por construcción, por lo cual sus autovectores serán necesariamente ortogonales y por lo tanto el proceso de deflación funciona correctamente. Por otro lado, la condición de que los autovalores fueran distintos en módulo no se pudo demostrar de forma teórica ya que es posible generar un conjunto de datos donde esto no se cumpla. Sin embargo, se corroboró empíricamente que esta condición efectivamente se verifica en el rango de autovalores de mayor valor absoluto, que son los de interés para esta aplicación (en contraposición a las direcciones principales de menor relevancia, las cuales se encuentran asociadas a autovalores muy similares y cercanos a cero).

Las características del método lo hicieron particularmente apto para esta aplicación, ya que solo fue necesario computar un subconjunto de autovectores dominantes de la matriz de covarianza. Por esta razón, fue posible aplicar el método tantas veces como componentes principales se desearan, intercalando el proceso de deflación entre distintas iteraciones. Este esquema es inherentemente más eficiente que otros métodos para calcular todos los autovectores de la matriz de forma simultánea, como el algoritmo QR en su versión ordinaria.

\subsubsection{Criterio de parada}

Si llamamos $v_i$ con $i = 0, 1, 2 ...$ a la aproximación generada por la i-ésima iteración, el criterio se parada consiste en considerar que el algoritmo ha convergido cuando se cumple:
$$ 1\, - \left | <v_{i-1}, v_{i}> \right | \, \leq \delta $$
donde $\delt \in \left ( 0, 1 \right )$ es el parámetro de la implementacion que determina la precisión en el cálculo de los autovectores.

Dado que ambos vectores son unitarios por construcción, el valor absoluto de su producto interno es $\leq 1$ \footnote{Por la desigualdad de Cauchy–Schwarz-Bunyakovsky}, y será más cercano a $1$ cuando la dirección de $v_{i-1}$ y de $v_{i}$ sean más próximas entre sí. De esta forma, cuanto mayor sea \delt, más restrictiva será la condición, requiriendo que los vectores sean colineales para $\delt = 1$. Esto significa que se terminará el proceso iterativo cuando sucesivas aproximaciones no perciban grandes variaciones en su dirección.

\subsection{Cantidad de componentes principales computadas}

Para determinar la cantidad de dimensiones apropiada 
Calculamos solo hasta 350 autovectores porque más del 99\% de la varianza queda comprendida ahí [hice la cuenta en matlab, sum(autovalores(1:350)) / sum(autovalores) > 0.99], y encima el método de las potencias se vuelve cada vez más lento

\subsection{Experimentación y criterio de clasificación}

cómo funciona todo una vez que ya tenes computado todo
cómo decidís a que clase pertenece una foto nueva -> (la transformás y comparás contra el promedio de las transformadas de cada dígito y te quedás con el más cercano en norma 2)
\vspace{0.5cm}

\clearpage

\section{Resultados}


En la figura (\ref{fig:aciertos-vs-k}), se muestra la cantidad de aciertos obtenidos para distintos k (componentes principales). 
Cada curva representa el delta utilizado en el criterio de parada.


\begin{figure}[h]
\begin{center}
  \includegraphics[scale=0.8]{imagenes/aciertos.pdf}
\end{center}
\caption{Gráfico de aciertos vs cantidad de iteraciones.}
\label{fig:aciertos-vs-k}
\end{figure}

En la figura (\ref{fig:tiempo-vs-k}), se muestra el tiempo de la ejecución del programa son contar el tiempo que tarda en
cargar la matriz de covarianza, para distintos valores de delta.

\begin{figure}[h]
\begin{center}
  \includegraphics[scale=0.8]{imagenes/tiempos.pdf}
\end{center}
\caption{Gráfico de tiempo de ejecución vs delta.}
\label{fig:tiempo-vs-k}
\end{figure}
\vspace{0.5cm}

\clearpage

\section{Discusión}

Los resultados obtenidos corroboran que si se descartan demasiados componentes principales el método se vuelve ineficaz. Esto a la vez 
depende del valor de delta escogido. Cuanto mayor es delta, mas empeoran los resultados al quitar muchas componentes.

Al calcular los autovectores con mucha precisión se requieren pocas componentes para obtener los 
mismo resultados que utilizando menos precisión con muchas componentes. 
En concreto utilizando delta 0.1 se requieren como mínimo unas 200 componentes para el nivel óptimo de aciertos, mientras que con delta 0.01
se requieren solo 60.

Esto es una relación de compromiso teniendo en cuenta que cuanto menor es delta, se requiere mayor tiempo de ejecución.
Relación que se puede desempatar cuando se observa que según el delta hay valores máximos de aciertos , y que es mayor en cuanto el delta es menor.
Como se observa en el gráfico la diferencia entre los niveles de aciertos son importantes para los delta 0.1 y 0.01.

Luego para deltas menores a 0.01 el nivel de acierto no se ve incrementado considerablemente y teniendo en cuenta que si se incrementa el tiempo de
ejecución, significa que no tiene mucho sentido utilizar $deltas < 0.01$

\vspace{0.5cm}

\clearpage

\section{Conclusiones}
El problema del reconocimiento óptico de de dígitos fue resuelto satisfactoriamente ya que se encontraron métodos 
confiables, automatizables y precisos encontrado efectividad en el orden del 80\% para los 10000 casos de test utilizados.

Dentro de las técnicas utilizadas para encontrar los autovectores, se exploraron la resolución por medio de factorización QR y 
método de la potencia. Encontrando el primero altamente ineficiente para nuestra aplicación, clasificable como inutilizable en la practica.
Utilizando el método de la potencia encontramos resultados de precisión , con tiempos de ejecución aceptables.

Queda abierta la generalización del método al conjuntos de todos lo caracteres del español, en el cual es posible que se obtengan también 
buenos resultados.



\vspace{0.5cm}

\clearpage

\section{Apéndices}
\input{secciones/apendices/pygmentize.tex}

\subsection{Apéndice A: Enunciado del TP}

\subsection{Apéndice B}

\subsection{Apéndice C}

% \subsubsection{MMMatrix.cpp}
% \input{secciones/codigo/MMMatrix.tex}

\vspace{0.5cm}

\clearpage

\section{Referencias}
\begin{enumerate}
	% \item Serie de Cosenos de Fourier - Wolfram MathWorld \\
	% link: \emph{http://mathworld.wolfram.com/FourierCosineSeries.html} \\
	% \item USC-SIPI Image Database - University of California, Signal and Image Processing Institute \\
	% link: \emph{http://sipi.usc.edu/database/} \\
	% \item Teorema Central del Limite - Blaiotta, Delieutraz (2004) \\
	% link: \emph{http://www.itescam.edu.mx/principal/sylabus/fpdb/recursos/r75990.PDF} \\
\end{enumerate}
\vspace{0.5cm}

\end{document}
